\chapter{Texts}\label{app:2}
\section{World War 2}\label{app:2:worldwar}
by Kalina Sarak  (extract, edited by Saror Aduna) \\
\ea
\gll  I  me  amis-ar-em-ik-omkun  iinan  aasa  iinan-pa  fan  ekap-emi  paran-em-yi-omak-e-k. \\
1p.\textsc{unm}  not  knowledge-\textsc{inch}-\textsc{ss}.\textsc{sim}-be-1s/p.\textsc{ds}  sky  canoe sky-\textsc{loc}  here  come-\textsc{ss}.\textsc{sim}  rumble-\textsc{ss}.\textsc{sim}-wander-\textsc{distr}/\textsc{pl}-\textsc{pa}-3s \\
\glt ‘We were not aware, when airplanes came here rumbling on the sky.’
\z


\ea
\gll  I  naap  koora=pa  ik-e-mik,  koora=pa  ik-e-mik. \\
1p.\textsc{unm}  thus  house-\textsc{loc}  be-\textsc{pa}-1/3p  house-\textsc{loc}  be-\textsc{pa}-1/3p \\
\glt ‘We were in the house like that, we were in the house.’
\z


\ea
\gll  Aria  yena  mua  pun  irak-owa  kerer-owa  epa weeser-em-ik-eya  iirar-iwkin  owowa  ekap-o-k, o  amia  mua=pa  ik-ok. \\
alright  1s.\textsc{gen}  man  too  fight-\textsc{nmz}  appear-\textsc{nmz}  time finish-\textsc{ss}.\textsc{sim}-be-2/3s.\textsc{ds}  dismiss-2/3p.\textsc{ds}  village  come-\textsc{pa}-3s 3s.\textsc{unm}  weapon  man-\textsc{loc}  be-\textsc{ss} \\
\glt ‘Alright my husband too, when the fighting time was getting close, was dismissed and came to the village, he having been a policeman.’ \\
\z


\ea
\gll  Maak-e-mik,  “No  ikiw-e,  irak-owa  maneka  fan-e-k a,”  ne  ekap-o-k. \\
tell-\textsc{pa}-1/3p  2s.\textsc{unm}  go-\textsc{imp}.2s  fight-\textsc{nmz}  big  be.here-\textsc{pa}-3s \textsc{intj}  \textsc{add}  come-\textsc{pa}-3s \\
\glt ‘They told him, “Go, the big war is here,” and he came.’ \\
\z


\ea
\gll  Ekap-emi  yo  efa  aaw-o-k. \\
come-\textsc{ss}.\textsc{sim}  1s.\textsc{unm}  1s.\textsc{acc}  take-\textsc{pa}-3s \\
\glt ‘He came and married me.’ \\
\z


\ea
\gll  Yo  efa  aaw-eya  i  owawiya  ik-omkun   Yaapan=ena  Wewak  kame=pa  nan  ir-a-mik. \\
1s.\textsc{unm}  1s.\textsc{acc}  take-2/3s.\textsc{ds}  1p.\textsc{unm}  together  be-1s/p.\textsc{ds}  Japan-\textsc{tp}  Wewak  side-\textsc{loc}  there  come-\textsc{pa}-1/3p \\
\glt ‘He married me and when we were together the Japanese came from the direction of Wewak there.’ \\
\z


\ea
\gll  Ne  i  Emer  Era=pa  nan  ik-e-mik,  ne  oos   onaiya  Nemuru  lul  nan  ir-am-ik-e-mik. \\
\textsc{add}  1p.\textsc{unm}  sago  road-\textsc{loc}  there  be-\textsc{pa}-1/3p  \textsc{add}  horse with  Nemuru  black.sand  there  come-\textsc{ss}.\textsc{sim}-be-\textsc{pa}-1/3p \\
\glt ‘And we were there in Sago Road (hamlet), and they were coming with horses there along Nemuru black sand beach.’ \\
\z


\ea
\gll  Ir-am-ik-aiwkin  yena  mua  Sarak=ke  wia  uruf-ap  ma-e-k,  “Yaapan  nan-e-mik. \\
come-\textsc{ss}.\textsc{sim}-be-2/3p.\textsc{ds}  1s.\textsc{gen}  man  Sarak-\textsc{cf}  3p.\textsc{acc}   see-\textsc{ss.seq}  say-\textsc{pa}-3s  Japan  come.there-\textsc{pa}-1/3p \\
\glt ‘As they were coming my husband Sarak saw them and said, “The Japanese have come there.’ \\
\z


\ea
\gll  Saa=iw  ir-am-ika-i-mik,  oos  ono-onaiya       ir-am-ika-i-mik.” \\
sand-\textsc{inst}  come-\textsc{ss}.\textsc{sim}-be-Np-\textsc{pr}.1/3p  horse  \textsc{rdp}-with come-\textsc{ss}.\textsc{sim}-be-Np-\textsc{pr}.1/3p \\
\glt ‘They are coming along the beach, they are coming with many horses.”’ \\
\z


\ea
\gll  Ne  i  owowa  ir-a-mik. \\
\textsc{add}  1p.\textsc{unm}  village  come-\textsc{pa}-1/3p \\
\glt ‘And we came to the village.’ \\
\z


\ea
\gll  Owowa=pa  fan  yia  maak-e-mik,  “Yaapan  fan  ikiw-e-mik,       oos  onaiya  Madang  ikiw-e-mik. \\
village-\textsc{loc}  here  1p.\textsc{acc}  tell-\textsc{pa}-1/3p  Japan  here  go-\textsc{pa}-1/3p horse  with  Madang  go-\textsc{pa}-1/3p \\
\glt ‘Here in the village they told us, “The Japanese went this way, they went to Madang with horses.’ \\
\z


\ea
\gll  Wi  sawur  nain  ir-ami  fan  yiar  pok-a-mik.” \\
3p.\textsc{unm}  spirit{{\footnotemark}} that1  come-\textsc{ss}.\textsc{sim}  here  1p.\textsc{dat}  sit-\textsc{pa}-1/3p \\
\glt ‘Those spirits came and sat here with us.’ \\
\footnotetext{ \textit{Inasina} ‘a bush spirit’ is commonly used for white-skinned outsiders, \textit{sawur} ‘spirit of long-ago dead’ only rarely.} \z

\ea
\gll  Wi  Yaapan  nain  naap  wia  ma-e-mik,  “Sawur=ke       fan  ikiw-e-mik.” \\
3p.\textsc{unm}  Japan  that1  thus  3p.\textsc{acc}  say-\textsc{pa}-1/3p  spirit-\textsc{cf} here  go-\textsc{pa}-1/3p \\
\glt ‘We said like that about the Japanese, “The spirits went this way.” ’ \\
\z


\ea
\gll  Aria  naap  ik-ok  i  baurar-e-mik. \\
Alright  thus  be-\textsc{ss}  1p.\textsc{unm}  flee-\textsc{pa}-1/3p \\
\glt ‘Alright we were like that and (then) we ran away.’ \\
\z


\ea
\gll  Muakura=ke  ma-e-k,  ``Ikoka  Yaapan=ke  ekap-emi  ni       emeria  unowa  fain  nia  aaw-urum-i-kuan.” \\
Muakura-\textsc{cf}  say-\textsc{pa}-3s  later  Japan-\textsc{cf}  come-\textsc{ss}.\textsc{sim}  2p.\textsc{unm} woman  many  this  2p.\textsc{acc}  take-\textsc{distr}/A-Np-\textsc{fu}.3p \\
\glt ‘It was Muakura who said, “Later the Japanese will come and take all of you women.” ’ \\
\z


\ea
\gll  I  uura  maa  unowa  op-ap  baurar-ep  koka       ikiw-e-mik. \\
1p.\textsc{unm}  night  thing  many  grab-\textsc{ss.seq}  flee-\textsc{ss.seq}  jungle go-\textsc{pa}-1/3p \\
\glt ‘At night we grabbed our belongings and fled and went into the jungle.’ \\
\z


\ea
\gll  Iki(w-e)p  koka=pa  nan  in-em-ik-e-mik,       ikoka  Yaapan  me  yia  aaw-uk  na-ep. \\
go-\textsc{ss.seq}  jungle-\textsc{loc}  there  sleep-\textsc{ss}.\textsc{sim}-be-\textsc{pa}-1/3p later  Japan  not  1p.\textsc{acc}  take-\textsc{imp}.3p  say-\textsc{ss.seq} \\
\glt ‘We went and kept sleeping in the jungle so that the Japanese would not take us.’ \\
\z


\ea
\gll  “Ni  ikoka  Yaapan=ke  emeria  niar  aaw-emi   ni  umakuna  nia  puuk-i-kuan.” \\
2p.\textsc{unm}  later  Japan-\textsc{cf}  woman  2p.\textsc{dat}  take-\textsc{ss}.\textsc{sim} 2p.\textsc{unm}  neck  2p.\textsc{acc}  cut-Np-\textsc{fu}.3p \\
\glt ‘ “Later the Japanese will take your wives and cut your necks.” ’ \\
\z


\ea
\gll  I  nain  kema  tooton-ep,  ikoka  Yaapan  me  yia       aaw-uk  na-ep  uura  kuisow  baurar-e-mik, maa  unowa  owowa=pa  feeke  ika-eya. \\
1p.\textsc{unm}  that1  liver  fear-\textsc{ss.seq}  later  Japan  not  1p.\textsc{acc} take-\textsc{imp}.3p  say-\textsc{ss.seq}  night  one  flee-\textsc{pa}-1/3 thing  many  village-\textsc{loc}  here.\textsc{cf}  be-2/3s.\textsc{ds}\\
\glt ‘We were afraid of that, (and) so that the Japanese would not take us we ran away right that night, (with) many belongings staying here in the village.’ \\
\z


\ea
\gll  Aakisa  Malala  suule  ik-ua,  i  naap  ikiw-ep yiena  manina  on-a-mik  nan  ik-e-mik. \\
now  Malala  school  be-\textsc{pa}.3s  1p.\textsc{unm}  thus  go-\textsc{ss.seq} 1p.\textsc{gen}  garden  make-\textsc{pa}-1/3p  there  be-\textsc{pa}-1/3p \\
\glt ‘Now there is the Malala school, we went like that and stayed where we (had) made our gardens.’ \\
\z


\ea
\gll  I  miiwa  nan  aruf-am-ik-e-mik,  ne  iki(w-e)p  nan  in-em-ik-e-mik. \\
1p.\textsc{unm}  ground  there  hit-\textsc{ss}.\textsc{sim}-be-\textsc{pa}-1/3p  \textsc{add}  go-\textsc{ss.seq} there  sleep-\textsc{ss}.\textsc{sim}-be-\textsc{pa}-1/3p \\
\glt ‘We kept tilling the soil there, and we went and kept sleeping there.’ \\
\z


\ea
\gll  Emeria  teeria  unow=iya  baurar-ep  koka   ikiw-urum-e-mik,  kuisow  owowa=pa=ko  me  ik-ua. \\
woman  group  many-\textsc{com}  flee-\textsc{ss.seq}  jungle go-\textsc{distr}/A-\textsc{pa}-1/3p  one  village-\textsc{loc}-\textsc{if}  not  be-\textsc{pa}.3s\\
\glt ‘The whole group of women fled into the jungle, not even one stayed in the village.’ \\
\z


\ea
\gll  Mua  muutiw  owowa=pa  ik-e-mik. \\
man  only  village-\textsc{loc}  be-\textsc{pa}-1/3p \\
\glt ‘Only men stayed in the village.’ \\
\z


\ea
\gll  Mua  kuisow,  Muakura=ke  i  yia  p-ikiw-o-k. \\
man  one  Muakura-\textsc{cf}  1p.\textsc{unm}  1p.\textsc{acc}  Bpx-go-\textsc{pa}-3s \\
\glt ‘One man, Muakura, was the one who took us (to the jungle).’ \\
\z


\ea
\gll  “Karu-eka,  ikoka  Yaapan  ir-ami  ni  nia  aaw-emi  yo  efa  ifakim-i-kuan,”  na-eya i  karu-em-ik-e-mik. \\
run-\textsc{imp}.2p  later  Japan  come-\textsc{ss}.\textsc{sim}  2p.\textsc{unm}  2p.\textsc{acc}   take-\textsc{ss}.\textsc{sim}  1s.\textsc{unm}  1s.\textsc{acc}  kill-Np-\textsc{fu}.3p  say-2/3s.\textsc{ds} 1s.\textsc{unm}  run-\textsc{ss}.\textsc{sim}-be-\textsc{pa}-1/3p \\
\glt ‘ “Run, (otherwise) later the Japanese will come and take you and kill me,” he said, and we kept running.’ \\
\z


\ea
\gll  Irak-owa  maneka  ewur  me  imen-ar-e-k,      wi  Yaapan  naap  kuisow=iw  ekap-em-ik-e-mik, owowa  yiar  kuuf-owa. \\  
fight-\textsc{nmz}  big  quickly  not  find-\textsc{inch}-\textsc{pa}-3s 3p.\textsc{unm}  Japan  thus  one-\textsc{lim}  come-\textsc{ss}.\textsc{sim}-be-\textsc{pa}-1/3p village  1p.\textsc{dat}  see-\textsc{nmz} \\ 
\glt ‘The big fight did not start quickly, the Japanese kept coming one by one to see/look at our village(s).’ \\
\z


\ea
\gll  Ekap-ep,  ekap-ep,  aakisa  fan  unowa  Wewak=pa  nan  urup-e-mik. \\
come-\textsc{ss.seq}  come-\textsc{ss.seq}  now  here  many  Wewak-\textsc{loc} there  ascend-\textsc{pa}-1/3p \\
\glt ‘The came and came, and just now many landed (lit: came up) there in Wewak.’ \\
\z


\ea
\gll  Ne  ekap-ep  Numbia=pa  nan  urup-e-mik. \\
\textsc{add}  come{}-\textsc{ss.seq}  Numbia-\textsc{loc}  there  ascend-\textsc{pa}-1/3p \\
\glt ‘They came and landed there at Numbia.’ \\
\z


\ea
\gll  Maa  unowa  ifer  aasa=ke  p-urup-eya  miiw-aasa=ke fan  p-ir-am-ik-ua.\\
thing  many  sea  canoe-\textsc{cf}  Bpx-ascend-2/3s.\textsc{ds}  land-canoe-\textsc{cf} here  Bpx-come-\textsc{ss}.\textsc{sim}-be-\textsc{pa}.3s\\ 
\glt ‘Many things were brought up by the ships and brought here by trucks.’ \\
\z


\ea
\gll  Yaapan  feenap  Madang  kame  ikiw-e-mik. \\
Japan  like.this  Madang  side  go-\textsc{pa}-1/3p \\
\glt ‘The Japanese went towards Madang like this.’ \\
\z


\ea
\gll  Ikoka  Yaapan=ke  i  emeria  yia  aaw-urum-i-kuan na-eya,  i  amirika  owowa  ewur  me ekap-em-ik-e-mik,  Yaapan  fan  naap  ekap-em-ika-iwkin. \\
Later  Japan-\textsc{cf}  1p.\textsc{unm}  woman  1p.\textsc{acc}  take-\textsc{distr}/A-Np-\textsc{fu}.3p say-2/3s.\textsc{ds}  1p.\textsc{unm}  daytime  village  quickly  not come-\textsc{ss}.\textsc{sim}-be-\textsc{pa}-1/3p  Japan  here  thus  come-\textsc{ss}.\textsc{sim}-be-2/3p.\textsc{ds} \\
\glt ‘He (had) said that the Japanese will take all of us women, and (so) we did not come quickly to the village in the daytime (i.e. we kept staying away from the village), as the Japanese kept coming here like that.’ \\
\z


\ea
\gll  Wi  Yaapan  emeria  weetak,  mua  manek-iw. \\
3p.\textsc{unm}  Japan  woman  no  man  big-\textsc{lim} \\
\glt ‘The Japanese had no women/wives, (they were) only men.’ \\
\z


\ea
\gll  Me  fan  Madang  kame=pa  ekap-e-mik, Wewak  kame=pa  fan  naap  ir-am-ik-e-mik. \\
not  here  Madang  side-\textsc{loc}  come-\textsc{pa}-1/3p Wewak  side-\textsc{loc}  here  thus  come-\textsc{ss}.\textsc{sim}-be-\textsc{pa}-1/3p\\ 
\glt ‘They did not come here from the Madang side/direction, they came here like that from the Wewak side.’ \\
\z


\ea
\gll  Wilkar,  miiw-aasa,  Madang  naap  irak-owa  ikiw-em-ik-e-mik, wi  Amerika  wiamiya  irak-owa  na-ep. \\
cart  land-canoe  Madang  thus  fight-\textsc{nmz}  go-\textsc{ss}.\textsc{sim}-be-\textsc{pa}-1/3p 3p.\textsc{unm}  America  3p.\textsc{com}  fight-\textsc{nmz}  say{}-\textsc{ss.seq}\\ 
\glt ‘Carts, trucks - they kept going like that to fight in Madang, to fight with the Americans.’ \\
\z


\ea
\gll  I  owowa=pa  fan  yiar  ik-e-mik. \\
1p.\textsc{unm}  village-\textsc{loc}  here  1p.\textsc{dat}  be-\textsc{pa}-1/3p \\
\glt ‘They were here in our village(s).’ \\
\z


\ea
\gll  Uura  feenap  nain,  wi  wilkar  nain  muf-emi “Wensa,  wensa,  wensa”,  naap  kirir-em-ik-e-mik.\\
night  like.this  that1  3p.\textsc{unm}  cart  that1  pull-\textsc{ss}.\textsc{sim} wensa  wensa  wensa  thus  shout-\textsc{ss}.\textsc{sim}-be-\textsc{pa}-1/3p \\ 
\glt ‘On nights like this they kept pulling the carts and shouting, “Wensa, wensa, wensa.” ’ \\
\z


\ea
\gll  Wi  owow  mua=ke  wilkar  wia muf-em-ik-om-a-mik,  mua  kui-kuisow wia  maak-iwkin.\\
3p.\textsc{unm}  village  man-\textsc{cf}  cart  3p.\textsc{acc} pull-\textsc{ss}.\textsc{sim}-be-\textsc{ben}-\textsc{bnfy}2.\textsc{pa}-1/3p  man  \textsc{rdp}-one 3p.\textsc{acc}  tell-2/3p.\textsc{ds}\\ 
\glt ‘The village men pulled carts for them, when they (the Japanese) had talked to a few men.’ \\
\z


\ea
\gll  “I  wilkar  yia  muf-om-aka,”  na-iwkin. \\
1p.\textsc{unm}  cart  1p.\textsc{acc}  pull-\textsc{ben}-\textsc{bnfy}2.\textsc{imp}.2p  say-2/3p.\textsc{ds} \\
\glt ‘When they had said, “Pull our carts for us.” ’ \\
\z


\ea
\gll  Wi  Malala=ke  muf-ep  ekap-emi  i  Moro mua  wia  wu-om-am-ik-om-a-mik.\\
3p.\textsc{unm}  Malala-\textsc{cf}  pull-\textsc{ss.seq}  come-\textsc{ss}.\textsc{sim}  1p.\textsc{unm}  Moro man  3p.\textsc{acc}  put-\textsc{ben}-\textsc{bnfy}2.\textsc{ss}.\textsc{sim}-be-\textsc{ben}-\textsc{bnfy}2.\textsc{pa}-1/3p \\ 
\glt ‘The Malala people kept pulling them and coming and putting them for the Moro men.’ \\
\z


\ea
\gll  Wilkar  nain  wiena  maa  koorma  unowa  ipar-iwkin wia  muf-em-ik-om-a-mik, feenap  Madang  p-ikiw-em-ik-e-mik, Australia  wiamiya  irak-owa  nain. \\
cart  that1  3p.\textsc{gen}  thing  cargo  many  fill-2/3p.\textsc{ds} 3p.\textsc{acc}  pull-\textsc{ss}.\textsc{sim}-be-\textsc{ben}-\textsc{bnfy}2.\textsc{pa}-1/3p ike.this  Madang  Bpx-go-\textsc{ss}.\textsc{sim}-be-\textsc{pa}-1/3p Australia  3p.\textsc{com}  fight-\textsc{nmz}  that1\\ 
\glt ‘They (J) filled the carts with lots of their cargo and they (=villagers) kept pulling them for them  – this way they (J) kept taking it to Madang, for the fighting with Australians.’ \\
\z


\ea
\gll  Nepa  opaimika  me  baliwep  amis-ar-ep wiena  opaimik=iw  yia  maak-em-ik-e-mik. \\
bird  talk  not  well  knowledge-\textsc{inch}-\textsc{ss.seq} 3p.\textsc{gen}  talk-\textsc{inst}  1p.\textsc{acc}  tell-\textsc{ss}.\textsc{sim}-be-\textsc{pa}-1/3p\\ 
\glt ‘They didn’t know Tok Pisin well and kept telling us in their own language.’ \\
\z


\ea
\gll  Aria  ‘taro’  yia  na-iwkin  miim-ap ma-em-ik-e-mik,  “Wi  moma  yia  maak-i-mik,  moma=ko  wi-i-yen.” \\
alright  taro  1p.\textsc{acc}  say-2/3p.\textsc{ds}  hear-\textsc{ss.seq} say-\textsc{ss}.\textsc{sim}-be-\textsc{pa}-1/3p  3p.\textsc{unm}  taro  1p.\textsc{acc} tell-Np-\textsc{pr}.1/3p  taro-\textsc{if}  give.them-Np-\textsc{fu}.1p \\
\glt ‘Alright when they said to us ‘Taro’, we heard and said, “They tell us (to get them) taro, (so) we’ll give them taro.” ’ \\
\z


\ea
\gll  “Banana  mau  oo,  yasi  kongkang,” iwera  yia  ir-om-aka,  “yasi  kongkang.” \\
banana  mau  oo  yasi  kongkang coconut  1p.\textsc{acc}  climb-\textsc{ben}-\textsc{bnfy}2.\textsc{imp}.2p  yasi  kongkang \\ 
\glt ‘ “Banana mau oo (ripe bananas oo, \textsc{t.p.}) , yasi kongkang” - climb coconut palms for us - “yasi kongkang.” ’ \\
\z


\ea
\gll  Iwera  “yasi”  yia  na-em-ik-e-mik. \\
coconut  yasi  1p.\textsc{acc}  say-\textsc{ss}.\textsc{sim}-be-\textsc{pa}-1/3p \\
\glt ‘They kept calling coconut “yasi” to us.’ \\
\z


\ea
\gll  Naeya  iwera  wia  uruk-am-ik-om-a-mik. \\
so  coconut  3p.\textsc{acc}  drop-\textsc{ss}.\textsc{sim}-be-\textsc{ben}-\textsc{bnfy}2.\textsc{pa}-1/3p \\
\glt ‘So they (=men) kept dropping coconuts for them.’ \\
\z


\ea
\gll  Iwera  uruk-am-ika-i-wkin  wi  ikiw-emi  aaw-em-ik-e-mik,  iwera=ke  wia  aruf-eya   ma-em-ik-e-mik,  “Oo  kanaka  oo,  yasi  paitim  mi  oo,” na-em-ik-e-mik. \\
coconut  drop-\textsc{ss}.\textsc{sim}-be-2/3p.\textsc{ds}  3p.\textsc{unm}  go-\textsc{ss}.\textsc{sim} take-\textsc{ss}.\textsc{sim}-be-\textsc{pa}-1/3p  coconut-\textsc{cf}  3p.\textsc{acc}  hit-2/3s.\textsc{ds} say-\textsc{ss}.\textsc{sim}-be-\textsc{pa}-1/3p  “Oo  kanaka  oo,  yasi  paitim  mi  oo” say-\textsc{ss}.\textsc{sim}-be-\textsc{pa}-1/3p\\
\glt ‘When they kept dropping the coconuts, they went and got them, and when the coconuts hit them they said (in Tok Pisin), “Oh savages, yasi hit me”, they kept saying (like that).’ \\
\z


\ea
\gll  I  me  wia  amukar-e-mik,  wis  pun  naap, i  me  yia  damol-a-mik. \\
1p.\textsc{unm}  not  3p.\textsc{acc}  scold-\textsc{pa}-1/3p  3p.\textsc{fc}  also  thus 1p.\textsc{unm}  not  1p.\textsc{acc}  bad-\textsc{pa}-1/3p \\ 
\glt ‘We didn’t scold them / quarrel with them, and they too likewise, they did not do damage to us.’ \\
\z


\ea
\gll  Ne  eka  opora  biiris  marew  nain  wiena  on-am-ik-e-mik. \\
\textsc{add}  river  mouth  bridge  no(ne)  \textsc{rm}  3p.\textsc{gen} make-\textsc{ss}.\textsc{sim}-be-\textsc{pa}-1/3p \\ 
\glt ‘And they themselves kept making bridges to rivers that didn’t have them.’ \\
\z


\ea
\gll  Nemuru  biiris  on-a-mik. \\
Nemuru  bridge  make-\textsc{pa}-1/3p \\
\glt ‘They made the Nemuru bridge.’ \\
\z


\ea
\gll  Biiris  me  eliwa,  damo-damola=ko. \\
bridge  not  good,  \textsc{rdp}-bad-\textsc{if} \\
\glt ‘The bridge was not good, it was very bad.’ \\
\z


\ea
\gll  Dabuel  poka-poka,  nomokowa  galua-galua,  nain=iw  biiris  on-am-ik-e-mik. \\
papaya  \textsc{rdp}-trunk  tree  \textsc{rdp}-soft that1-\textsc{inst}  bridge  make-\textsc{ss}.\textsc{sim}-be-\textsc{pa}-1/3p\\ 
\glt ‘Papaya trunks, soft timber, that’s what they kept making the bridges with.’ \\
\z


\ea
\gll  Ikoka  kuisow  miiw-aasa=ke  karu-eya  ku-ku-ep   or-om-ik-ua. \\
later  one  land-canoe-\textsc{cf}  run-2/3s.\textsc{ds}  \textsc{rdp}-break-\textsc{ss.seq} descend-\textsc{ss}.\textsc{sim}-be-\textsc{pa}.3s \\ 
\glt ‘Straight away when trucks ran (over them) they kept breaking and falling down.’ \\
\z


\ea
\gll  Ne  uurika  naap  nain  nainiw,  mauw-am-ik-e-mik. \\
\textsc{add}  tomorrow  thus  that1  again  work-\textsc{ss}.\textsc{sim}-be-\textsc{pa}.1/3p \\
\glt ‘And the following day it was like that again, they kept on working.’ \\
\z


\ea
\gll  Waaya  yia  na-iwkin  waaya  wienak-em-ik-e-mik. \\
pig  1p.\textsc{acc}  say-2/3p.\textsc{ds}  pig  feed.them{}-\textsc{ss}.\textsc{sim}-be-\textsc{pa}-1/3p \\
\glt ‘When they talked to us about pig/pork, we kept giving them pigs to eat.’ \\
\z


\ea
\gll  Aria  naap  ik-ok  wi  Australia  kerer-e-mik. \\
alright  thus  be-\textsc{ss}  3p.\textsc{unm}  Australia  arrive-\textsc{pa}-1/3p \\
\glt ‘Alright it was like that and then the Australians arrived.’ \\
\z


\ea
\gll  Ne  wi  Yaapan  ikiw-ep  Ulingan=pa  owowa war-e-mik. \\
\textsc{add}  3p.\textsc{unm}  Japan  go-\textsc{ss.seq}  Ulingan-\textsc{loc}  village found-\textsc{pa}-1/3p \\ 
\glt ‘And the Japanese (had) set up a village at Ulingan.’ \\
\z


\ea
\gll  Ne  nan  wi  owow  mua  wia  maak-e-mik, “Ni  kanaka  yia  uf-om-aka.” \\
\textsc{add}  there  3p.\textsc{unm}  village  man  3p.\textsc{acc}  tell-\textsc{pa}-1/3p  2p.\textsc{unm}  kanaka  1p.\textsc{acc}  dance-\textsc{ben}-\textsc{bnfy}2.\textsc{imp}.2p \\ 
\glt ‘And there they told the village men, “You kanakas (savages), dance for us.” ’ \\
\z


\ea
\gll  Na-iwkin  wia  uf-om-a-mik. \\
say-2/3p.\textsc{ds}  3p.\textsc{acc}  dance-\textsc{ben}-\textsc{bnfy}2.\textsc{pa}-1/3p \\
\glt ‘They said (so) and we/they danced for them.’ \\
\z


\ea
\gll  I  Moro  wenup  uf-e-mik, i  bidaru  uf-e-mik. \\
1p.\textsc{unm}  Moro  separately  dance-\textsc{pa}-1/3p 1p.\textsc{unm}  bidaru  dance-\textsc{pa}-1/3p \\ 
\glt ‘We Moro people danced on our own, we danced “bidaru”.’ \\
\z


\ea
\gll  Uf-em-ika-iwkin,  Amerika  irak-ow  iinan  aasa  ekap-ep  Ulingan  nan  bom  fu-fuurik-ikiw-e-mik. \\
dance-\textsc{ss}.\textsc{sim}-be-2/3p.\textsc{ds}  America  fight-\textsc{nmz}  sky  canoe come-\textsc{ss.seq}  Ulingan  there  bomb  \textsc{rdp}-drop-go-\textsc{pa}-1/3p \\ 
\glt ‘As they were dancing, American fighter planes came and went dropping bombs at Ulingan. ’ \\
\z


\ea
\gll  I  iinan  aasa  me  kuuf-a-mik. \\
1p.\textsc{unm}  sky  canoe  not  see-\textsc{pa}-1/3p \\
\glt ‘We didn’t see the planes.’ \\
\z


\ea
\gll  Iinan=iw  iinan=iw  wu-ami  feenap  Wewak  kame naap  ikiw-o-k.\\
on.top-\textsc{inst}  on.top-\textsc{inst}  put-\textsc{ss}.\textsc{sim}  like.this  Wewak  side thus  go-\textsc{pa}-3s \\
\glt ‘They stayed really high up and went like this towards Wewak.’ \\
\z


\ea
\gll  Ne  Sarak  opaimika  wiar  paayar-e-k,  siiwa,  epa  maak-e-mik  nain  paayar-ep  ma-e-k, “Amerika  aakisa  irak-owa  kerer-e-mik,”  na-ep  wi  Yaapan  wia  uf-om-owa  ikiw-e-mik  nain  wia  maak-e-k, “Ni  uf-owa  ikiw-eka,  yo  miatin-i-yem. \\
\textsc{add}  Sarak  talk  3.\textsc{dat}  understand-\textsc{pa}-3s  moon  place  tell-\textsc{pa}-1/3p  \textsc{rm}  understand-\textsc{ss.seq}  say-\textsc{pa}-3s   America  now  fight-\textsc{nmz}  arrive-\textsc{pa}-1/3p  say-\textsc{ss.seq}  3p.\textsc{unm}  Japan  3p.\textsc{acc}  dance-\textsc{ben}-\textsc{nmz}  go-\textsc{pa}-1/3p  \textsc{rm}  3p.\textsc{acc}  tell-\textsc{pa}-3s  2p.\textsc{unm}  dance-\textsc{nmz}  go-\textsc{imp}.2p  1s.\textsc{unm}  dislike-Np-\textsc{pr}.1s \\
\glt ‘And/but Sarak understood their talk, and he knew the month and place/time that they had told, and he said, “The Americans have now arrived to fight” - he said that and told those who went to dance for the Japanese, “You go to dance, I don’t like (to go). ’ \\
\z


\ea
\gll  Irak-owa  maneka  ikoka  kerer-i-non. \\
fight-\textsc{nmz}  big  later  appear-Np-\textsc{fu}.3s \\
\glt ‘The big fighting will start later.’ \\
\z


\ea
\gll  Ni  uf-ep=na  ni  maadara  me iirar-eka,  mokok  urupa  kaik-i-man  nain  feekiya Nomon  owowa,  Medebur  karu-eka,  baurar-eka. \\
2p.\textsc{unm}  dance-\textsc{ss.seq}-\textsc{tp}  2p.\textsc{unm}  forehead.ornament  not remove-\textsc{imp}.2p  eye  cup  tie-Np-\textsc{pr}.2p  \textsc{rm}  with stone/reef  village  Medebur  run-\textsc{imp}.2p  flee-\textsc{imp}.2p \\ 
\glt ‘If/when you have danced, don’t remove your forehead ornaments, and, with your eye cups that you have tied on, run to the reef village, Medebur, flee (there).’ \\
\z


\ea
\gll  Ne  wi  ikiw-iwkin  Amerika  iinan  aasa  nainiw  kir-e-k. \\
\textsc{add}  3p.\textsc{unm}  go-2/3p.\textsc{ds}  America  sky  canoe  again  turn-\textsc{pa}-3s \\
\glt ‘And when they had gone the American planes turned again (and came back).’ \\
\z


\ea
\gll  Iinan=iw  iki(w-e)p  kirip-ap  enen=iw, enen=iw  wu-a-k  ne  i  fiker  gone=pa  ik-emkun  ekap-o-k. \\
on.top-\textsc{inst}  go-\textsc{ss.seq}  turn-\textsc{ss.seq}  low.down-\textsc{inst}   low.down-\textsc{inst}  put-\textsc{pa}-3s  \textsc{add}  1p.\textsc{unm}  kunai.grass  middle-\textsc{loc} 
be-1s/p.\textsc{ds}  come-\textsc{pa}-3s \\
\glt ‘They went high up and turned and flew down, and came when we were in the middle of the kunai grass.’ \\
\z


\ea
\gll  Malala  suule  ik-ua  nain,  Siburten  ema,  oram  tene-ten-ep  or-op  epa  iiwawun  ifemak-ep  ekap-e-mik. \\
Malala  school  be-\textsc{pa}.3s  that1  Siburten  hill  just  \textsc{rdp}-fall-\textsc{ss.seq} descend-\textsc{ss.seq}  place  altogether  press-\textsc{ss.seq}  come-\textsc{pa}-1/3 \\
\glt ‘Where the Malala school is, and Siburten hill - (the bombs) just fell down and came covering the whole place.’ \\
\z


\ea
\gll  Ne  bom=iya,  kateres=iya,  bom=iya,  kateres=iya. \\
\textsc{add}  bomb-\textsc{com},  cartridge-\textsc{com},  bomb-\textsc{com},  cartridge-\textsc{com} \\
\glt ‘And boms and cartridges, bombs and cartridges.’ \\
\z


\ea
\gll  Koora  pun  ariwa=ke  kuum-eya  aw-omak-e-k,  i  koora  kuisow  yiar  aw-o-k. \\
house  also  arrow-\textsc{cf}  burn-2/3s.\textsc{ds}  burn-\textsc{distr}/\textsc{pl}-\textsc{pa}-3s 1p.\textsc{unm}  house  one  1p.\textsc{dat}  burn-\textsc{pa}-3s \\
\glt ‘Many houses too were burned by their firing, one house burned from us.’ \\
\z


\ea
\gll  Ne  yena  aamun  ariwa=ke  aaw-o-k, weepa  Aduna  ikos  iimar-ep  ika-iwkin,  owow  erepura=pa  fan. \\
\textsc{add}  1s.\textsc{gen}  1s/p.younger.sibling  arrow-\textsc{cf}  get-\textsc{pa}-3s 3s/p.elder.sibling  Aduna  with  stand.up-\textsc{ss.seq}  be-2/3p.\textsc{ds} village  outskirts-\textsc{loc}  here \\
\glt ‘And my younger brother was killed by a bullet, as he was standing with his elder brother Aduna, here on the outskirts of the village.’ \\
\z


\ea
\gll  Wi  Amerika=ke  war-e-mik. \\
3p.\textsc{unm}  America-\textsc{cf}  shoot-\textsc{pa}-1/3p \\
\glt ‘The Americans shot him.’ \\
\z


\ea
\gll  Yena  aamun  unuma  Saawoka. \\
1s.\textsc{gen}  1s/p.younger.sibling  name  Saawoka \\
\glt ‘The name of my younger brother was Saawoka.’ \\
\z

\largerpage
\ea
\gll  I  fiker  gone=pa  ik-emkun  kiikir  iinan  aasa=ke maifa  fu-fuurk-ikiw-o-k,  wi  Amerika=ke. \\
1p.\textsc{unm}  kunai.grass  middle-\textsc{loc}  be-1s/p.\textsc{ds}  first  sky  canoe-\textsc{cf} paper  \textsc{rdp}-drop-go-\textsc{pa}-3s  3p.\textsc{unm}  America-\textsc{cf} \\
\glt ‘When we were in the middle of the kunai grass, first the planes went dropping papers, the Americans (did that).’ \\
\z


\ea
\gll  Ulingan  fa=na  iinan  aasa  nepa  saarik,  unow  akena. \\
Ulingan  \textsc{intj}-\textsc{tp}  sky  canoe  bird  like  many  very \\
\glt ‘Ulingan, phew! There were planes like birds, very many.’ \\
\z


\ea
\gll  Wi  Yaapan  nan  ik-e-mik  nain  wia  uruf-ap. \\
3p.\textsc{unm}  Japan  there  be-\textsc{pa}-1/3p  \textsc{rm}  3p.\textsc{acc}  see-\textsc{ss.seq} \\
\glt ‘They had seen the Japanese that were there (and came).’ \\
\z


\ea
\gll  I  karan-e-mik,  yena  mua  Sarak=na  fiker  gone  nomokowa  onoma=pa  in-ep  ik-ua. \\
1p.\textsc{unm}  shake-\textsc{pa}-1/3p  1s.\textsc{gen}  man  Sarak-\textsc{tp}  kunai.grass middle  tree  base-\textsc{loc}  sleep-\textsc{ss.seq}  be-\textsc{pa}.3s \\
\glt ‘We were afraid, (but) my husband Sarak was asleep at the base of a tree in the middle of the kunai grass area.’ \\
\z


\ea
\gll  Ne  yo  iki(w-e)p  mesa  asia  aaw-em-ik-e-m, mesa  asia  fiker  gone=pa  ika-i-ya  nain  aaw-em-ik-e-m. \\
\textsc{add}  1s.\textsc{unm}  go-\textsc{ss.seq}  winged.bean  wild  get-\textsc{ss}.\textsc{sim}-be-\textsc{pa}-1s winged.bean  wild  kunai.grass  middle-\textsc{loc}  be-Np-\textsc{pr}.3s  \textsc{rm} get-\textsc{ss}.\textsc{sim}-be-\textsc{pa}-1s\\ 
\glt ‘And I went and was picking wild winged beans, I was picking those wild winged beans that grow in the middle of the kunai grass.’ \\
\z


\ea
\gll  Ne  iinan  aasa  uruf-ap  fiker  tepak  iw-ap naap  ik-e-m,  ne  iinan  aasa  yia  nomak-ep ikiw-eya  Sarak  far-e-m,  “Sarak,  yo  damol-al-e-m oo,  fiker  fufa  iw-a-m  oo!” \\
\textsc{add}  sky  canoe  see-\textsc{ss.seq}  kunai.grass  inside  enter-\textsc{ss.seq} thus  be-\textsc{pa}-1s  \textsc{add}  sky  canoe  1p.\textsc{acc}  pass-\textsc{ss.seq} go-2/3s.\textsc{ds}  Sarak  call-\textsc{pa}-1s  Sarak  1s.\textsc{unm}  bad-\textsc{inch}-\textsc{pa}-1s \textsc{intj}  kunai.grass  old.grass  enter-\textsc{pa}-1s  \textsc{intj} \\ 
\glt ‘And when I saw the planes I went inside the kunai grass and stayed like that, and when the planes passed over us and went I called to Sarak, “Sarak, oh I'm in a bad way, I am hiding among the kunai grass!” ’ \\
\z


\ea
\gll  “Mauwa  ar-e-n,  amia=iya  nenar-e-mik=i?”  Sarak=ke  Ø. \\
what  become-\textsc{pa}-2s  gun-\textsc{com}  shoot.you-\textsc{pa}-1/3p-\textsc{qm} Sarak-\textsc{cf} \\ 
\glt ‘ “What happened to you, did they shoot you with a gun?” Sarak (asked).’ \\
\z


\ea
\gll  “Wia,  nos=na,  yo  fiker  fufa  iw-ap nefa  far-i-yem.” \\
no  2s.\textsc{fc}-\textsc{tp}  1s.\textsc{unm}  kunai.grass  old.grass  enter-\textsc{ss.seq} 2s.\textsc{acc}  call-Np-\textsc{pr}.1s \\
\glt ‘ “No, you see, I went inside the kunai grass and am calling you (from there).”~’ \\
\z


\ea
\gll  “Momora,  no  naap  me  ma-e. \\
fool  2s.\textsc{unm}  thus  not  say-\textsc{imp}.2s \\
\glt ‘ “Fool, don’t say like that.’ \\
\z


\ea
\gll  Nefa  war-iwkin  naap  ma-e. \\
2s.\textsc{acc}  shoot-2/3p.\textsc{ds}  thus  say-\textsc{imp}.2s \\
\glt ‘When they shoot you, (then) say like that.’ \\
\z


\ea
\gll  Yo  kema  efa  bagiw-ir-ow-a-n,  yaa.” \\
1s.\textsc{unm}  liver  1s.\textsc{acc}  anger-come-\textsc{appl}-\textsc{pa}-2s  \textsc{intj} \\
\glt ‘You really made me angry.” ’ \\
\z


\ea
\gll  Yo  me  efa  war-e-mik,  yo  iinan  aasa  fan or-om-ik-omak-eya  mesa  fufa=pa  fan  erewar-ep  iima  ifemak-ep  ik-e-m. \\
1s.\textsc{unm}  not  1s.\textsc{acc}  shoot-\textsc{pa}-1/3p  1s.\textsc{unm}  sky  canoe  here descend-\textsc{ss}.\textsc{sim}-be-\textsc{distr}/\textsc{pl}-2/3s.\textsc{ds}  winged.bean  old.grass-\textsc{loc} here  shelter-\textsc{ss.seq}  chest  press-\textsc{ss.seq}  be-\textsc{pa}-1s \\
\glt ‘They didn’t shoot me; when the many planes were coming down here, I sheltered among the winged bean grass and was lying face down.’ \\
\z


\ea
\gll  Irak-owa  nain  kekan-owa  akena,  Amerika  kerer-e-k  nain,  ne  irak-owa  naap  ik-ua. \\
fight-\textsc{nmz}  that  be.strong-\textsc{nmz}  very  America  arrive-\textsc{pa}-3s  that1 \textsc{add}  fight-\textsc{nmz}  thus  be-\textsc{pa}.3s \\
\glt ‘The fighting was very strong when the Americans came, and the fighting continued like that.’ \\
\z


\ea
\gll  Uura  feenap  nain,  i  me  in-em-ik-e-mik,  amirika maa  me  en-em-ik-e-mik. \\
night  like.this  that1  1p.\textsc{unm}  not  sleep-\textsc{ss}.\textsc{sim}-be-\textsc{pa}-1/3p  daytime   food  not  eat-\textsc{ss}.\textsc{sim}-be-\textsc{pa}-1/3p\\
\glt ‘On nights like this we did not sleep, in the daytime we did not eat.’ \\
\z


\ea
\gll  Maa  uura  uup-ep  en-em-ik-e-mik. \\
food  night  cook-\textsc{ss.seq}  eat-\textsc{ss}.\textsc{sim}-be-\textsc{pa}-1/3p \\
\glt ‘The food we used to cook at night.’ \\
\z


\ea
\gll  Ikoka  epia  wilin-owa  uruf-ap  bom  yia   fuurk-om-i-kuan  na-ep. \\
later  fire  shine-\textsc{nmz}  see-\textsc{ss.seq}  bomb  1p.\textsc{acc}  drop-\textsc{ben}-Np-\textsc{fu}.3p  say/think-\textsc{ss.seq} \\ 
\glt ‘We thought that later when they see the shine of the fire(s) they will drop bombs at us (and so we did like that).’ \\
\z


\ea
\gll  Iinan  aasa  gurun-owa  miim-ap  eka=iw   umuk-owa  ewur. \\
sky  canoe  rumble-\textsc{nmz}  hear-\textsc{ss.seq}  water-\textsc{inst} extinguish-\textsc{nmz}  quickly \\


\glt ‘When we heard the rumble of the planes, we extinguished (the fires) quicky with water.’ \\
\z


\ea
\gll  Epia  wilinowa  urufap  bom  yia  wafur-om-i-kuan        na-ep  naap  on-am-ik-e-mik. \\
fire  shine-\textsc{nmz}  see-\textsc{ss.seq}  bomb  1p.\textsc{acc}  throw-\textsc{ben}-Np-\textsc{fu}.3p  say/think-\textsc{ss.seq}  thus  do{}-\textsc{ss}.\textsc{sim}-be-\textsc{pa}-1/3p \\
\glt ‘We thought that when they see the shine of the fire they will throw bombs at us, and we did like that.’ \\
\z


\ea
\gll  I  Sarak  ikos  owowa  ekap-em-ik-e-mik. \\
1p.\textsc{unm}  Sarak  with  village  come-\textsc{ss}.\textsc{sim}-be-\textsc{pa}-1/3p \\
\glt ‘Sarak and I kept coming to the village.’ \\
\z


\ea
\gll  Yiena  koora  pun  me  aw-o-k. \\
1p.\textsc{gen}  house  also  not  burn-\textsc{pa}-3s \\
\glt ‘Our house too did not burn.’ \\
\z


\ea
\gll  Wi  yapen=pa  ik-omak-iwkin  Amerika  kerer-ep             “Eliwa,  eliwa”  nae-ekap-e-mik,  “irak-owa  weeser-e-k.” \\
3p.\textsc{unm}  inland-\textsc{loc}  be-\textsc{distr}/\textsc{pl}-2/3p.\textsc{ds}  America  arrive-\textsc{ss.seq}  good  good  say-come-\textsc{pa}-1/3p  fight-\textsc{nmz}  finish-\textsc{pa}-3s \\
\glt ‘When they (=other villagers) were inland the Americans arrived saying, “Good, good - the war is finished.” ’ \\
\z


\ea
\gll  Wi  Yaapan  pun  iiwawun  ikiw-urum-e-mik. \\
3p.\textsc{unm}  Japan  also  altogether  go-\textsc{distr}/A-\textsc{pa}-1/3p \\
\glt ‘Also, all the Japanese went.’ \\
\z


\ea
\gll  Wi  Yaapan  Madang  kame  ikiw-iwkin  Amerika  irak-owa mua  urup-e-mik. \\
3p.\textsc{unm}  Japan  Madang  side  go-2/3p.\textsc{ds}  America  fight-\textsc{nmz}   man  ascend-\textsc{pa}-1/3p \\


\glt ‘When the Japanese went/had gone towards Madang, the American soldiers came up (here).’ \\
\z


\ea
\gll  Urup-ep  koka-koka=pa  nan,  “Yapen  oo,  Luluai,  Tultul,   ni  kaaneke  ik-e-man  oo,  ni  ekap-omak-eka               oo!” \\
ascend-\textsc{ss.seq}  \textsc{rdp}-jungle-\textsc{loc}  there  inland  \textsc{intj}  Luluai  Tultul 2p.\textsc{unm}  where  be-\textsc{pa}-2p  \textsc{intj}  2p.\textsc{unm}  come-\textsc{distr}/\textsc{pl}-\textsc{imp}.2p \textsc{intj} \\




\glt ‘They came up and (shouted) all over the jungle: “You in the inland, local administrators, wherever you are – come!” ’ \\
\z


\ea
\gll  Wi  Amerika  fan  “Epa  eliwa”  nae-ekap-e-mik,  “Yaapan  weeser-e-mik.” \\
3p.\textsc{unm}  America  here  time  good  say-come-\textsc{pa}-1/3p    Japan  finish-\textsc{pa}-1/3p \\


\glt ‘The Americans came here saying, “It is peace, the Japanese are finished.” ’ \\
\z


\ea
\gll  Naap  yia  maake-miaw-e-mik. \\
thus  1p.\textsc{acc}  tell-go.around-\textsc{pa}-1/3p \\
\glt ‘They went around telling us like that.’ \\
\z


\ea
\gll  I  amirik=iw  yapen  yiena  Gawar  wiar  ikiw-e-mik. \\
1p.\textsc{unm}  daytime-\textsc{inst}  inland  1p.\textsc{gen}  Gawar  3.\textsc{dat}  go-\textsc{pa}-1/3p \\


\glt ‘In the daytime we went to our (relatives at) Gawar.’ \\
\z


\ea
\gll  Ne  Sarak  ikos  Gawar=pa  ik-emkun  yia  maak-e-mik     “Sarak  oo,  Amerika  ekap-ep  Ulingan  nan  ik-e-mik,           nefa  ikum-i-mik  na-i-mik  oo.” \\
\textsc{add}  Sarak  with  Gawar-\textsc{loc}  be-1s/p.\textsc{ds}  1p.\textsc{acc}  tell-\textsc{pa}-1/3p  Sarak  \textsc{intj}  America  come-\textsc{ss.seq}  Ulingan  there  be-\textsc{pa}-1/3p    2s.\textsc{acc}  wonder.about-Np-\textsc{pr}.1/3p  say-Np-\textsc{pr}-1/3p  \textsc{intj} \\




\glt ‘And when Sarak and I were at Gawar they told us, “Sarak oh, the Americans have come and are there in Ulingan and they say that they are wondering about you.” ’ \\
\z


\ea
\gll  “Aria,  Kalina,  Amerika  ekap-e-mik  na-i-mik. \\
alright  Kalina  America  come-\textsc{pa}-1/3p  say-Np-\textsc{pr}.1/3p \\
\glt ‘ “Alright, Kalina, they say that the Americans have come.’ \\
\z


\ea
\gll  I  or-u”. \\
1p.\textsc{unm}  descend-\textsc{imp}.1d \\
\glt ‘Let’s go down.’ \\
\z


\ea
\gll  Aria  i  yapen=pa  ik-ok  owowa  or-o-mik. \\
alright  1p.\textsc{unm}  inland-\textsc{loc}  be-\textsc{ss}  village  descend-\textsc{pa}-1/3p \\
\glt ‘Alright, after having stayed inland we went down to the village.’ \\
\z


\ea
\gll  Or-omi  yo  koka  koora=pa  nan  efa               wu-ami  ma-e-k,  “No  feeke  ik-e,  yo  Amerika                   wia  akup-ikiw-i-yem.” \\
descend-\textsc{ss}.\textsc{sim}  1s.\textsc{unm}  jungle  house-\textsc{loc}  there  1s.\textsc{acc} put-\textsc{ss}.\textsc{sim}  say-\textsc{pa}-3s  2s.\textsc{unm}  here.\textsc{cf}  be-\textsc{imp}.2s  1s.\textsc{unm} America  3p.\textsc{acc}  search-go-Np-\textsc{pr}.1s \\


\glt ‘When we were coming down he put me in (our) jungle house and said, “Stay here, I go and look for the Americans.” ’ \\
\z


\ea
\gll  Aria  yo  nan  efa  wu-ap-pu-ami                  o  Ulingan  ikiw-o-k. \\
alright  1s.\textsc{unm}  there  1s.\textsc{acc}  put-\textsc{ss.seq}-\textsc{cmpl}-\textsc{ss}.\textsc{sim}  3s.\textsc{unm}  Ulingan  go-\textsc{pa}-3s \\


\glt ‘Alright he put me there and went to Ulingan.’ \\
\z


\ea
\gll  Ikiw-o-k=na  wi  Amerika  maneka  unuma  Magerka,  o  kerer-ep  nan  ik-ua. \\
go-\textsc{pa}-3s-\textsc{tp}  3p.\textsc{unm}  America  big  name  MacArthur       3s.\textsc{unm}  arrive-\textsc{ss.seq}  there  be-\textsc{pa}.3s \\


\glt ‘He went, and the chief of the Americans, whose name was MacArthur. he had arrived and was there.’ \\
\z


\ea
\gll  O  ik-ip  uruf-eya  maak-ek,  ``O  Sarak,  no    kaaneke  ik-ok  kerer-e-n  a? \\
3s.\textsc{unm}  go-\textsc{ss.seq}  see-2/3s.\textsc{ds}  tell-\textsc{pa}-3s  \textsc{intj}  Sarak  2s.\textsc{unm}     where  be-\textsc{ss}  arrive-\textsc{pa}-2s  \textsc{intj} \\


\glt ‘He (Sarak) went and saw him, and he (MacA.) said, “Sarak, where have you been?” ’ \\
\z


\ea
\gll  “Yo  koka=pa  ik-e-m.” \\
1s.\textsc{unm}  jungle-\textsc{loc}  be.\textsc{pa}-1s \\
\glt ‘ “I have been in the jungle.” ’ \\
\z


\ea
\gll  Na-eya  Magerka=ke  Ø,  ``No  kamenap  ik-o-n  noma? \\
say-2/3s.\textsc{ds}  MacArthur-\textsc{cf}  Ø  2s.\textsc{unm}  how  be-\textsc{pa}-2s  \textsc{intj} \\
\glt ‘He said that and MacArthur asked, “Well, how are you?’ \\
\z


\ea
\gll  Yo  irak-owa  nomak-e-m. \\
1s.\textsc{unm}  fight-\textsc{nmz}  win-\textsc{pa}-1s \\
\glt ‘I won the war.’ \\
\z


\ea
\gll  Nomak-ep  fan  kerer-e-m.” \\
win-\textsc{ss.seq}  here  arrive-\textsc{pa}-1s \\
\glt ‘I won and I arrived here.” ’ \\
\z


\ea
\gll  “Yo  bom  yo  kateres  koor  miira=pa  efar           or-om-ik-ua. \\
1s.\textsc{unm}  bomb  1s.\textsc{unm}  cartridge  house  front-\textsc{loc}  1s.\textsc{dat}   descend-\textsc{ss}.\textsc{sim}-be-\textsc{pa}.3s \\


\glt ‘ “Bombs and cartridges kept dropping in front of my house.’ \\
\z


\ea
\gll  Nain  yo  me  baurar-em-ik-e-m. \\
that1  1s.\textsc{unm}  not  flee-\textsc{ss}.\textsc{sim}-be-\textsc{pa}-1s \\
\glt ‘But I did not keep running away.’ \\
\z


\ea
\gll  I  iinan  aasa  ekap-em-ika-eya  uruwa  ain-ep             or-op  yena  koor  miira=pa  iimar-ep                     ik-e-m,  yena  emeria  sosora  ain-ep  or-op                        koor  miira=pa  iimar-em-ik-ua. \\
1p.\textsc{unm}  sky  canoe  come-\textsc{ss}.\textsc{sim}-be-2/3.\textsc{ds}  loincloth  tie-\textsc{ss.seq} descend-\textsc{ss.seq}  1s.\textsc{gen}  house  front-\textsc{loc}  stand.up-\textsc{ss.seq} be-\textsc{pa}-1s  1s.\textsc{gen}  woman  grass.skirt  tie-\textsc{ss.seq}  descend-\textsc{ss.seq}    house  front-\textsc{loc}  stand-\textsc{ss}.\textsc{sim}-be-\textsc{pa}.3s \\






\glt ‘When the planes kept coming I tied the loincloth and went down and stood in front of my house, and my wife used to tie the grass skirt and go down and stand in front of the house.’ \\
\z


\ea
\gll  Ne  i  me  yiar-e-mik,  Amerika  iinan  aasa      ekap-em-ik-ua  nain. \\
\textsc{add}  1p.\textsc{unm}  not  shoot-\textsc{pa}-1/3p  America  sky  canoe    come-\textsc{ss}.\textsc{sim}-be-\textsc{pa}.3s  \textsc{rm} \\


\glt ‘And/but they didn’t shoot at us, the American planes that kept coming.’ \\
\z


\ea
\gll  Amina,  woowa,  eka  napia  koor  miira=pa       iimar-aw-ikiw-e-mik,  yena  emeria  ikos,  ne        wi  unowa  baurar-e-mik. \\
pot  spear  water  bamboo.container  house  front-\textsc{loc}  stand.up-\textsc{appl}-go-\textsc{pa}-1/3p  1s.\textsc{gen}  woman  with  \textsc{add}   3p.\textsc{unm}  many  flee-\textsc{pa}-1/3p \\




\glt ‘We lined up pots, spears and bamboo water containers in front of the house, (I) with my wife, but many people fled.’ \\
\z


\ea
\gll  I  iisow  naap  ik-emkun  Amerika  kerer-e-mik.” \\
1p.\textsc{unm}  1p.\textsc{isol}  thus  be-1s/p.\textsc{ds}  America  arrive-\textsc{pa}-1/3p \\
\glt ‘When we were by ourselves like that, the Americans arrived.” ’ \\
\z


\ea
\gll  “Ni  yapen  koka-koka=pa  wiar  in-em-ik-e-man          nain  kerer-omak-eka.” \\
2p.\textsc{unm}  inland  \textsc{rdp}-jungle-\textsc{loc}  3.\textsc{dat}  sleep-\textsc{ss}.\textsc{sim}-be-\textsc{pa}-2p  \textsc{rm}  arrive-\textsc{distr}/\textsc{pl}-\textsc{imp}.2p \\


\glt ‘Those (many) of you who have stayed in the inland villages, arrive (back to your villages).” ’ \\
\z


\ea
\gll  Amerika  fan  ekap-ep  Uligan=pa  manua  maneka   urup-ep  irak-ow(a)  mua  wia  wu-eya  nan                   ik-e-mik. \\
America  here  come-\textsc{ss.seq}  Ulingan-\textsc{loc}  maneuver  big  ascend-\textsc{ss.seq}  fight-\textsc{nmz}  man  3p.\textsc{acc}  put-2/3s.\textsc{ds}  there    be-\textsc{pa}-1/3p \\




\glt ‘The Americans came here and went up to Ulingan for a big maneuver and put soldiers to stay there.’ \\
\z


\ea
\gll  Palauwa  tin,  maa  eneka  kes  mane-maneka  fa,     oram  iw-e-mik,  Sarak  iw-e-mik. \\
flour  tin  thing  tooth  case  \textsc{rdp}-big  \textsc{intj}              just  give.him-\textsc{pa}-1/3p  Sarak  give.him-\textsc{pa}-1/3p \\


\glt ‘Flour tins, big cases of meat (tins), wow, they gave to him for nothing, they gave them to Sarak.’ \\
\z


\ea
\gll  “No  fain  wiim-ep  amap-ikiw-e.” \\
2s.\textsc{unm}  this  take.along-\textsc{ss.seq}  Bpx-go-\textsc{imp}.2s \\
\glt ‘ “Take these along and go.” ’ \\
\z


\ea
\gll  O  wi  Laman,  Malager  naap  wia  maak-e-k,        “Arika,  takira,  yo  yook-eka.” \\
3s.\textsc{unm}  3p.\textsc{unm}  Laman  Malager  thus  3p.\textsc{acc}  tell-\textsc{pa}-3s  alright  boy  1s.\textsc{unm}  follow.me-\textsc{imp}.2p \\


\glt ‘He (Sarak) had told Laman and Malager, “Alright boys, follow me.’ \\
\z


\ea
\gll  Amerika  kerer-e-mik  naeya  ikiwep=ko  wia  uruf-ikua. \\
America  arrive-\textsc{pa}-1/3p  so  go-\textsc{ss.seq}-\textsc{if}  3p.\textsc{acc}  see-\textsc{imp}.1p \\
\glt ‘The Americans have arrived, so let’s go and see them.’ \\
\z


\ea
\gll  O  Ulingan  ikiw-eya  bom  iw-e-mik. \\
3s.\textsc{unm}  Ulingan  go-2/3s.\textsc{ds}  bomb  give.him-\textsc{pa}-1/3p \\
\glt ‘When he went to Ulingan they gave him a bomb/bombs.’ \\
\z


\ea
\gll  “No  bom  fain=iw  mera  kuum-e,”  naak-e-mik. \\
2s.\textsc{unm}  bomb  this-\textsc{inst}  fish  burn-\textsc{imp}.2s  tell-\textsc{pa}-1/3p \\
\glt ‘ “Blast fish with this bomb/these bombs,” they told him.’ \\
\z


\ea
\gll  Maa  en-owa  iw-e-mik,  palauwa  dram,    bata  tin  naap  nain,  ne  rais  weetak. \\
thing  eat-\textsc{nmz}  give.him-\textsc{pa}-1/3p  flour  drum   butter  tin  thus  that1  \textsc{add}  rice  no \\


\glt ‘They gave him food, flour drums, and things like butter tins, but no rice.’ \\
\z


\ea
\gll  Ne  irak-owa  weeser-eya  i  owowa  or-o-mik. \\
\textsc{add}  figth-\textsc{nmz}  finish-2/3s.\textsc{ds}  1p.\textsc{unm}  village  descend-\textsc{pa}-1/3p \\
\glt ‘And when the war ended we came down to the village.’ \\
\z


\ea
\gll  Irak-owa  fa,  opaimika  eeya  akena  yaa,  kamenap  aakun-i-yen  yaa? \\
fight-\textsc{nmz}  \textsc{intj}  talk  long.lasting  very  \textsc{intj}  how       talk-Np-\textsc{fu}.1p  \textsc{intj} \\


\glt ‘Oh the war - the talk lasts very long - how can we talk about it?’ \\
\z


\ea
\gll  Aakun-i-yen,  aakun-i-yen,  me  pepek,  me  welaw-i-yen. \\
talk-Np-\textsc{fu}.1p  talk-Np-\textsc{fu}.1p  not  enough  not  finish-Np-\textsc{fu}.1p \\
\glt ‘We will talk and talk, but it is not enough, we won’t finish it.’ \\
\z

I  me  amisarem-ikomkun  iinan  aasa  iinan-pa  fan  ekapemi  paranem-yiomakek. 
I  naap  koora-pa  ikemik,  koora-pa  ikemik. 

Aria  yena  mua  pun  irakowa  kererowa  epa weeserem-ikeya  iirariwkin  owowa  ekapok, o  amia  mua-pa  ikok. 
Maakemik,  “No  ikiwe,  irakowa  maneka  fanek a,”  ne  ekapok. 
Ekapemi  yo  efa  aawok. 
Yo  efa  aaweya  i  owawiya  ikomkun   Yaapan ena  Wewak  kame-pa  nan  iramik.

Ne  i  Emer  Era-pa  nan  ikemik,  ne  oos   onaiya  Nemuru  lul  nan  iram-ikemik. 
Iram-ikaiwkin  yena  mua  Sarak-ke  wia  urufap  maek,  “Yaapan  nanemik. 
Saa-iw  iram-ikaimik,  oos  ono-onaiya       iram-ikaimik.”

Ne  i  owowa  iramik. 
Owowa-pa  fan  yia  maakemik,  “Yaapan  fan  ikiwemik,       oos  onaiya  Madang  ikiwemik. 
Wi  sawur  nain  irami  fan  yiar  pokamik.” 
Wi  Yaapan  nain  naap  wia  maemik,  “Sawur-ke       fan  ikiwemik.” 

Aria  naap  ikok  i  bauraremik. 
Muakura-ke  maek,  ``Ikoka  Yaapan-ke  ekapemi  ni       emeria  unowa  fain  nia  aawurumikuan.” 
I  uura  maa  unowa  opap  baurarep  koka       ikiwemik. 
Iki(we)p  kokapa  nan  inem-ikemik,       ikoka  Yaapan  me  yia  aawuk  naep. 
“Ni  ikoka  Yaapan-ke  emeria  niar  aawemi   ni  umakuna  nia  puukikuan.” 
I  nain  kema  tootonep,  ikoka  Yaapan  me  yia       aawuk  naep  uura  kuisow  bauraremik, maa  unowa  owowa-pa  feeke  ikaeya. 
Aakisa  Malala  suule  ikua,  i  naap  ikiwep yiena  manina  onamik  nan  ikemik. 
I  miiwa  nan  arufam-ikemik,  ne  iki(we)p  nan  inemikemik. 
Emeria  teeria  unowiya  baurarep  koka   ikiwurumemik,  kuisow  owowapa-ko  me  ikua. 
Mua  muutiw  owowa-pa  ikemik. 
Mua  kuisow,  Muakura-ke  i  yia  pikiwok. 
“Karueka,  ikoka  Yaapan  irami  ni  nia  aawemi  yo  efa  ifakimikuan,”  naeya i  karuem-ikemik. 

Irakowa  maneka  ewur  me  imenarek,      wi  Yaapan  naap  kuisowiw  ekapem-ikemik, owowa  yiar  kuufowa.   
Ekapep,  ekapep,  aakisa  fan  unowa  Wewak-pa  nan  urupemik. 
Ne  ekapep  Numbia-pa  nan  urupemik. 
Maa  unowa  ifer  aasa-ke  purupeya  miiw-aasa-ke fan  piram-ikua.
Yaapan  feenap  Madang  kame  ikiwemik. 
Ikoka  Yaapan-ke  i  emeria  yia  aawurumikuan naeya,  i  amirika  owowa  ewur  me ekapem-ikemik,  Yaapan  fan  naap  ekapem-ikaiwkin. 
Wi  Yaapan  emeria  weetak,  mua  manekiw. 
Me  fan  Madang  kamepa  ekapemik, Wewak  kame-pa  fan  naap  iram-ikemik. 
Wilkar,  miiw-aasa,  Madang  naap  irakowa  ikiwem-ikemik, wi  Amerika  wiamiya  irakowa  naep. 
I  owowa-pa  fan  yiar  ikemik. 

Uura  feenap  nain,  wi  wilkar  nain  mufemi “Wensa,  wensa,  wensa”,  naap  kirirem-ikemik.
Wi  owow  muake  wilkar  wia mufem-ikomamik,  mua  kui-kuisow wia  maakiwkin.
“I  wilkar  yia  mufomaka,”  naiwkin. 
Wi  Malalake  mufep  ekapemi  i  Moro mua  wia  wuomam-ikomamik.
Wilkar  nain  wiena  maa  koorma  unowa  ipariwkin wia  mufem-ikomamik, feenap  Madang  pikiwem-ikemik, Australia  wiamiya  irakowa  nain. 

Nepa  opaimika  me  baliwep  amisarep wiena  opaimikiw  yia  maakem-ikemik. 
Aria  ‘taro’  yia  naiwkin  miimap maem-ikemik,  “Wi  moma  yia  maakimik,  moma-ko  wiiyen.” 
“Banana  mau  oo,  yasi  kongkang,” iwera yia iromaka, “banana  mau  oo  yasi  kongkang.” 
Iwera  “yasi”  yia  naem-ikemik. 
Naeya  iwera  wia  urukam-ikomamik. 
Iwera  urukam-ikaiwkin  wi  ikiwemi  aawem-ikemik,  iwera-ke  wia  arufeya   maem-ikemik,  “Oo  kanaka  oo,  yasi  paitim  mi  oo,” naem-ikemik. 
I  me  wia  amukaremik,  wis  pun  naap, i  me  yia  damolamik. 

Ne  eka  opora  biiris  marew  nain  wiena  onam-ikemik. 
Nemuru  biiris  onamik. 
Biiris  me  eliwa,  damo-damola-ko. 
Dabuel  poka-poka,  nomokowa  galua-galua,  nainiw  biiris  onam-ikemik. 
Ikoka  kuisow  miiw-aasa-ke  karueya  ku-kuep   orom-ikua. 
Ne  uurika  naap  nain  nainiw,  mauwam-ikemik. 
Waaya  yia  naiwkin  waaya  wienakem-ikemik. 

Aria  naap  ikok  wi  Australia  kereremik. 
Ne  wi  Yaapan  ikiwep  Ulingan-pa  owowa waremik. 
Ne  nan  wi  owow  mua  wia  maakemik, “Ni  kanaka  yia  ufomaka.” 
Naiwkin  wia  ufomamik. 
I  Moro  wenup  ufemik, i  bidaru  ufemik. 
Ufem-ikaiwkin,  Amerika  irakow  iinan  aasa  ekapep  Ulingan  nan  bom  fu-fuurik-ikiwemik. 
I  iinan  aasa  me  kuufamik. 
Iinaniw  iinaniw  wuami  feenap  Wewak  kame naap ikiwok.

Ne  Sarak  opaimika  wiar  paayarek,  siiwa,  epa  maakemik  nain  paayarep  maek, “Amerika  aakisa  irakowa  kereremik,”  naep  wi  Yaapan  wia  ufomowa  ikiwemik  nain  wia  maakek, “Ni  ufowa  ikiweka,  yo  miatiniyem. 
Irakowa  maneka  ikoka  kererinon. 
Ni  ufep-na  ni  maadara  me iirareka,  mokok  urupa  kaikiman  nain  feekiya Nomon  owowa,  Medebur  karueka,  baurareka.” 

Ne  wi  ikiwiwkin  Amerika  iinan  aasa  nainiw  kirek. 
Iinaniw  iki(we)p  kiripap  eneniw, eneniw  wuak  ne  i  fiker  gone-pa  ikemkun  ekapok. 
Malala  suule  ikua  nain,  Siburten  ema,  oram  tene-tenep  orop  epa  iiwawun  ifemakep  ekapemik. 
Ne  bomiya,  kateresiya,  bomiya,  kateresiya. 
Koora  pun  ariwa-ke  kuumeya  awomakek,  i  koora  kuisow  yiar  awok. 
Ne  yena  aamun  ariwa-ke  aawok, weepa  Aduna  ikos  iimarep  ikaiwkin,  owow  erepura-pa  fan. 
Wi  Amerika-ke  waremik. 
Yena  aamun  unuma  Saawoka. 

I  fiker  gone-pa  ikemkun  kiikir  iinan  aasa-ke maifa  fu-fuurk-ikiwok,  wi  Amerika-ke. 
Ulingan  fa-na  iinan  aasa  nepa  saarik,  unow  akena. 
Wi  Yaapan  nan  ikemik  nain  wia  urufap. 
I  karanemik,  yena  mua  Sarak-na  fiker  gone  nomokowa  onoma-pa  inep  ikua. 
Ne  yo  iki(we)p  mesa  asia  aawem-ikem, mesa  asia  fiker  gone-pa  ikaiya  nain  aawem-ikem. 
Ne  iinan  aasa  urufap  fiker  tepak  iwap naap  ikem,  ne  iinan  aasa  yia  nomakep ikiweya  Sarak  farem,  “Sarak,  yo  damolalem oo,  fiker  fufa  iwam  oo!” 
“Mauwa  aren,  amiaiya  nenaremik-i?”  Sarak-ke. 
“Wia,  nos-na,  yo  fiker  fufa  iwap nefa  fariyem.” 
“Momora,  no  naap  me  mae. 
Nefa  wariwkin  naap  mae. 
Yo  kema  efa  bagiwirowan,  yaa.” 
Yo  me  efa  waremik,  yo  iinan  aasa  fan orom-ikomakeya  mesa  fufa-pa  fan  erewarep  iima  ifemakep  ikem. 

Irakowa  nain  kekanowa  akena,  Amerika  kererek  nain,  ne  irakowa  naap  ikua. 
Uura  feenap  nain,  i  me  inem-ikemik,  amirika maa  me  enem-ikemik. 
Maa  uura  uupep  enem-ikemik. 
Ikoka  epia  wilinowa  urufap  bom  yia   fuurkomikuan  naep. 
Iinan  aasa  gurunowa  miimap  ekaiw   umukowa  ewur. 
Epia  wilinowa  urufap  bom  yia  wafuromikuan  naep  naap  onam-ikemik. 
I  Sarak  ikos  owowa  ekapem-ikemik. 
Yiena  koora  pun  me  awok. 

Wi  yapen-pa  ikomakiwkin  Amerika  kererep  “Eliwa,  eliwa”  nae-ekapemik,  “irakowa  weeserek.” 
Wi  Yaapan  pun  iiwawun  ikiwurumemik. 
Wi  Yaapan  Madang  kame  ikiwiwkin  Amerika  irakowa mua  urupemik. 
Urupep  koka-koka-pa  nan,  “Yapen  oo,  Luluai,  Tultul,   ni  kaaneke  ikeman  oo,  ni  ekapomakeka  oo!” 
Wi  Amerika  fan  “Epa  eliwa”  nae-ekapemik,  “Yaapan  weeseremik.” 
Naap  yia  maake-miawemik. 

I  amirikiw  yapen  yiena  Gawar  wiar ikiwemik. 
Ne  Sarak  ikos  Gawar-pa  ikemkun  yia  maakemik,  “Sarak  oo,  Amerika  ekapep  Ulingan  nan  ikemik, nefa  ikumimik  naimik  oo.” 
“Aria,  Kalina,  Amerika  ekapemik  naimik. 
I  oru”. 

Aria  i  yapen-pa  ikok  owowa  oromik. 
Oromi  yo  koka  koora-pa  nan  efa  wuami  maek,  “No  feeke  ike,  yo  Amerika  wia  akup-ikiwiyem.” 

Aria  yo  nan  efa  wuap-puami o  Ulingan  ikiwok. 
Ikiwok-na  wi  Amerika  maneka  unuma  Magerka,  o  kererep  nan  ikua. 
O  ikip  urufeya  maakek,  “O  Sarak,  no    kaaneke  ikok  kereren  a? 
“Yo  koka-pa  ikem.” 
Naeya  Magerkake,  “No  kamenap  ikon  noma? 
Yo  irakowa  nomakem. 
Nomakep  fan  kererem.” 
“Yo  bom  yo  kateres  koor  miira-pa  efar orom-ikua. 
Nain  yo  me  baurarem-ikem. 
I  iinan  aasa  ekapem-ikaeya  uruwa  ainep orop  yena  koor  miira-pa  iimarep  ikem,  yena  emeria  sosora  ainep  orop  koor  miira-pa  iimarem-ikua. 
Ne  i  me  yiaremik,  Amerika  iinan  aasa  ekapem-ikua  nain. 
Amina,  woowa,  eka  napia  koor  miirapa  iimaraw-ikiwemik,  yena  emeria  ikos,  ne  wi  unowa  bauraremik. 
I  iisow  naap  ikemkun  Amerika  kereremik.” 
“Ni  yapen  koka-kokapa  wiar  inem-ikeman nain  kereromakeka.” 

Amerika  fan  ekapep  Uliganpa  manua  maneka   urupep  irakow(a)  mua  wia  wueya  nan  ikemik. 
Palauwa  tin,  maa  eneka  kes  mane-maneka  fa,  oram  iwemik,  Sarak  iwemik. 
“No  fain  wiimep  amapikiwe.” 
O  wi  Laman,  Malager  naap  wia  maakek, “Arika,  takira,  yo  yookeka.” 
Amerika  kereremik  naeya  ikiwep-ko  wia  urufikua. 
O  Ulingan  ikiweya  bom  iwemik. 
“No  bom  fainiw  mera  kuume,”  naakemik. 
Maa  enowa  iwemik,  palauwa  dram,    bata  tin  naap  nain,  ne  rais  weetak. 

Ne  irakowa  weesereya  i  owowa  oromik. 
Irakowa  fa,  opaimika  eeya  akena  yaa,  kamenap  aakuniyen  yaa? 
Aakuniyen,  aakuniyen,  me  pepek,  me  welawiyen. 

%\setcounter{page}{131} \\ 
\section{Uncle Tup}\label{app:2:uncle}
by Saror Aduna
\ea
\gll  Yo  yena  yaiya  Tup  ifa  ku-o-k  nain  opaimika     ma-i-yem. \\
1s.\textsc{unm}  1s.\textsc{gen}  uncle  Tup  snake  bite-\textsc{pa}-3s  that1  talk  say-Np-\textsc{pr}.1s \\
\glt ‘I tell a story about that when my uncle Tup was bitten by a snake.’ \\
\z


\ea
\gll  Ae,  o  fiker  gone  urup-o-k. \\
yes  3s.\textsc{unm}  kunai.grass  middle  ascend-\textsc{pa}-3s \\
\glt ‘Yes, he went up to the middle of the kunai grass (area).’ \\
\z


\ea
\gll  Fikera  aw-em-ik-eya  nain  umuk-i-nen                       na-ep  urup-o-k. \\
kunai.grass  burn-\textsc{ss}.\textsc{sim}-be-2/3s.\textsc{ds}  that1  extinguish-Np-\textsc{fu}.1s    say-\textsc{ss.seq}  ascend-\textsc{pa}-3s \\
\glt ‘The kunai grass was burning and he went up to extinguish it.’ \\
\z


\ea
\gll  Urup-ep  ek-a-k. \\
ascend-\textsc{ss.seq}  go.eastwards-\textsc{pa}-3s \\
\glt ‘He went up and eastwards.’ \\
\z


\ea
\gll  Ek-ap  umuk-i-nen  na-ep                        on-am-ik-eya  ifa=ke  keraw-a-k,  mamepaperuma             gele-gelemuti-tik  nain=ke. \\
go.eastwards-\textsc{ss.seq}  extinguish-Np-\textsc{fu}.1s  say-\textsc{ss.seq}  do-\textsc{ss}.\textsc{sim}-be-2/3s.\textsc{ds}  snake-\textsc{cf}  bite-\textsc{pa}-3s  death.adder    \textsc{rdp}-small-\textsc{rdp}  that1-\textsc{cf} \\




\glt ‘He went eastwards and when he was trying to extinguish the fire a snake bit him, the very small death adder.’ \\
\z


\ea
\gll  Keraw-eya,  aria  nomokowa  gelemuta  puuk-ap  ifa  makena  nain  ifakim-o-k. \\
bite-2/3s.\textsc{ds}  alright  tree  small  cut-\textsc{ss.seq}  snake  being      that1  kill-\textsc{pa}-3s. \\


\glt ‘It bit him, and he cut a small tree and killed the snake.’ \\
\z


\ea
\gll  Ifakim-ep,  nomokowa  ekeka=pa  sererim-ep-pu-a-k. \\
kill-\textsc{ss.seq}  tree  branch-\textsc{loc}  hang-\textsc{ss.seq}-\textsc{cmpl}-\textsc{pa}-3s \\
\glt ‘He killed it and hung it on a tree branch.’ \\
\z


\ea
\gll  Sererim-ep-pu-ap  or-o-k,  owowa  or-o-k. \\
hang-\textsc{ss.seq}-\textsc{cmpl}-\textsc{ss.seq}  descend-\textsc{pa}-3s  village  descend-\textsc{pa}-3s \\
\glt ‘He hung it and came down, he came down to the village.’ \\
\z


\ea
\gll  Owowa  or-op,  wuailal-ep,  akia  ik-e-k. \\
village  descend-\textsc{ss.seq}  hunger-\textsc{ss.seq}  banana  roast-\textsc{pa}-3s \\
\glt ‘He came down to the village and was hungry and roasted bananas.’ \\
\z


\ea
\gll  Akia  ik-ep  en-em-ik-ok,  ifa  marasin  nain=ke              kema  wiar  iw-a-k. \\
banana  roast-\textsc{ss.seq}  eat-\textsc{ss}.\textsc{sim}-be-\textsc{ss}  snake  medicine  that1-\textsc{cf}   liver  3.\textsc{dat}  enter-\textsc{pa}-3s \\


\glt ‘He roasted bananas and when he was eating them the snake poison entered his liver.’ \\
\z


\ea
\gll  Iw-aya  nan  miira  saawirin-e-k. \\
enter-2/3s.\textsc{ds}  there  face  become.round-\textsc{pa}-3s \\
\glt ‘It entered his liver and he felt dizzy there.’ \\
\z


\ea
\gll  Ne  auwa  ame  wia  maak-eya  res  aaw-ep                  merena  ifa  keraw-a-k  nain  puuk-a-mik. \\
\textsc{add}  1s/p.father  \textsc{ass}  3p.\textsc{acc}  tell-2/3s.\textsc{ds}  razor  take-\textsc{ss.seq}  leg  snake  bite-\textsc{pa}-3s  that1  cut-\textsc{pa}-1/3p \\


\glt ‘And when he told my father and others, they took a razor and made a cut into the leg that the snake had bitten.’ \\
\z


\ea
\gll  Puuk-ap  marasin  wu-om-a-mik. \\
cut-\textsc{ss.seq}  medicine  put-\textsc{ben}-\textsc{bnfy}2.\textsc{pa}-1/3p \\
\glt ‘They made a cut and put medicine into it.’ \\
\z


\ea
\gll  Marasin  wu-om-a-mik  na  weetak. \\
medicine  put-\textsc{ben}-\textsc{bnfy}2.\textsc{pa}-1/3p  \textsc{tp}  no \\
\glt ‘They put medicine but no (it didn’t help).’ \\
\z


\ea
\gll  Iiriw  ifa  marasin=ke  kekan-e-k. \\
earlier  snake  medicine-\textsc{cf}  be.strong-\textsc{pa}-3s \\
\glt ‘The snake poison was already strong.’ \\
\z


\ea
\gll  Ne  akia  ik-e-k  nain  me  en-e-k. \\
\textsc{add}  banana  roast-\textsc{pa}-3s  that1  not  eat-\textsc{pa}-3s \\
\glt ‘And he didn’t eat the bananas that he roasted.’ \\
\z


\ea
\gll  Nan  mukuna=pa  ik-eya  o  nan  samor  aaw-o-k. \\
There  fire-\textsc{loc}  be-2/3s.\textsc{ds}  3s.\textsc{unm}  there  badly  get-\textsc{pa}-3s \\
\glt ‘They were there on the fire and he really got bad there.’ \\
\z


\ea
\gll  Miira  saawirin-e-k. \\
face  become.round-\textsc{pa}-3s \\
\glt ‘He felt dizzy.’ \\
\z


\ea
\gll  Ne  wi  emeria  papako  wia  maak-e-k,  “Ni  ifa       nia  keraw-i-ya  nain  sira  kamenap  on-i-man?” \\
\textsc{add}  3p.\textsc{unm}  woman  some  3p.\textsc{acc}  tell-\textsc{pa}-3s  2p.\textsc{unm}  snake  2p.\textsc{acc}  bite-Np-\textsc{pr}.3s  that1  custom  what.like  do-Np-\textsc{pr}.2p \\


\glt ‘And he asked some women, “When a snake bites you, what do you do?” ’ \\
\z


\ea
\gll  Ne  maak-e-mik,  “Ifa  yia  keraw-i-ya  nain      miira  saawirin-i-mik,  ookakim-i-nan.” \\
\textsc{add}  tell-\textsc{pa}-1/3p  snake  1p.\textsc{acc}  bite-Np-\textsc{pr}.3s  that1  face  become.round-Np-\textsc{pr}.1/3p  vomit-Np-\textsc{fu}.2s \\


\glt ‘And the told him, “When a snake bites us, we feel dizzy, you will vomit.’ \\
\z


\ea
\gll  Naap  maak-iwkin  naap  ik-ua. \\
Thus  tell-2/3p.\textsc{ds}  thus  be-\textsc{pa}.3s \\
\glt ‘They told him like that and he was like that.’ \\
\z


\ea
\gll  Naap  ik-ok  uruf-am-ika-iwkin  wia. \\
thus  be-\textsc{ss}  see-\textsc{ss}.\textsc{sim}-be-2/3p.\textsc{ds}  no \\
\glt ‘He was like that and they watched him but no (he didn’t get better.)’ \\
\z


\ea
\gll  Epa  naap  kokom-ar-eya  o  lawisiw  samor         aaw-o-k. \\
place  thus  dark-\textsc{inch}-2/3s.\textsc{ds}  3s.\textsc{unm}  somewhat  badly  get-\textsc{pa}-3s \\


\glt ‘When it got dark he became worse.’ \\
\z


\ea
\gll  Ne  haussik  p-ek-u  na-ep  miiw-aasa  nop-a-mik. \\
\textsc{add}  aidpost  Bpx-go-\textsc{imp}.1d  say-\textsc{ss.seq}  land-canoe  search-\textsc{pa}-1/3p \\
\glt ‘And they looked for a vehicle to take him to the aidpost.’ \\
\z


\ea
\gll  Miiw-aasa  nop-ap,  iiriw  Naawura  miiw-aasa  awona  nain     wiar  aaw-e-mik. \\
land-canoe  search-\textsc{ss.seq}  earlier  Naawura  land-canoe  old  that1  3.\textsc{dat}  get-\textsc{pa}-1/3p \\


\glt ‘They looked for a vehicle and got Naawura’s earlier old vehicle/truck.’ \\
\z


\ea
\gll  Aaw-ep  p-ikiw-e-mik,  haussik. \\
get-\textsc{ss.seq}  Bpx-go-\textsc{pa}-1/3p  aidpost \\
\glt ‘They got it and took him to the aidpost.’ \\
\z


\ea
\gll  Saapara  p-ek-ap,  Saapara=pa  neeke  ikemika  kaik-ow(a) mua  pun  iiriw  ona  owowa  Medebur  ek-a-k. \\
Saapara  Bpx-go-\textsc{ss.seq}  Saapara-\textsc{loc}  there.\textsc{cf}  wound  tie-\textsc{nmz}   man  also  earlier  3s.\textsc{unm}  village  Medebur  go-\textsc{pa}-3s \\


\glt ‘They took him to Saapara, and there in Saapara (they found that) the health officer too had already gone to his village, Medebur.’ \\
\z


\ea
\gll  Ne  kiiriw  nan  Medebur  ek-a-mik,  mua  napuma  onaiya. \\
\textsc{add}  again  there  Medebur  go-\textsc{pa}-1/3p  man  sick  with \\
\glt ‘And again from there they went to Medebur, with the sick man.’ \\
\z


\ea
\gll  Ek-ap  Medebur=pa  neeke  ikemika  kaik-owa  mua  nain  nop-a-mik,  imen-ap  maak-iwkin  o  miim-o-k. \\
go-\textsc{ss.seq}  Medebur-\textsc{loc}  there.\textsc{cf}  wound  tie-\textsc{nmz}  man  that1  search-\textsc{pa}-1/3p  find-\textsc{ss.seq}  tell-2/3p.\textsc{ds}  3s.\textsc{unm}  precede-\textsc{pa}-3s \\


\glt ‘They went and there in Medebur they searched for the health officer, and when they found him and told him he went ahead of them (to the aidpost).’ \\
\z


\ea
\gll  Aria,  wi  kiiriw  neeke  miiw-aasa  um-o-k. \\
alright  3p.\textsc{unm}  again  there.\textsc{cf}  land-canoe  die-\textsc{pa}-3s \\
\glt ‘Alright, again when they were there the truck broke down.’ \\
\z


\ea
\gll  Miiw-aasa  ume-ya  miiw-aasa  nain  on-am-ikai-wkin         epa  kokom-ar-e-k,  epa  iimeka  tuun-e-k. \\
land-canoe  die-2/3s.\textsc{ds}  land-canoe  that1  do-\textsc{ss}.\textsc{sim}-be-2/3p.\textsc{ds}  place  dark-\textsc{inch}-\textsc{pa}-3s  place/time  ten  count-\textsc{pa}-3s \\


\glt ‘The truck broke down and while they were working on the truck it became dark, it became past midnight.’ \\
\z


\ea
\gll  Ne  kiiriw  miiw-aasa  nan  ik-eya  mua  nain  nabena          suuw-a-mik. \\
\textsc{add}  again  land-canoe  there  be-2/3s.\textsc{ds}  man  that1  carrying.pole push-\textsc{pa}-1/3p \\


\glt ‘And again the truck stayed there and they carried the man on a carrying pole.’ \\
\z


\ea
\gll  Nabena  suuw-ap  ep-ap,  Saapara=pa  Ø. \\
carrying.pole  push-\textsc{ss.seq}  go-\textsc{ss.seq}  Saapara-\textsc{loc} \\
\glt ‘They carried him on a carrying pole, went (and arrived) at Saapara.’ \\
\z


\ea
\gll  Saapara=pa  nan  suusa  iw-e-mik,  wiena  ifa  suusa  nain. \\
Saapara-\textsc{loc}  there  needle  give.him-\textsc{pa}-1/3p  3p.\textsc{gen}  snake  needle  that1 \\
\glt ‘There in Saapara they gave him an injection, that snake injection of theirs.’ \\
\z


\ea
\gll  Nain  iw-iwkin,  weetak,  me  pepek. \\
that1  give.him-2/3p.\textsc{ds}  no  not  enough/able \\
\glt ‘They gave him that but no, it was not enough.’ \\
\z


\ea
\gll  O  iiwawun  samor  aaw-o-k. \\
3s.\textsc{unm}  altogether  badly  get-\textsc{pa}-3s \\
\glt ‘He got really bad.’ \\
\z


\ea
\gll  Ne  nan  ik-e-mik. \\
\textsc{add}  there  be-\textsc{pa}-1/3p \\
\glt ‘And they stayed there.’ \\
\z


\ea
\gll  Nan  ik-ok  ik-ok  neeke  pu-o-k. \\
there  be-\textsc{ss}  be-\textsc{ss}  there.\textsc{cf}  die-\textsc{pa}-3s \\
\glt ‘They stayed and stayed there and there he died.’ \\
\z


\ea
\gll  Neeke  pu-eya  oram  akua  aaw-e-mik. \\
there.\textsc{cf}  die-2/3s.\textsc{ds}  just  shoulder  take-\textsc{pa}-1/3p \\
\glt ‘There he died and they just carried him on their shouders.’ \\
\z


\ea
\gll  Miiw-aasa  samor-ar-e-k. \\
land-canoe  bad-\textsc{inch}-\textsc{pa}-3s \\
\glt ‘The truck broke/was broken.’ \\
\z


\ea
\gll  Miiw-aasa  samor-ar-eya  oram  akua  aaw-ep  ekap-e-mik. \\
land-canoe  bad-\textsc{inch}-2/3s.\textsc{ds}  just  shoulder  take-\textsc{ss.seq}  come-\textsc{pa}-1/3p \\
\glt ‘The truck was broken and they just carried him on their shoulders and came.’ \\
\z


\ea
\gll  Akua  aaw-ep  ekap-em-ika-iwkin  senam             pin-ar-e-k. \\
shoulder  take-\textsc{ss.seq}  come-\textsc{ss}.\textsc{sim}-be-2/3p.\textsc{ds}  too.much  heavy-\textsc{inch}-\textsc{pa}-3s \\


\glt ‘They carried him on their shoulders and as they were coming he got very heavy.’ \\
\z


\ea
\gll  Pin-ar-eya  aria  Kuten  ame=ke  miim-e-mik. \\
heavy-\textsc{inch}-2/3s.\textsc{ds}  alright  Kuuten  \textsc{ass}-\textsc{cf}  precede-\textsc{pa}-1/3p \\
\glt ‘He got heavy and so Kuuten and (some) others walked ahead.’ \\
\z


\ea
\gll  Miim-ep  ekap-ep  owow  mua  wia  maak-e-mik. \\
precede-\textsc{ss.seq}  come-\textsc{ss.seq}  village  man  3p.\textsc{acc}  tell-\textsc{pa}-1/3p \\
\glt ‘They came ahead and told the village people.’ \\
\z


\ea
\gll  Ne  owow  mua  wia  aaw-ep  kiiriw  kir-e-mik. \\
\textsc{add}  village  man  3p.\textsc{acc}  get-\textsc{ss.seq}  again  turn-\textsc{pa}-1/3p \\
\glt ‘And they got village men and turned back again.’ \\
\z


\ea
\gll  Kir-ep  ek-ap  nabena  suuw-ap  ep-am-ika-iwkin,                miiw-aasa  oko,  wi  Manub  miiw-aasa  Ø. \\
turn-\textsc{ss.seq}  go-\textsc{ss.seq}  carrying.pole  push  come-\textsc{ss}.\textsc{sim}-be-2/3p.\textsc{ds}   land-canoe  other  3p.\textsc{unm}  Manub  land-canoe \\


\glt ‘They turned and went and as they were coming carrying him, another truck, Manub village truck (took them).’ \\
\z


\ea
\gll  Miiw-aasa  awona  nain  miiw-aasa  fa-owa  mua=ke  neeke   wia  aaw-o-k. \\
land-canoe  old  that1  land-canoe  drive-\textsc{nmz}  man-\textsc{cf}  there.\textsc{cf}  3p.\textsc{acc}  take-\textsc{pa}-3s \\


\glt ‘The driver of that old truck took them from there.’ \\
\z


\ea
\gll  Neeke  wia  aaw-ep  ep-a-mik. \\
there.\textsc{cf}  3p.\textsc{acc}  take-\textsc{ss.seq}  come-\textsc{pa}-1/3p \\
\glt ‘He took them from there and they came.’ \\
\z


\ea
\gll  Ep-ap  ona  koora=pa  in-aw-ap  naap              arew-ap  pok-ap  ik-e-mik. \\
come-\textsc{ss.seq}  3s.\textsc{gen}  house-\textsc{loc}  sleep-\textsc{appl}-\textsc{ss.seq}  thus wait-\textsc{ss.seq}  sit-\textsc{ss.seq}  be-\textsc{pa}-1/3p \\


\glt ‘They came and laid him in his house and they/we sat and waited like that.’ \\
\z


\ea
\gll  Pok-ap  ik-omkun  epa  wiim-o-k. \\
sit-\textsc{ss.seq}  be-1s/p.\textsc{ds}  place  dawn-\textsc{pa}-3s \\
\glt ‘As we were waiting it dawned.’ \\
\z


\ea
\gll  Epa  wiim-eya  mua  karer-omak-e-mik. \\
place  dawn-2/3s.\textsc{ds}  man  gather-\textsc{distr}/\textsc{pl}-\textsc{pa}-1/3p \\
\glt ‘After it dawned many people gathered.’ \\
\z


\ea
\gll  Karer-ap  ma-e-mik,  mua  iperowa=ke  ma-e-mik,  “I             feeke  me  soop-i-yen,  ona  owowa  p-ikiw-i-yan. \\
gather-\textsc{ss.seq}  say-\textsc{pa}-1/3p  man  middle.aged-\textsc{cf}  say-\textsc{pa}-1/3p  1p.\textsc{unm}  here.\textsc{cf}  not  bury-Np-\textsc{fu}.1p  3s.\textsc{gen}  village  Bpx-go-Np-\textsc{fu}.1p \\


\glt ‘They gathered and said - the middle-aged men/elders said, “We will not bury him here, we’ll take him to his own village.” ’ \\
\z


\ea
\gll  Ne  wia,  papako=ke  ma-e-mik,  “Weetak,  moram  owowa  p-ikiw-i-yan? \\
\textsc{add}  no  some-\textsc{cf}  say-\textsc{pa}-1/3p  no  why  village  Bpx-go-\textsc{fu}.1p \\
\glt ‘But no, others said, “No, why take him to his village?’ \\
\z


\ea
\gll  Yiena  miiwa  kuisow,  eliw  feeke  soop-i-yen.” \\
1p.\textsc{gen}  land  one  well  here.\textsc{cf}  bury-Np-\textsc{fu}.1p \\
\glt ‘Our land is one, we can well bury him here.” ’ \\
\z


\ea
\gll  Ne  wia.  Iperowa=ke  senam  kekan-e-mik. \\
\textsc{add}  no  middle.aged-\textsc{cf}  too.much  be.strong-\textsc{pa}-1/3p \\
\glt ‘But no. The elders insisted.’ \\
\z


\ea
\gll  Kekan-iwkin  ma-e-mik,  “Aria,  iperowa  opora  wia            ook-i-yen. \\
be.strong-2/3p.\textsc{ds}  say-\textsc{pa}-1/3p  alright  middle.aged  talk  3p.\textsc{acc}   follow-Np-\textsc{fu}.1p \\


\glt ‘They insisted and we said, “Alright, we’ll follow the talk of the elders.’ \\
\z


\ea
\gll  Ona  owowa  p-ikiw-i-yan.” \\
3s.\textsc{gen}  village  Bpx-go-Np-\textsc{fu}.1p \\
\glt ‘We’ll take him to his village.” ’ \\
\z


\ea
\gll  Malala  miiw-aasa  wia  maak-e-mik. \\
Malala  land-canoe  3p.\textsc{acc}  tell-\textsc{pa}-1/3p \\
\glt ‘We talked to the owners of the Malala truck.’ \\
\z


\ea
\gll  Miiw-aasa  wia  maak-ep,  ekap-eya,  kes          on-om-a-mik. \\
land-canoe  3p.\textsc{acc}  tell-\textsc{ss.seq}  come-2/3s.\textsc{ds}  coffin   make-\textsc{ben}-\textsc{bnfy}2.\textsc{pa}-1/3p \\


\glt ‘We talked to them about the truck and when it came we made a coffin for him.’ \\
\z


\ea
\gll  Kes  tepak=pa  wu-ap  p-ikiw-e-mik. \\
coffin  inside-\textsc{loc}  put-\textsc{ss.seq}  Bpx-go-\textsc{pa}-1/3p \\
\glt ‘We put him inside the coffin and took him (away).’ \\
\z


\ea
\gll  P-ikiw-ep  Bogia=pa  nan  wu-a-mik. \\
Bpx-go-\textsc{ss.seq}  Bogia-\textsc{loc}  there  put-\textsc{pa}-1/3p \\
\glt ‘We took him and buried him in Bogia.’ \\
\z


\ea
\gll  Bogia=pa  nan  wu-ap  i  kiiriw  ekap-e-mik. \\
bogia-\textsc{loc}  there  put-\textsc{ss.seq}  1p.\textsc{unm}  again  come-\textsc{pa}-1/3p \\
\glt ‘We buried him in Bogia and came back again.’ \\
\z


\ea
\gll  Ekap-ep  uura  owowa  kerer-e-mik. \\
come-\textsc{ss.seq}  night  village  arrive-\textsc{pa}-1/3p \\
\glt ‘We came and arrived in the village at night.’ \\
\z


\ea
\gll  Opora  nain  muut  naap,  weeser-e-k. \\
talk  that1  only  thus  finish-\textsc{pa}-3s \\
\glt ‘The talk/story is just like that, (it is) finished.’ \\
\z

Yo  yena  yaiya  Tup  ifa  kuok  nain  opaimika     maiyem. 

Ae,  o  fiker  gone  urupok. 
Fikera  awem-ikeya  nain  umukinen  naep  urupok. 
Urupep  ekak. 
Ekap  umukinen  naep onam-ikeya  ifa-ke  kerawak,  mamepaperuma gele-gelemutitik  nain-ke. 
Keraweya,  aria  nomokowa  gelemuta  puukap  ifa  makena  nain  ifakimok. 
Ifakimep,  nomokowa  ekeka-pa  sererimep-puak. 
Sererimep-puap  orok,  owowa  orok. 
Owowa  orop,  wuailalep,  akia  ikek. 
Akia  ikep  enem-ikok,  ifa  marasin  nain-ke kema  wiar  iwak. 
Iwaya  nan  miira  saawirinek.

Ne  auwa  ame  wia  maakeya  res  aawep  merena  ifa  kerawak  nain  puukamik. 
Puukap  marasin  wuomamik. 
Marasin  wuomamik  na  weetak. 
Iiriw  ifa  marasin-ke  kekanek.
Ne  akia  ikek  nain  me  enek.

Nan  mukuna-pa  ikeya  o  nan  samor  aawok. 
Miira  saawirinek. 
Ne  wi  emeria  papako  wia  maakek,  “Ni  ifa nia  kerawiya  nain  sira  kamenap  oniman?” 
Ne  maakemik,  “Ifa  yia  kerawiya  nain  miira  saawirinimik,  ookakiminan.” 
Naap  maakiwkin  naap  ikua. 
Naap  ikok  urufam-ikaiwkin  wia.

Epa  naap  kokomareya  o  lawisiw  samor aawok. 
Ne  haussik  peku  naep  miiw-aasa  nopamik. 
Miiwaasa  nopap,  iiriw  Naawura  miiwaasa  awona  nain     wiar  aawemik. 
Aawep  pikiwemik,  haussik. 
Saapara  pekap,  Saapara-pa  neeke  ikemika  kaikow(a) mua  pun  iiriw  ona  owowa  Medebur  ekak.

Ne  kiiriw  nan  Medebur  ekamik,  mua  napuma  onaiya. 
Ekap  Medebur-pa  neeke  ikemika  kaikowa  mua  nain  nopamik,  imenap  maakiwkin  o  miimok.

Aria,  wi  kiiriw  neeke  miiw-aasa  umok. 
Miiwaasa  umeya  miiw-aasa  nain  onam-ikaiwkin  epa  kokomarek,  epa  iimeka  tuunek. 
Ne  kiiriw  miiw-aasa  nan  ikeya  mua  nain  nabena          suuwamik. 
Nabena  suuwap  epap,  Saapara-pa. 
Saapara-pa  nan  suusa  iwemik,  wiena  ifa  suusa  nain. 
Nain  iwiwkin,  weetak,  me  pepek.

O  iiwawun  samor  aawok.
Ne  nan  ikemik. 
Nan  ikok  ikok  neeke  puok. 
Neeke  pueya  oram  akua  aawemik. 
Miiw-aasa  samorarek. 
Miiw-aasa  samorareya  oram  akua  aawep  ekapemik. 
Akua  aawep  ekapem-ikaiwkin  senam  pinarek. 
Pinareya  aria  Kuten  ame-ke  miimemik. 
Miimep  ekapep  owow  mua  wia  maakemik.

Ne  owow  mua  wia  aawep  kiiriw  kiremik. 
Kirep  ekap  nabena  suuwap  epam-ikaiwkin,  miiw-aasa  oko,  wi  Manub  miiw-aasa. 
Miiw-aasa  awona  nain  miiw-aasa  faowa  muake  neeke   wia  aawok. 
Neeke  wia  aawep  epamik. 
Epap  ona  koorapa  inawap  naap  arewap  pokap  ikemik. 
Pokap  ikomkun  epa  wiimok. 
Epa  wiimeya  mua  kareromakemik. 
Karerap  maemik,  mua  iperowa-ke  maemik,  “I feeke  me  soopiyen,  ona  owowa  pikiwiyan.

Ne  wia,  papako-ke  maemik,  “Weetak,  moram  owowa  pikiwiyan? 
Yiena  miiwa  kuisow,  eliw  feeke  soopiyen.” 
Ne  wia.  Iperowa-ke  senam  kekanemik. 
Kekaniwkin  maemik,  “Aria,  iperowa  opora  wia  ookiyen. 
Ona  owowa  pikiwiyan.”

Malala  miiw-aasa  wia  maakemik. 
Miiw-aasa  wia  maakep,  ekapeya,  kes onomamik. 
Kes  tepak-pa  wuap  pikiwemik. 
Pikiwep  Bogia-pa  nan  wuamik. 
Bogia-pa  nan  wuap  i  kiiriw  ekapemik. 
Ekapep  uura  owowa  kereremik. 
Opora  nain  muut  naap,  weeserek. 

%\setcounter{page}{138} \\

\section{Catching a turtle}\label{app:2:turtle}
by Kuuten  
\ea
\gll  Yo  aakisa  opaimika=ko  ma-i-nen,  pon  aaw-e-m  nain. \\
1s.\textsc{unm}  now  speech-\textsc{if}  say-Np-\textsc{fu}.1s,  turtle  get-\textsc{pa}-1s  that1 \\
\glt ‘I will now tell a story about that when I caught a turtle.’ \\
\z


\ea
\gll  Yo  uura  arua  isim-ap  aasa=pa  karue-miaw-ik-ok           pon  sisina=pa  ik-eya,  piipa  unowa=pa               soomar-em-ik-eya  mik-a-m. \\
1s.\textsc{unm}  night  lantern  light-\textsc{ss.seq}  canoe-\textsc{loc}  run-wander-be-\textsc{ss}  turtle  shallow.water-\textsc{loc}  be-2/3s.\textsc{ds}  seaweed  many-\textsc{loc} walk-\textsc{ss}.\textsc{sim}-be-2/3s.\textsc{ds}  spear-\textsc{pa}-1s \\




\glt ‘At night I lighted a lantern and was paddling in the canoe and I saw a turtle in the shallow water, walking among a lot of seaweed, and I speared it.’ \\
\z


\ea
\gll  Mik-ap,  patot=iw  mik-ap,  aaw-ep,                    aasa=pa  wu-ap,  amap-urup-ep,  yena  koora=pa             wu-ap,  uuriw  epa  wiim-eya  or-op,  saa=pa  pa-ep  uup-e-mik. \\
 spear-\textsc{ss.seq}  fishing.spear-\textsc{inst}  spear-\textsc{ss.seq}  take-\textsc{ss.seq}  canoe-\textsc{loc}  put-\textsc{ss.seq}  Bpx-ascend-\textsc{ss.seq}  1s.\textsc{gen}  house-\textsc{loc}  put-\textsc{ss.seq}  morning  place  dawn-2/3s.\textsc{ds}  descend-\textsc{ss.seq}   beach-\textsc{loc}  butcher{}-\textsc{ss.seq}  cook-\textsc{pa}-1/3p \\






\glt ‘I speared it, speared it with a fishing spear, took it, put it in the canoe, brought it up (to the beach), put it in my house, (and) in the morning when it dawned went down and butchered it on the beach and we cooked it.’ \\
\z


\ea
\gll  Uup-ep  weeser-eya,  aria  oposia  gelemuta  wiam  erup     fain  wia  wu-om-a-m. \\
 cook-\textsc{ss.seq}  finish-2/3s.\textsc{ds}  alright  meat  little  3p.\textsc{refl}  two  this  3p.\textsc{acc}  put-\textsc{ben}-\textsc{bnfy}2.\textsc{pa}-1s \\


\glt ‘When the cooking was finished, alright I put a little (aside) for these two (women).’ \\
\z


\ea
\gll  En-ep  uruf-ap  efa  maak-e-mik,  “No  opaimika     pon  aaw-o-n  nain  ma-eya  i  miim-i-yen. \\
eat-\textsc{ss.seq}  see-\textsc{ss.seq}  1s.\textsc{acc}  tell-\textsc{pa}-1/3p  2s.\textsc{unm}  speech  turtle  get-\textsc{pa}-2s  that1  say-2/3s.\textsc{ds}  1p.\textsc{unm}  hear-Np-\textsc{fu}.1p \\


\glt ‘They ate and saw (what it was like) and told me, “Tell us the story of how you you caught the turtle, and we’ll listen.” ’ \\
\z


\ea
\gll  Ne  yo  aakisa  tep=pa  ma-i-yem. \\
\textsc{add}  1s.\textsc{unm}  now  tape.recorder-\textsc{loc}  speak-Np-\textsc{pr}.1s \\
\glt ‘And now I speak it on a tape recorder.’ \\
\z


\ea
\gll  En-e-mik  na  ma-e-mik,  “Eliwa,  aara  oposia  saarik.” \\
eat-\textsc{pa}-1/3p  and(\textsc{tp})  say-\textsc{pa}-1/3p  good  chicken  meat  like \\
\glt ‘They ate it and said, “(It is) good, like chicken meat.” ’ \\
\z


\ea
\gll  Ne  yo  efa  maak-iwkin  yo  aakisa       tep=pa  ma-e-m. \\
 \textsc{add}  1s.\textsc{unm}  1s.\textsc{acc}  tell-2/3p.\textsc{ds}  1s.\textsc{unm}  now  tape.recorder-\textsc{loc}  speak-\textsc{pa}-1s \\


\glt ‘And they told me and now I spoke it on a taperecorder.’ \\
\z


\ea
\gll  Yo  opaimika  muut  nan-e-k. \\
1s.\textsc{unm}  speech  only  be.there-\textsc{pa}-3s \\
\glt ‘My speech is there.’ \\
\z


Yo  aakisa  opaimika-ko  mainen,  pon  aawem  nain. 
Yo  uura  arua  isimap  aasapa  karue-miawikok   pon  sisina-pa  ikeya,  piipa  unowa-pa soomarem-ikeya  mikam. 
Mikap,  patotiw  mikap,  aawep,  aasa-pa  wuap,  amapurupep,  yena  koora-pa wuap,  uuriw  epa  wiimeya  orop,                     saa-pa  paep  uupemik. 
Uupep  weesereya,  aria  oposia  gelemuta  wiam  erup fain  wia  wuomam. 
Enep  urufap  efa  maakemik,  “No  opaimika     pon  aawon  nain  maeya  i  miimiyen. 
Ne  yo  aakisa  tep-pa  maiyem. 
Enemik  na  maemik,  “Eliwa,  aara  oposia  saarik.” 
Ne  yo  efa  maakiwkin  yo  aakisa  tep-pa  maem. 
Yo  opaimika  muut  nanek. 

%\setcounter{page}{140} \\

\section{Fishing customs}\label{app:2:fishing}
by Saror Aduna
\ea
\gll  Mera  aaw-owa  sira  e  era  wapen  in-aw-i-ya        okaiwi=pa  kuisow. \\
fish  get-\textsc{nmz}  custom  or  way  hand  sleep-\textsc{appl}-Np-\textsc{pr}.3s   other.side-\textsc{loc}  one \\


\glt ‘There are six ways to do fishing.’ \\
\z


\ea
\gll  Kaul=iw  aaw-i-mik,  arua  karu-i-mik,     maer  puuk-i-mik,  patopat=iw  mera                urum-i-mik,  oko  galasim-i-mik. \\
hook-\textsc{inst}  get-Np-\textsc{pr}.1/3p  torch  run-Np-\textsc{pr}.1/3p surface  cut-Np-\textsc{pr}.1/3p  fishing.spear-\textsc{inst}  fish  watch-Np-\textsc{pr}.1/3p  other  dive.with.goggles-Np-\textsc{pr}.1/3p \\




\glt ‘They (or: we) catch them with hooks, fish with torches, hit the surface, search the fish with fishing spears and dive with goggles.’ \\
\z


\ea
\gll  Oko  soo. \\
other  fishtrap \\
\glt ‘Another (way) is a fishtrap.’ \\
\z


\ea
\gll  Soo  nain  feenap:  era  erup  ik-ua. \\
fishtrap  that1  like.this  way  two  be-\textsc{pa}.3s \\
\glt ‘Fishtrap is like this: there are two ways.’ \\
\z


\ea
\gll  Oko  sisina=pa  ifemak-i-mik,  oko  malol=pa               ifemak-i-mik. \\
other  shallow.water-\textsc{loc}  press-Np-\textsc{pr}.1/3p  other  deep.sea-\textsc{loc}  press-Np-\textsc{pr}.1/3p \\


\glt ‘One is that they let it down in the shallow water, the other is that they let it down in the deep sea.’ \\
\z


\ea
\gll  Ne  sisina  nain  me  yoowa  akena. \\
\textsc{add}  shallow.water  that1  not  hard  very \\
\glt ‘(Letting it down in) the shallow water is not very hard.’ \\
\z


\ea
\gll  Oo  malol  lawisiw  yoowa. \\
\textsc{intj}  deep.sea  a.little  hard \\
\glt ‘Oh the deep sea is a bit hard.’ \\
\z


\ea
\gll  Sisina  nain,  soo  ika  on-i-nan,              soo  ika=pa  kaik-ap  otal  opora=pa              ifemak-i-nan. \\
shallow.water  that1  fishtrap  stand  make-Np-\textsc{fu}.2s  fishtrap  stand-\textsc{loc}  tie-\textsc{ss.seq}  reef  mouth-\textsc{loc}  press-Np-\textsc{fu}.2s \\




\glt ‘As for the shallow water (style), you make a stand for the fistrap, tie the trap to the stand and let it down at the mouth of the reef.’ \\
\z


\ea
\gll  Ifemak-ep  nomona  iinan=pa  wua-i-nan,  ikoka  ifera  me  p-ikiw-owa  nain. \\
press-\textsc{ss.seq}  stone  on.top-\textsc{loc}  put-Np-\textsc{fu}.2s  later  sea    not  Bpx-go-\textsc{nmz}  that1 \\


\glt ‘You let it down and then put stones on top of it, so that later the sea won’t take it away.’ \\
\z


\ea
\gll  Soo  nain  ona  malin  saana=pa  ifemak-i-mik. \\
fishtrap  that1  3s.\textsc{gen}  calm  season-\textsc{loc}  press-Np-\textsc{pr}.1/3p \\
\glt ‘The fishtrap is let down during the calm season.’ \\
\z


\ea
\gll  Oo  ifera  ku-owa  epa=pa  weetak. \\
\textsc{intj}  sea  break-\textsc{nmz}  time-\textsc{loc}  no \\
\glt ‘Not during the stormy time.’ \\
\z


\ea
\gll  Aria  malol=pa  ifemak-i-mik  nain  aana        puuk-i-mik,  makera  unowa  puuk-ap             makera  nain  anetir-i-mik. \\
alright  deep.sea-\textsc{loc}  press-Np-\textsc{pr}.1/3p  that1  cane  cut-Np-\textsc{pr}.1/3p  makera.cane  many  cut-\textsc{ss.seq}   cane  that1  tie-Np-\textsc{pr}.1/3p \\




\glt ‘Alright the one that is let down in the deep sea (is like this), they cut cane, and after cutting a lot of \textit{makera} cane, tie the canes (making a long rope).’ \\
\z


\ea
\gll  Anetir-ikiw-ep  uruf-i-mik,  makera  nain  maaya  pepek. \\
tie-go-\textsc{ss.seq}  see-Np-\textsc{pr}.1/3p  cane  that1  long  enough \\
\glt ‘They keep tying them and then see that the cane is long enough.’ \\
\z


\ea
\gll  Aria  makera  miirifa  okaiwi  soo=pa  kaik-i-mik,       okaiwi  pia  kaik-i-mik,  piakina  na-i-mik  nain. \\
alright  cane  end  other.side  fishtrap-\textsc{loc}  tie-Np-\textsc{pr}.1/3p   other.side  bamboo  tie{}-Np-\textsc{pr}.1/3p  piakina  say-Np-\textsc{pr}.1/3p  \textsc{rm} \\


\glt ‘Alright one end of the cane is tied to the trap, to the other end they tie a (piece of) bamboo called \textit{piakina}.’ \\
\z


\ea
\gll  Ne  soo  ifemak-owa  epa=pa  aasa  suuw-i-mik. \\
\textsc{add}  fishtrap  press-\textsc{nmz}  time-\textsc{loc}  canoe  push{}-Np-\textsc{pr}.1/3p \\
\glt ‘And at the time of letting the trap down they push the canoe out.’ \\
\z


\ea
\gll  Aasa  suuw-ap  soo  aasa  iinan=pa  wu-ap           p-ora-i-mik. \\
canoe  push-\textsc{ss.seq}  fishtrap  canoe  top-\textsc{loc}  put-\textsc{ss.seq}  Bpx-descend-Np-\textsc{pr}.1/3p \\


\glt ‘They push the canoe out and put the trap on top of the canoe and take it down (to the sea).’ \\
\z


\ea
\gll  Or-op  malol=pa  soo  nain  fuurk-i-mik. \\
descend-\textsc{ss.seq}  deep.sea-\textsc{loc}  fishtrap  that1  throw{}-Np-\textsc{pr}.1/3p \\
\glt ‘They go down and at the deep sea they throw the fishtrap down.’ \\
\z


\ea
\gll  Fuurk-ap  makera  nain  op-ap  ika-i-kuan. \\
throw-\textsc{ss.seq}  cane  that1  hold-\textsc{ss.seq}  be-Np-\textsc{fu}-3p \\
\glt ‘They throw it and hold on to the cane.’ \\
\z


\ea
\gll  Op-ap  lawiliw  piipu-am-ika-i-kuan. \\
hold-\textsc{ss.seq}  a.little  leave-\textsc{ss}.\textsc{sim}-be-Np-\textsc{fu}.3p \\
\glt ‘They hold it and let go a little.’ \\
\z


\ea
\gll  Or-op  saa=pa  pok-aya  wi  piipua-i-kuan. \\
descend-\textsc{ss.seq}  sand-\textsc{loc}  sit-2/3s.\textsc{ds}  3p.\textsc{unm}  leave-Np-\textsc{fu}.3p \\
\glt ‘When it has gone down and sits on the sand they will let go of it.’ \\
\z


\ea
\gll  Ne  soo  nainiw  muf-owa  pun  naap,  aana=pa  neeke  mufimik. \\
\textsc{add}  fishtrap  again  pull-\textsc{nmz}  also  thus  cane-\textsc{loc}  there.\textsc{cf}  pull-Np-\textsc{pr}-1/3p \\


\glt ‘And the pulling (up) of the fishtrap is like that too, it is pulled by the cane there.’ \\
\z


\ea
\gll  Aana=pa  neeke  muf-ep,  p-urup-ep  aasa=pa  wua-i-mik. \\
cane-\textsc{loc}  there.\textsc{cf}  pull-\textsc{ss.seq}  Bpx-ascend-\textsc{ss.seq}     canoe-\textsc{loc}  put-Np-\textsc{pr}.1/3p \\


\glt ‘They pull it by the cane there, bring it up and put it in the canoe.’ \\
\z


\ea
\gll  Aasa=pa  wu-ap,  mera  aaw-ep  weeser-eya           nainiw  fuurk-i-mik. \\
canoe-\textsc{loc}  put-\textsc{ss.seq}  fish  get-\textsc{ss.seq}  finish-2/3s.\textsc{ds}   again  throw-Np-\textsc{pr}.1/3p \\


\glt ‘They put it in the canoe, get the fish and when that is finished they throw it down again.’ \\
\z


\ea
\gll  Nain  soo  era=ke. \\
that1  fishtrap  way-\textsc{cf} \\
\glt ‘That is the fishtrap way.’ \\
\z


\ea
\gll  Aria  maer  pun  naap. \\
alright  surface  also  thus \\
\glt ‘Alright (hitting the surface) is like that too.’ \\
\z


\ea
\gll  Wi  emeria  kaalal-i-mik,  kaalal-ep     or-op  otal=pa  maer  ar-i-mik. \\
3p.\textsc{unm}  woman  wade{}-Np-\textsc{pr}.1/3p  wade-\textsc{ss.seq}  descend-\textsc{ss.seq}  reef-\textsc{loc}  surface  become-Np-\textsc{pr}.1/3p \\


\glt ‘The women wade, they wade down and at the reef they start the surface-hitting.’ \\
\z


\ea
\gll  Maer  ar-ep  urup-ep  urup-ep,  pona. \\
surface  become-\textsc{ss.seq}  ascend-\textsc{ss.seq}  ascend-\textsc{ss.seq}  shore \\
\glt ‘They start hitting the surface and keep coming up towards the shore.’ \\
\z


\ea
\gll  Pona=pa  neeke  mera  nomona  ona  iw-omak-eya,    wi  wapen=iw  ona  nain  suuw-i-mik. \\
shore-\textsc{loc}  there.\textsc{cf}  fish  reef  hole  enter-\textsc{distr}/\textsc{pl}-2/3s.\textsc{ds}  3p.\textsc{unm}  hand-\textsc{inst}  hole  that1  push{}-Np-\textsc{pr}.1/3p \\


\glt ‘When they have reached the shore, a lot of fish have entered holes in the reef and they (the women) push their hands in the holes (and pull the fish out).’ \\
\z


\ea
\gll  Ne  mua  patopat  aaw-ep  saa=pa  iimar-ep                  ika-i-kuan. \\
\textsc{add}  man  fishing.spear  take-\textsc{ss.seq}  sand-\textsc{loc}  stand.up-\textsc{ss.seq}   be-Np-\textsc{fu}.3p \\


\glt ‘And the men will take the fishing spears and be standing on the beach.’ \\
\z


\ea
\gll  Mera  sisina  urup-eya  patopat=iw                 mik-i-kuan. \\
fish  shallow.water  ascend-2/3s.\textsc{ds}  fishing.spear-\textsc{inst}  spear-Np-\textsc{fu}.3p \\


\glt ‘When the fish comes up to the shallow water they will spear it with a fishing spear.’ \\
\z


\ea
\gll  Ne  oko  galasim-owa. \\
\textsc{add}  other  dive.with.goggles-\textsc{nmz} \\
\glt ‘And another (way) is diving with goggles.’ \\
\z


\ea
\gll  Nain  iiriw  me  kerer-e-k,  aakisa  fan. \\
that1  earlier  not  appear-\textsc{pa}-3s  now  here. \\
\glt ‘That didn’t come up in the old days, (but) just recently.’ \\
\z


\ea
\gll  Wiena  galasim-owa  amia  on-i-mik,           on-ap  galasim-i-mik. \\
3p.\textsc{gen}  dive.with.goggles-\textsc{nmz}  gun  make-Np-\textsc{pr}.1/3p make-\textsc{ss.seq}  dive.with.goggles-Np-\textsc{pr}.1/3p \\


\glt ‘They make their own fishing guns, and having made them they fish diving with goggles.’ \\
\z


\ea
\gll  Ne  oko,  mera  urum-i-mik,  patopat=iw. \\
\textsc{add}  other  fish  search-Np-\textsc{pr}.1/3p  fishing.spear-\textsc{inst} \\
\glt ‘And another (way): they watch for fish (and spear them) with a fishing spear.’ \\
\z


\ea
\gll  Nain  ona  mua  taraka  nain=ke  mera  mik-i-nan,       patopat  opo-wa  nain. \\
that1  3s.\textsc{gen}  man  accuracy  that-\textsc{cf}  fish  spear-Np-\textsc{fu}.2s   fishing.spear  hold-\textsc{nmz}  that1. \\


\glt ‘If you are an accurate man, you will spear the fish, (accurate in) holding the spear.’ \\
\z


\ea
\gll  O  mua  naap  nain  ikoka  ufer-i-nan. \\
3s.\textsc{unm}  man  thus  that1  later  miss-Np-\textsc{fu}.2s \\
\glt ‘A man with poor aim (literally: so-so) will miss.’ \\
\z


\ea
\gll  Ne  oko,  afukar-i-mik. \\
\textsc{add}  other  fish.with.torch-Np-\textsc{pr}.1/3p \\
\glt ‘And another (way): they do torch-fishing.’ \\
\z


\ea
\gll  Parina  isim-ap  afukar-i-mik. \\
lamp  light-\textsc{ss.seq}  fish.with.torch-Np-\textsc{pr}.1/3p \\
\glt ‘They light lamps and do torch-fishing.’ \\
\z


\ea
\gll  Parina  isim-ap  afukar-ep,  nain            ona  kak  saana=pa. \\
lamp  light-\textsc{ss.seq}  fish.with.torch-\textsc{ss.seq}  that1  3s.\textsc{gen}  flying.fish  season-\textsc{loc} \\


\glt ‘They light lamps and do torch-fishing and – that (takes place) in the flying fish season.’ \\
\z


\ea
\gll  Kak  saana=pa  wi  parina  isim-ap              or-op  kak  isak-i-mik. \\
flying.fish  season-\textsc{loc}  3p.\textsc{unm}  lamp  light-\textsc{ss.seq}   descend-\textsc{ss.seq}  flying.fish  spear-Np-\textsc{pr}.1/3p \\


\glt ‘In the flying fish season they light lamps and go down (to the sea) and spear flying fish.’ \\
\z


\ea
\gll  Kak,  pirit,  mera  papako. \\
Flying.fish  longtom  fish  other \\
\glt ‘Flying fish, longtom and other fish.’ \\
\z


\ea
\gll  Tokol  gelemutitik,  kookari,  nain  isak-i-mik. \\
dussumier’s.garfish,  small-\textsc{rdp}  fish.sp  that1  spear-Np-\textsc{pr}.3p \\
\glt ‘They spear small dussumier’s garfish, \textit{kookari} fish, (all) those.’ \\
\z


\ea
\gll  Ne  kak  saana=pa  kak  ora-i-mik,  ne                      parina  ona  wakesim-owa  onaiya  ika-i-ya. \\
\textsc{add}  flying.fish  season-\textsc{loc}  flying.fish  descend-Np-\textsc{pr}.3p  \textsc{add}  lamp  3s.\textsc{gen}  cover-\textsc{nmz}  with  be-Np-\textsc{pr}.3s \\


\glt ‘And in the flying fish season they go down for flying fish, and the lamp has a covering.’ \\
\z


\ea
\gll  Ikoka  wakesim-eya  mera  mamaiya  ekap-i-non,  aasa  mamaiya. \\
later  cover-2/3s.\textsc{ds}  fish  close  come-Np-\textsc{fu}.3s      canoe  close \\


\glt ‘Later when you cover it the fish will come close, close to the canoe.’ \\
\z


\ea
\gll  Aasa  mamaiya  ekap-eya  aria  parina  kiiriw    mauwa  on-e. \\
canoe  close  come-2/3s.\textsc{ds}  alright  lamp  again       what  do-\textsc{imp}.2s \\


\glt ‘When they come close to the canoe, then again do what (reveal the lamp).’ \\
\z


\ea
\gll  Mauwa  on-ap  mera  isak-e. \\
what  do-\textsc{ss.seq}  fish  spear-\textsc{imp}.2s \\
\glt ‘Do that and spear the fish.’ \\
\z


\ea
\gll  Isak-ep  weeser-eya  kiiriw  wakesim-i-nan. \\
spear-\textsc{ss.seq}  finish-2/3s.\textsc{ds}  again  cover-Np-\textsc{fu}.2s \\
\glt ‘When you have finished spearing them you will cover it again.’ \\
\z


\ea
\gll  Nain  arua  karu-i-mik. \\
that1  lamp  run-Np-\textsc{pr}.1/3p \\
\glt ‘That is how they fish with a lamp.’ \\
\z


\ea
\gll  Aria  oko,  kaul  wafur-owa. \\
alright  other  hook  throw-\textsc{nmz} \\
\glt ‘Alright another way is throwing the hook.’ \\
\z


\ea
\gll  Kumin  wiim-ep,  uuriw  or-op  kaul                   wafur-i-mik,  aasa  suuw-ap,  o  papako  uura. \\
hermit.crab  gather-\textsc{ss.seq}  morning  descend-\textsc{ss.seq}  hook   throw-Np-\textsc{pr}.1/3p  canoe  push-\textsc{ss.seq}  or  some  night \\


\glt ‘They gather hermit crabs and go down in the morning and throw the hook, having pushed the canoe out – or some (do it) at night.’ \\
\z


\ea
\gll  Kaul  wafur-owa  maa  eneka,  ona  mera  maa  eneka,   kumin,  wutkekela,  ne  mera  gelemuti-tik,        nain  kaul  wafur-i-mik. \\
hook  throw-\textsc{nmz}  food  tooth  3s.\textsc{gen}  fish  food  tooth      hermit.crab,  calamari  \textsc{add}  fish  small-\textsc{rdp}       that  hook  throw-Np-\textsc{pr}.1/3p \\




\glt ‘Fish-throwing bait, fish-bait, hermit crabs, calamari and small fish – with those they throw the hook.’ \\
\z


\ea
\gll  Ne  emeria  wiena  pona=pa  iimar-ep       kaul  wafur-i-mik,  ifer  pona=pa. \\
\textsc{add}  woman  3p.\textsc{gen}  shore-\textsc{loc}  stand.up-\textsc{ss.seq}   hook  throw-Np-\textsc{pr}.1/3p  sea  shore-\textsc{loc} \\


\glt ‘And the women themselves stand on the shore and throw the hook, on the seashore.’ \\
\z


\ea
\gll  O  mua=ke  aasa  suuw-ap  kaul  wafur-i-mik. \\
3s.\textsc{unm}  man-\textsc{cf}  canoe  push-\textsc{ss.seq}  hook  throw-Np-\textsc{pr}.1/3p \\
\glt ‘The men push the canoe and throw the hook (out in the sea).’ \\
\z


\ea
\gll  Ne  kaul  wafur-owa  mera  aaw-owa  eliw,  maa  marew. \\
\textsc{add}  hook  throw-\textsc{nmz}  fish  get-\textsc{nmz}  well  thing  none \\
\glt ‘And the hook-throwing is a good way to get fish, there is nothing to it.’ \\
\z


\ea
\gll  O  galasim-owa  lawisiw  yoowa. \\
\textsc{intj}  dive.with.goggles-\textsc{nmz}  a.little  hot/hard \\
\glt ‘Diving with goggles is a bit hard.’ \\
\z


\ea
\gll  Kemawisa  puk-i-nan,  mua  bug  maaya  nain=ke  eliw  mera  unowa  isak-i-non,  mua  bug  iiwa  nain  weetak. \\
breath  break-Np-\textsc{fu}.2s  man  wind  long  that1-\textsc{cf}  well     fish  many  spear-Np-\textsc{fu}.3s  man  wind  short  that1  no \\


\glt ‘You will be out of breath, a man with big lungs will spear many fish, a man with small lungs won’t.’ \\
\z


\ea
\gll  Soo  eliw. \\
fishtrap  well \\
\glt ‘Fishtrap is good.’ \\
\z

Mera  aawowa  sira  e  era  wapen  inawiya okaiwi-pa  kuisow. 
Kauliw  aawimik,  arua  karuimik,  maer  puukimik,  patopatiw  mera  urumimik,  oko  galasimimik.

Oko  soo. 
Soo  nain  feenap:  era  erup  ikua. 
Oko  sisina-pa  ifemakimik,  oko  malol-pa   ifemakimik. 
Ne  sisina  nain  me  yoowa  akena. 
Oo  malol  lawisiw  yoowa. 

Sisina  nain,  soo  ika  oninan, soo  ika-pa  kaikap  otal  opora-pa ifemakinan. 
Ifemakep  nomona  iinan-pa  wuainan,  ikoka  ifera  me  pikiwowa  nain. 
Soo  nain  ona  malin  saana-pa  ifemakimik. 
O  ifera  kuowa  epa-pa  weetak.

Aria  malol-pa  ifemakimik  nain  aana puukimik,  makera  unowa  puukap makera  nain  anetirimik. 
Anetir-ikiwep  urufimik,  makera  nain  maaya  pepek. 
Aria  makera  miirifa  okaiwi  soo-pa  kaikimik, okaiwi  pia  kaikimik,  piakina  naimik  nain. 
Ne  soo  ifemakowa  epa-pa  aasa  suuwimik. 
Aasa  suuwap  soo  aasa  iinan-pa  wuap poraimik. 
Orop  malol-pa  soo  nain  fuurkimik. 
Fuurkap  makera  nain  opap  ikaikuan. 
Opap  lawiliw  piipuam-ikaikuan. 
Orop  saa-pa  pokaya  wi  piipuaikuan. 
Ne  soo  nainiw  mufowa  pun  naap,  aana-pa  neeke  mufimik. 
Aana-pa  neeke  mufep,  purupep  aasa-pa  wuaimik. 
Aasa-pa  wuap,  mera  aawep  weesereya  nainiw  fuurkimik. 
Nain  soo  era-ke. 

Aria  maer  pun  naap. 
Wi  emeria  kaalalimik,  kaalalep   orop  otal-pa  maer  arimik. 
Maer  arep  urupep  urupep,  pona. 
Pona-pa  neeke  mera  nomona  ona  iwomakeya, wi  wapeniw  ona  nain  suuwimik. 
Ne  mua  patopat  aawep  saa-pa  iimarep ikaikuan. 
Mera  sisina  urupeya  patopatiw  mikikuan.

Ne  oko  galasimowa. 
Nain  iiriw  me  kererek,  aakisa  fan. 
Wiena  galasimowa  amia  onimik, onap  galasimimik. 
Ne  oko,  mera  urumimik,  patopatiw. 
Nain  ona  mua  taraka  nainke  mera  mikinan, patopat  opowa  nain. 
O  mua  naap  nain  ikoka  uferinan. 

Ne  oko,  afukarimik. 
Parina  isimap  afukarimik. 
Parina  isimap  afukarep,  nain ona  kak  saana-pa. 
Kak  saana-pa  wi  parina  isimap orop  kak  isakimik. 
Kak,  pirit,  mera  papako. 
Tokol  gelemutitik,  kookari,  nain  isakimik. 
Ne  kak  saana-pa  kak  oraimik,  ne  parina  ona  wakesimowa  onaiya  ikaiya. 
Ikoka  wakesimeya  mera  mamaiya  ekapinon,  aasa  mamaiya. 
Aasa  mamaiya  ekapeya  aria  parina  kiiriw    mauwa  one. 
Mauwa  onap  mera  isake. 
Isakep  weesereya  kiiriw  wakesiminan. 
Nain  arua  karuimik. 

Aria  oko,  kaul  wafurowa. 
Kumin  wiimep,  uuriw  orop  kaul wafurimik,  aasa  suuwap,  o  papako  uura. 
Kaul  wafurowa  maa  eneka,  ona  mera  maa  eneka,   kumin,  wutkekela,  ne  mera  gelemutitik, nain  kaul  wafurimik. 
Ne  emeria  wiena  pona-pa  iimarep  kaul  wafurimik,  ifer  pona-pa. 
O  mua-ke  aasa  suuwap  kaul  wafurimik. 
Ne  kaul  wafurowa  mera  aawowa  eliw,  maa  marew. 
O  galasimowa  lawisiw  yoowa. 
Kemawisa  pukinan,  mua  bug  maaya  nain-ke  eliw  mera  unowa  isakinon,  mua  bug  iiwa  nain  weetak. 
Soo  eliw. 


%\setcounter{page}{1} \\

\section{Dog and snake}\label{app:2:dog}
by Saror Aduna
\ea
\gll  Yo  aakisa  opora  gelemuta=ko  ma-i-yem. \\
1s.\textsc{unm}  now  talk  little-\textsc{nf}  say-Np-\textsc{pr}.1s \\
\glt ‘Now I tell a little story.’ \\
\z


\ea
\gll  Emer  en-ow(a)  mua=ko  emeria  fan  aaw-o-k. \\
sago  eat-\textsc{nmz}  man-\textsc{nf}  woman  here  take-\textsc{pa}-3s \\
\glt ‘A Sepik man married a woman from here.’ \\
\z


\ea
\gll  Ne  manina  ikiw-o-k. \\
\textsc{add}  garden  go-\textsc{pa}-3s \\
\glt ‘And he went to the garden.’ \\
\z


\ea
\gll  Ona  siowa  ikos  manina  ikiw-e-mik,  pika  on-owa  na-ep. \\
3s.\textsc{gen}  dog  with  garden  go-\textsc{pa}-1/3p  fence  make-\textsc{nmz}  say-\textsc{ss.seq} \\
\glt ‘He went to the garden with his dog, to make a fence.’ \\
\z


\ea
\gll  Manina=pa  nan  koka  iw-a-mik,  pika  ifara  muf-owa  na-ep. \\
garden-\textsc{loc}  there  jungle  go-\textsc{pa}-1/3p  fence  vine  pull-\textsc{nmz}  say-\textsc{ss.seq} \\
\glt ‘From the garden there they went to the jungle to pull vines for the fence.’ \\
\z


\ea
\gll  Pika  ifara  mufe-wiaw-ikok  ifa  maneka=ke  siowa  wiar  aaw-o-k. \\
Fence  vine  pull-move.around-be.\textsc{ss}  snake  big-\textsc{cf}  dog  3.\textsc{dat}  take-\textsc{pa}.3s \\
\glt ‘When he was moving around (and) pulling the vines, a big snake grabbed his dog.’ \\
\z


\ea
\gll  Siowa  ikos  irak-em-ika-iwkin  siowa  wiawi  nain=ke    siowa  alu-owa  miim-ap  karu-(o)w(a)=iw  ekap-o-k. \\
dog  with  fight-\textsc{ss}.\textsc{sim}-be-2/3s.\textsc{ds}  dog  3s/p.father  that1-\textsc{cf} dog  make.noise-\textsc{nmz}  hear-\textsc{ss.seq}  run-\textsc{nmz}-\textsc{inst}  come-\textsc{pa}-3s \\


\glt ‘When it was fighting with the dog, the dog’s owner heard the dog’s noise and came running.’ \\
\z


\ea
\gll  Karu-(o)w(a)=iw  ekap-ep  uruf-a-k=na  ifa  maneka=ke  siowa  wasi-ep-pu-eya,  aria  nainiw  baurar-ep  ikiw-ep  nomokowa            maaya  war-ep,  ekap-ep,  ifa  nain  ifakim-o-k. \\
run-\textsc{nmz}-\textsc{inst}  come  see-\textsc{pa}-3s-\textsc{tp}  snake  big-\textsc{cf}  dog                wrap-\textsc{ss.seq}-\textsc{cmpl}-2/3s.\textsc{ds}  alright  again  flee-\textsc{ss.seq}  go-\textsc{ss.seq}  tree long  cut-\textsc{ss.seq}  come-\textsc{ss.seq}  snake  that1  kill-\textsc{pa}-3s \\




\glt ‘He came running and saw (to his surprise) a big snake wrapped around the dog; alright he ran away again and cut a long stick, came and killed the snake.’ \\
\z


\ea
\gll  Ifakim-em-ik-eya  ifa  nain=ke  siowa  wasirk-a-k. \\
kill-\textsc{ss}.\textsc{sim}-be-2/3s.\textsc{ds}  snake  that1-\textsc{cf}  dog  release-\textsc{pa}-3s \\
\glt ‘As he was killing it, the snake released the dog.’ \\
\z


\ea
\gll  Siowa  wasirk-ap  ifa  nain  baurar-e-k. \\
dog  release-\textsc{ss.seq}  snake  that1  flee-\textsc{pa}-3s \\
\glt ‘The snake released the dog and fled.’ \\
\z


\ea
\gll  Baurar-ep  iki(w-e)m-ik-eya  siowa  wiawi  ikos  pun  baurar-e-mik. \\
flee-\textsc{ss.seq}  go-\textsc{ss}.\textsc{sim}-be-2/3s.\textsc{ds}  dog  3s/p.father  with  too  flee-\textsc{pa}-1/3p \\
\glt ‘As it was fleeing, the dog with its owner ran away too.’ \\
\z


\ea
\gll  Baurar-ep  owowa  or-o-mik. \\
flee-\textsc{ss.seq}  village  descend-\textsc{pa}-1/3p \\
\glt ‘They ran away and came down to the village.’ \\
\z


\ea
\gll  Opaimika  muut  nan-e-k,  weeser-e-k. \\
talk  only  be.there-\textsc{pa}-3s  finish-\textsc{pa}-3s \\
\glt ‘The talk/story is there, it is finished.’ \\
\z

Yo aakisa opora gelemuta-ko maiyem.

Emer enow mua-ko emeria fan aawok. 
Ne manina ikiwok. 
Ona siowa ikos manina ikiwemik, pika onowa naep. 
Manina-pa nan koka iwamik, pika ifara mufowa naep. 
Pika ifara mufe-wiaw-ikok ifa maneka-ke siowa wiar aawok. 
Siowa ikos irakem-ikaiwkin siowa wiawi nain-ke siowa aluowa miimap karuw-iw ekapok. 
Karuw-iw ekapep urufak-na ifa maneka-ke siowa wasiep-pueya,  aria nainiw baurarep ikiwep nomokowa maaya warep, ekapep, ifa nain ifakimok. 
Ifakimem-ikeya ifa nain-ke siowa wasirkak. 
Siowa wasirkap ifa nain baurarek. 
Baurarep ikim-ikeya siowa wiawi ikos pun bauraremik. 
Baurarep owowa oromik. 
Opaimika muut nanek, weeserek. 


\section{Piglet}\label{app:2:piglet}
by Saror Aduna
\ea
\gll  Tunde=pa  fikera  kuum-iwkin  ikiw-ep  waaya      mik-a-m  nain  ma-i-yem. \\
Tuesday-\textsc{loc}  kunai.grass  burn-2/3p.\textsc{ds}  go-\textsc{ss.seq}  pig  spear-\textsc{pa}-1s  that1  say-Np-\textsc{pr}.1s \\


\glt ‘I tell about that when on Tuesday they burned kunai grass and I went and speared a pig.’ \\
\z


\ea
\gll  Tunde  uuriw  kiikir  akena  iwera  fook-ap  ikiw-e-mik. \\
Tuesday  morning  first  very  coconut  split-\textsc{ss.seq}  go-\textsc{pa}-1/3p \\
\glt ‘On Tuesday morning we split a coconut\footnote{Coconut-splitting ceremony is part of pig hunting.} and (then) went.’ \\
\z


\ea
\gll  Ne  iwera  fook-owa  garanga  nain  wiena  owow  ara=pa      wiar  ik-ua,  ne  fikera  pun  wiena  nain=ke. \\
\textsc{add}  coconut  split-\textsc{nmz}  family  that1  3p.\textsc{gen}  village  part-\textsc{loc}  3.\textsc{dat}  be-\textsc{pa}.3s  \textsc{add}  kunai.grass  also  3p.\textsc{gen}  that1-\textsc{cf} \\


\glt ‘And the coconut splitting was in that family’s section of the village, and the kunai grass (area ) was theirs too.’ \\
\z


\ea
\gll  Iwera  fook-owa  epa=pa  maa  uup-e-mik. \\
coconut  split-\textsc{nmz}  place-\textsc{loc}  food  cook-\textsc{pa}-1/3p \\
\glt ‘They cooked food in the coconut splitting place.’ \\
\z


\ea
\gll  Maa  en-emi  iwera  op-ap  fook-a-mik. \\
food  eat-\textsc{ss}.\textsc{sim}  coconut  hold-\textsc{ss.seq}  split-\textsc{pa}-1/3p \\
\glt ‘We ate and held a coconut and split it.’ \\
\z


\ea
\gll  Nain  weeser-eya  woowa  akua  aaw-ep  ikiw-e-mik. \\
that1  finish-2/3s.\textsc{ds}  spear  shoulder  take-\textsc{ss.seq}  go-\textsc{pa}-1/3p \\
\glt ‘When that was finished, we carried spears on our shoulders and went.’ \\
\z


\ea
\gll  Ikiw-ep  fikera  onaiya=pa  nan  pok-ap,  wi  yapen  mua            unow=iya  pok-ap,  nainiw  neeke  iwera  fook-a-mik,           maa  en-e-mik. \\
go-\textsc{ss.seq}  kunai.grass  with-\textsc{loc}  there  sit-\textsc{ss.seq}  3p.\textsc{unm}  inland  man  many-\textsc{com}  sit-\textsc{ss.seq}  again  there.\textsc{cf}  coconut  split-\textsc{pa}-1/3p  food  eat-\textsc{pa}-1/3p \\




\glt ‘We went and sat down there among the kunai grass, we sat down together with the inland men, and again there we split a coconut and ate food.’ \\
\z


\ea
\gll  Weeser-eya  ama  kuisow  naap  mukuna  enek-a-mik. \\
finish-2/3s.\textsc{ds}  sun  one  thus  fire  light-\textsc{pa}-1/3p \\
\glt ‘When that was finished at about one o’clock the fire was lighted.’ \\
\z


\ea
\gll  Uma  kiikir  mukuna  enek-a-mik. \\
top/ridge  first  fire  light-\textsc{pa}-1/3p \\
\glt ‘The fire was first lighted at the top/ridge.’ \\
\z


\ea
\gll  Neeke  awe-or-om-ik-eya  i  fiker  saakia                or-op  fikera  epia  ook-a-mik. \\
there.\textsc{cf}  burn-descend-be-2/3s.\textsc{ds}  1p.\textsc{unm}  kunai.grass  ashes  descend.\textsc{ss.seq}  kunai.grass  fire  follow-\textsc{pa}-1/3p \\


\glt ‘From there it was burning down and we went down (along) the ashes and followed the grass fire.’ \\
\z


\ea
\gll  Ook-ap  or-op  ema  mamaiya=pa  waaya  gelemuta         urup-o-k. \\
follow-\textsc{ss.seq}  descend-\textsc{ss.seq}  hill  near-\textsc{loc}  pig  small     ascend-\textsc{pa}-3s \\


\glt ‘We followed it down and near the hill a small pig came up.’ \\
\z


\ea
\gll  Me  maneka,  muuka,  kia  gelemuta. \\
not  big,  child,  white  small \\
\glt ‘It was not big, it was a small light-coloured piglet.’ \\
\z


\ea
\gll  Mua  arow  akena  epa  nain  iimar-e-mik,  yos=ke  erepam. \\
man  three  very  place  that1  stand-\textsc{pa}-1/3p  1s.\textsc{fc}-\textsc{cf}  four \\
\glt ‘Exactly three men stood at that place, I was the fourth.’ \\
\z


\ea
\gll  Waaya  gelemuta  nain  urup-em-ik-eya  mua  arow  naap  ufer-a-mik. \\
pig  small  that1  ascend-\textsc{ss}.\textsc{sim}-be-2/3s.\textsc{ds}  man  three  thus  miss-\textsc{pa}-1/3s \\
\glt ‘When the small pig was coming up those three men missed it.’ \\
\z


\ea
\gll  Ufer-iwkin  urup-em-ik-eya  yos=ke  mik-a-m. \\
miss-2/3p.\textsc{ds}  ascend-\textsc{ss}.\textsc{sim}-be-2/3s.\textsc{ds}  1s.\textsc{fc}-\textsc{cf}  spear-\textsc{pa}-1s \\
\glt ‘When they missed it and it was coming up I speared it.’ \\
\z


\ea
\gll  Mik-ap  yena  inasina  unuma  unuf-a-m. \\
spear-\textsc{ss.seq}  1s.\textsc{gen}  spirit  name  call-\textsc{pa}-1s \\
\glt ‘I speared it and called my spirit name.’ \\
\z


\ea
\gll  Ne  sira  naap  ik-ua:  waaya  mik-ap,  inasina  unuma  unuf-eya         mua  unowa  miim-ap  ma-i-kuan,  ``O  waaya  mik-a-k.`` \\
\textsc{add}  custom  thus  be-\textsc{pa}.3s  pig  spear-\textsc{ss.seq}  spirit  name  call-2/3s.\textsc{ds}     man  many  hear-\textsc{ss.seq}  say-Np-\textsc{fu}.3p  3s.\textsc{unm}  pig  spear-\textsc{pa}-3s \\


\glt ‘And the custom is like that: when you spear a pig and call your spirit name, many men will hear it and say,  “He has speared a pig.” ’ \\
\z


\ea
\gll  No  inasina  unuma  me  unuf-i-nan  mua  oko=ke  waaya  nain    mik-ap  nefar  aaw-i-non. \\
2s.\textsc{unm}  spirit  name  not  call-Np-\textsc{fu}.2s  man  other-\textsc{cf}  pig  that1   spear.\textsc{ss.seq}  2s.\textsc{dat}  take-Np-\textsc{fu}.3s \\


\glt ‘If you do not call (your) spirit name, another man will spear that pig and take it from you.’ \\
\z


\ea
\gll  Nain  opora  marew,  no  inasina  unuma  me  unuf-a-n. \\
that  talk  no(ne)  2s.\textsc{unm}  spirit  name  not  call-\textsc{pa}-2s \\
\glt ‘You have no say, (because) you did not call the spirit name.’ \\
\z


\ea
\gll  Aria  mik-amkun  me  um-o-k,  woowa  onaiya  iki(w-e)m-ik-eya       Olas=ke  ekap-emi  woowa  wafur-om-a-k. \\
alright  spear-1s/p.\textsc{ds}  not  die-\textsc{pa}-3s  spear  with  go-\textsc{ss}.\textsc{sim}-be-2/3s.\textsc{ds} Olas-\textsc{cf}  come-\textsc{ss}.\textsc{sim}  spear  throw-\textsc{ben}-\textsc{bnfy}2.\textsc{pa}-3s \\


\glt ‘Alright when I speared it, it didn’t die; when it was going with the spear Olas came and threw a spear for it.’ \\
\z


\ea
\gll  Wafur-a-k  na  weetak,  ufer-a-k. \\
throw-\textsc{pa}-3s  \textsc{tp}  no  miss-\textsc{pa}-3s \\
\glt ‘He threw but no, he missed.’ \\
\z


\ea
\gll  Ufer-ap  nainiw  burir  aaw-ep  woosa=pa  aruf-eya      waaya  nain  in-e-k. \\
miss-\textsc{ss.seq}  again  axe  take-\textsc{ss.seq}  head-\textsc{loc}  hit-2/3s.\textsc{ds}   pig  that1  sleep-\textsc{pa}-3s \\


\glt ‘He missed and again took an axe and when he hit it in the head the pig fell.’ \\
\z


\ea
\gll  In-eya  yena  ikiw-emi  nainiw  woowa  erup  ar-ow-amkun           iiwawun  um-o-k. \\
sleep-2/3s.\textsc{ds}  1s.\textsc{gen}  go-\textsc{ss}.\textsc{sim}  again  spear  two  become-\textsc{appl}-1s/p.\textsc{ds} altogether  die-\textsc{pa}-3s \\


\glt ‘It fell and I myself went and speared it again twice and it died altogether.’ \\
\z


\ea
\gll  Ume-ya  merena  ere-erup  ifara  aaw-ep  kaik-ap  nabena            suuw-ap  akua  aaw-ep  or-o-m. \\
die-2/3s.\textsc{ds}  leg  \textsc{rdp}-two  vine  take-\textsc{ss.seq}  tie-\textsc{ss.seq}  carrying.pole   push-\textsc{ss.seq}  shoulder  take-\textsc{ss.seq}  descend-\textsc{pa}-1s \\


\glt ‘It died, and having taken a vine I tied its legs two and two (together) and pushed it onto a carrying pole and carried it on my shoulder and came down.’ \\
\z


\ea
\gll  Or-om-ik-ok  Pauli  ame  era=pa  wia  uruf-ap  Pauli  ikos             waaya  nain  akua  aaw-e-mik. \\
descend-\textsc{ss}.\textsc{sim}-be-\textsc{ss}  Pauli  \textsc{ass}  road-\textsc{loc}  3p.\textsc{acc}  see-\textsc{ss.seq}  Pauli  with  pig  that1  shoulder  take-\textsc{pa}-1/3p \\


\glt ‘Coming down I saw Pauli and others on the road and (then) I carried it with Pauli on our shoulders.’ \\
\z


\ea
\gll  Akua  aaw-ep  owowa  or-o-mik. \\
shoulder  take-\textsc{ss.seq}  village  descend-\textsc{pa}-1/3p \\
\glt ‘We carried it on our shoulders and came down to the village.’ \\
\z


\ea
\gll  Urera  waaya  gelemuta  nain  pa-ep,  kio-kiowa  naap  aaw-ep         uup-ep  en-e-mik. \\
afternoon  pig  small  that1  butcher-\textsc{ss.seq}  \textsc{rdp}-piece  thus  take-\textsc{ss.seq}  cook-\textsc{ss.seq}  eat-\textsc{pa}-1/3p \\


\glt ‘In the afternoon we butchered the small pig, took the pieces like that, cooked and ate.’ \\
\z


\ea
\gll  Waaya  maneka  marew  pun,  mua  unowa  me  wia  pepek-er-a-k. \\
pig  big  no(ne)  also  man  many  not  3p.\textsc{acc}  enough-come-\textsc{pa}-3s \\
\glt ‘It wasn’t a big pig either, it wasn’t enough for many people.’ \\
\z


\ea
\gll  Yiena  iisow  yiena  garanga  muutiw  aaw-ep  uup-ep         en-e-mik. \\
1p.\textsc{gen}  1p.\textsc{isol}  1p.\textsc{gen}  family  only  take-\textsc{ss.seq}  cook-\textsc{ss.seq}    eat-\textsc{pa}-1/3p \\


\glt ‘Just our family by ourselves took and cooked and ate it.’ \\
\z


\ea
\gll  Ne  waaya  nain  pun  afila  marew,  waaya  asia  pun. \\
\textsc{add}  pig  that1  also  sweet/fatty  no(ne)  pig  wild  also. \\
\glt ‘And the pig wasn’t fatty/sweet either, it was a wild pig too.’ \\
\z


\ea
\gll  Opaimika  muut  naap,  weeser-e-k. \\
talk  only  thus,  finish-\textsc{pa}-3s \\
\glt ‘The talk is only like that, it is finished.’ \\
\z

Tunde-pa  fikera  kuumiwkin  ikiwep  waaya mikam  nain  maiyem.

Tunde  uuriw  kiikir  akena  iwera  fookap  ikiwemik. 
Ne  iwera  fookowa  garanga  nain  wiena  owow  ara-pa wiar  ikua,  ne  fikera  pun  wiena  nain-ke. 
Iwera  fookowa  epa-pa  maa  uupemik. 
Maa  enemi  iwera  opap  fookamik. 
Nain  weesereya  woowa  akua  aawep  ikiwemik. 
Ikiwep  fikera  onaiya-pa  nan  pokap,  wi  yapen  mua            unowiya  pokap,  nainiw  neeke  iwera  fookamik, maa  enemik. 
Weesereya  ama  kuisow  naap  mukuna  enekamik. 
Uma  kiikir  mukuna  enekamik. 
Neeke  awe-orom-ikeya  i  fiker  saakia orop  fikera  epia  ookamik. 
Ookap  orop  ema  mamaiya-pa  waaya  gelemuta  urupok. 
Me  maneka,  muuka,  kia  gelemuta. 
Mua  arow  akena  epa  nain  iimaremik,  yos-ke  erepam.
Waaya  gelemuta  nain  urupem-ikeya  mua  arow  naap  uferamik. 
Uferiwkin  urupem-ikeya  yos-ke  mikam. 
Mikap  yena  inasina  unuma  unufam. 

Ne  sira  naap  ikua:  waaya  mikap,  inasina  unuma  unufeya mua  unowa  miimap  maikuan,  ``O  waaya  mikak.” 
No  inasina  unuma  me  unufinan,  mua  okoke  waaya  nain    mikap  nefar  aawinon. 
Nain  opora  marew,  no  inasina  unuma  me  unufan.

Aria  mikamkun  me  umok,  woowa  onaiya  iki(we)m-ikeya       Olas-ke  ekapemi  woowa  wafuromak. 
Wafurak  na  weetak,  uferak. 
Uferap  nainiw  burir  aawep  woosa-pa  arufeya waaya  nain  inek. 
Ineya  yena  ikiwemi  nainiw  woowa  erup  arowamkun           iiwawun  umok. 
Umeya  merena  ere-erup  ifara  aawep  kaikap  nabena            suuwap  akua  aawep  orom. 
Orom-ikok  Pauli  ame  era-pa  wia  urufap  Pauli  ikos             waaya  nain  akua  aawemik. 
Akua  aawep  owowa  oromik.

Urera  waaya  gelemuta  nain  paep,  kio-kiowa  naap  aawep   uupep  enemik. 
Waaya  maneka  marew  pun,  mua  unowa  me  wia  pepekerak. 
Yiena  iisow  yiena  garanga  muutiw  aawep  uupep         enemik. 
Ne  waaya  nain  pun  afila  marew,  waaya  asia  pun. 

Opaimika  muut  naap,  weeserek. 

%\setcounter{page}{1} \\

\section{Man’s lover}\label{app:2:lover}
by Kinangir Saror
\ea
\gll  Mua  nain  ona  emeria  onaria  ik-ua,  ne  mua  nain  urema  osarena  ikiw-o-k. \\
man  that1  3s.\textsc{gen}  woman  with  be-\textsc{pa}.3s  \textsc{add}  man  that1  bandicoot  path  go-\textsc{pa}-3s \\


\glt ‘There was that man with his wife, and the man went to hunt bandicoots.’ \\
\z


\ea
\gll  Ikiw-ep  ik-eya  ona  soma  emeria  nain  kukusa  nain=ke        ekap-ep  ona  emeria  nain  maa  wiar  wafu-fur-eya                 naap  maak-e-k,  “Nena  mua=na  urema  osarena  ikiw-o-k”. \\
go-\textsc{ss.seq}  be-2/3s.\textsc{ds}  3s.\textsc{gen}  lover  woman  that1  spirit  that1-\textsc{cf}   come-\textsc{ss.seq}  3s.\textsc{gen}  woman  that1  thing  3.\textsc{dat}  throw-\textsc{rdp}-2/3s.\textsc{ds}  thus  say-\textsc{pa}-3s  2s.\textsc{gen}  man-\textsc{tp}  bandicoot  path  go-\textsc{pa}-3s \\




\glt ‘When he was gone and his lover’s spirit came and threw around his wife’s things she told her, “Your man went to hunt bandicoots.” ’ \\
\z


\ea
\gll  Naap  maak-eya  aria  ona  womar  emeria  nain  kukusa  nain  miim-a-k. \\
thus  tell-2/3s.\textsc{ds}  alright  3s.\textsc{gen}  friend  woman  that1  spirit  that1  heard-\textsc{pa}-3s \\
\glt ‘When she told her like that alright the spirit of his lady friend heard that.’ \\
\z


\ea
\gll  Miim-ap  ikiw-o-k,  ikiw-ep  mua  nain  urema  osarena=pa    iimar-ep  ik-eya  ona  mua  nain  ifakim-o-k. \\
hear-\textsc{ss.seq}  go-\textsc{pa}-3s  go-\textsc{ss.seq}  man  that1  bandicoot  path-\textsc{loc}  stand-\textsc{ss.seq}  be-2/3s.\textsc{ds}  3s.\textsc{gen}  man  that1  kill-\textsc{pa}-3s \\


\glt ‘She heard it and went; she went and, as the man was standing on the bandicoot path, killed his man.’ \\
\z


\ea
\gll  Ifakim-eya  pu-ep  ik-eya  om-em-ik-ua. \\
kill-2/3s.\textsc{ds}  die-\textsc{ss.seq}  be-2/3s.\textsc{ds}  cry-\textsc{ss}.\textsc{sim}-be-\textsc{pa}.3s \\
\glt ‘She killed him and as he was dead she was crying.’ \\
\z


\ea
\gll  Sawur  emeria  nain=ke  ona  soma  mua  nain  ifakim-o-k. \\
spirit  woman  that1-\textsc{cf}  3s.\textsc{gen}  lover  man  that1  kill-\textsc{pa}-3s \\
\glt ‘The spirit woman killed her lover.’ \\
\z


\ea
\gll  Om-em-ik-eya  sawur  emeria  ona  wiawi  onak=ke                     ekap-emi  naap  maak-e-mik,  “No  moram  naap  om-em-ika-i-n? \\
cry-\textsc{ss}.\textsc{sim}-be-2/3s.\textsc{ds}  spirit  woman  3s.\textsc{gen}  3s/p.father  3s/p.mother-\textsc{cf}  come-\textsc{ss}.\textsc{sim}  thus  tell-\textsc{pa}-1/3p  2s.\textsc{unm}  why  thus  cry-\textsc{ss}.\textsc{sim}-be-Np-\textsc{pr}.2s \\


\glt ‘As she was crying the spirit woman’s father and mother came and told her, “Why are you crying like that? ’ \\
\z


\ea
\gll  A=na  naap  ma-emi  omom-e  na,  ‘Mua  yii,  mua  yee.’ \\
\textsc{intj}-\textsc{tp}  thus  say-\textsc{ss}.\textsc{sim}  cry-\textsc{imp}.2s  \textsc{intj}  man  \textsc{intj}  man  \textsc{intj} \\
\glt ‘Cry like that, ‘Man yii, man yee.’ ’ \\
\z


\ea
\gll  Naap  ma-emi  om-em-ika-i-nan  na.” \\
thus  say-\textsc{ss}.\textsc{sim}  cry-\textsc{ss}.\textsc{sim}-be-Np-\textsc{fu}.2s  \textsc{intj} \\
\glt ‘Keep crying and saying like that.” ’ \\
\z


\ea
\gll  Naap  maak-e-mik. \\
thus  tell-\textsc{pa}-1/3p \\
\glt ‘They told her like that.’ \\
\z


\ea
\gll  Ne  mauwa  nain  aaw-ep  iima=pa  wu-om-ap                 om-em-ik-ua,  sawur  emeria  nain=ke. \\
\textsc{add}  what  that1  take-\textsc{ss.seq}  chest-\textsc{loc}  put-\textsc{ben}-\textsc{bnfy}2-\textsc{ss.seq}   cry-\textsc{ss}.\textsc{sim}-be-\textsc{pa}.3s  spirit  woman  that-\textsc{cf} \\


\glt ‘And she took this thing whatever and put it on his chest and cried, the spirit woman (did).’ \\
\z


\ea
\gll  Om-em-ik-eya  om-em-ik-eya  epa  wiim-o-k. \\
cry-\textsc{ss}.\textsc{sim}-be-2/3s.\textsc{ds}  cry-\textsc{ss}.\textsc{sim}-be-2/3s.\textsc{ds}  place  dawn-\textsc{pa}-3s \\
\glt ‘As she was crying and crying it dawned.’ \\
\z


\ea
\gll  Epa  wiim-eya  sawur  emeria  nain  ikiw-eya,  o  iikir-ami             owowa  ekap-o-k. \\
place  dawn-2/3s.\textsc{ds}  spirit  woman  that1  go-2/3s.\textsc{ds}  3s.\textsc{unm}  get.up-\textsc{ss}.\textsc{sim}  village  come-\textsc{pa}-3s \\


\glt ‘When it dawned and the spirit woman went, he got up and came to the village.’ \\
\z


\ea
\gll  Ekap-emi  ona  emeria  maak-e-k,  “Yo  soma  emeria  nain=ke  efa          ifakim-eya  pu-ep  ik-omkun  efa  om-em-ik-eya                       epa  wiim-o-k. \\
come-\textsc{ss}.\textsc{sim}  3s.\textsc{gen}  woman  tell-\textsc{pa}-3s  1s.\textsc{unm}  lover  woman  that1-\textsc{cf}  1s.\textsc{acc}   kill-2/3s.\textsc{ds}  die-\textsc{ss.seq}  be-1s/p.\textsc{ds}  1s.\textsc{acc}  cry-\textsc{ss}.\textsc{sim}-be-2/3s.\textsc{ds}  place  dawn-\textsc{pa}-3s \\




\glt ‘He came and told his wife, “My lover came and killed me, and when I was dead and she was crying over me it dawned.’ \\
\z


\ea
\gll  Ne  yo  aakisa  fan  epa  wiim-eya  uuriw  ekap-i-yem.” \\
\textsc{add}  1s.\textsc{unm}  now  here  place  dawn-2/3s.\textsc{ds}  morning  come-Np-\textsc{pr}.1s \\
\glt ‘And just now that it has dawned in the morning I come.” ’ \\
\z


\ea
\gll  Opora  muut  naap. \\
talk  only  thus \\
\glt ‘The story is like that.’ \\
\z

Mua nain ona emeria onaria ikua, ne mua nain urema osarena ikiwok. 
\textrm{Ikiwep ikeya ona soma emeria nain kukusa nain-ke ekapep ona emeria nain maa wiar wafufureya naap maakek, “Nena mua=na urema osarena ikiwok.”  }
Naap maakeya aria ona womar emeria nain kukusa nain miimak. 
Miimap ikiwok, ikiwep mua nain urema osarena-pa iimarep ikeya ona mua nain ifakimok. 
Ifakimeya puep ikeya omem-ikua. 

Sawur emeria nain-ke ona soma mua nain ifakimok. 
\textrm{Omem-ikeya sawur emeria ona wiawi onak-ke ekapemi naap maakemik, “No moram naap omem-ikain?  }
\textrm{A=na naap maemi omome-na, ‘Mua yii, mua yee.’ }
\textrm{Naap maemi omem-ikainan na.”  }
Naap maakemik.

Ne mauwa nain aawep iima-pa wuomap omem-ikua, sawur emeria nain-ke. 
Omem-ikeya omem-ikeya epa wiimok. 
Epa wiimeya sawur emeria nain ikiweya, o iikirami owowa ekapok. 
\textrm{Ekapemi ona emeria maakek, “Yo soma emeria nain-ke efa ifakimeya puep ikomkun efa omem-ikeya epa wiimok.  }
\textrm{Ne yo aakisa fan epa wiimeya uuriw ekapiyem.”  }
Opora muut naap.

 
\section{A flood story}\label{app:2:flood}
by Yaura
\ea
\gll  Yiena  emeria  mia  damol-al-i-mik  nain  mia  aka  nain      aaw-ep  p-ikiw-ep  manina=pa  upuna=pa  wu-a-k,               mia  aka  nain  aaw-ep. \\
1p.\textsc{gen}  woman  body  bad-\textsc{appl}-Np-\textsc{pr}.1/3p  that1  body  blood  that1 take-\textsc{ss.seq}  Bpx-go-\textsc{ss.seq}  garden-\textsc{loc}  furrow-\textsc{loc}  put-\textsc{pa}-3s body  blood  that1  take-\textsc{ss.seq} \\




\glt ‘We women have menstruation (and a woman) took that menstrual blood and took it to the garden and put it in a furrow, having taken that menstrual blood.’ \\
\z


\ea
\gll  Upuna=pa  wu-eya  muuka  wiip=iya  kerer-e-mik. \\
furrow-\textsc{loc}  put-2/3s.\textsc{ds}  son  daughter-\textsc{com}  appear-\textsc{pa}-1/3p \\
\glt ‘When she put it in the furrow both a son and a daughter appeared.’ \\
\z


\ea
\gll  Kerer-ep  onak  maak-e-mik,  “Aite,  i  aaya=ko yia  aaw-om-aya  enim-i-yan.” \\
appear-\textsc{ss.seq}  3s/p.mother  tell-\textsc{pa}-1/3p  1s/p.mother  1p.\textsc{unm}  sugarcane-\textsc{nf} 1p.\textsc{acc}  get-\textsc{ben}-\textsc{bnfy}2.2/3s.\textsc{ds}  eat-Np-\textsc{fu}.1p \\
\glt ‘They appeared and told their mother, “Mother, get sugarcane for us and we will eat it.” ’ \\
\z


\ea
\gll  Ne  onak=ke  aaya  wia  aaw-om-aya                       enim-or-om-ik-e-mik. \\
\textsc{add}  3s/p.mother-\textsc{cf}  sugarcane  3p.\textsc{acc}  get-\textsc{ben}-\textsc{bnfy}2.2/3s.\textsc{ds}  eat-descend-\textsc{ss}.\textsc{sim}-be-\textsc{pa}-1/3p \\


\glt ‘And their mother got sugarcane for them and they went down eating it.’ \\
\z


\ea
\gll  Onak  owawiya  owowa  or-o-mik. \\
3s/p.mother  with  village  descend-\textsc{pa}-1/3p \\
\glt ‘They went down to the village with their mother.’ \\
\z


\ea
\gll  Owowa  or-op  owowa=pa  onak  maa  uup-o-k. \\
village  descend-\textsc{ss.seq}  village-\textsc{loc}  3s/p.mother  food  cook-\textsc{pa}-3s \\
\glt ‘They went down to the village and in the village their mother cooked food.’ \\
\z


\ea
\gll  Maa  uup-ep  fofola  urup-eya  maa  op-iya  iiw-o-k. \\
food  cook-\textsc{ss.seq}  foam  rise-2/3s.\textsc{ds}  food  be.done-2/3s.\textsc{ds}  dish.out-\textsc{pa}-3s \\
\glt ‘She cooked the food and it boiled and was done and (then) she dished it out.’ \\
\z


\ea
\gll  Iiw-ep  wiipa  muuka  nain  wia  maak-e-k,  “Ni              auwa  maa  p-ikiw-om-aka.” \\
dish.out-\textsc{ss.seq}  daughter  son  that1  3p.\textsc{acc}  tell-\textsc{pa}-3s  2p.\textsc{unm}  1s/p.father  food  Bpx-go-\textsc{ben}-\textsc{bnfy}2.\textsc{imp}.2p \\


\glt ‘She dished it out and said to the son and daughter, “Take food to father.” ’ \\
\z


\ea
\gll  Maa  p-ikiw-om-iwkin  wiawi=ke  wia  mokak-urup-o-k              wia  mokak-or-o-k,  “I  muuka  marew  a,  wiipa                 marew  a”,  naap  wia  maak-e-k. \\
food  Bpx-go-\textsc{ben}-2/3p.\textsc{ds}  3s/p.father-\textsc{cf}  3p.\textsc{acc}  stare-ascend-\textsc{pa}-3s   3p.\textsc{acc}  stare-descend-\textsc{pa}-3s  1p.\textsc{unm}  son  none  \textsc{intj}  daughter  none  \textsc{intj}  thus  3p.\textsc{acc}  tell-\textsc{pa}-3s \\




\glt ‘When they took food to him, their father stared them up and down, “We have no son, no daughter,” he told them.’ \\
\z


\ea
\gll  Ne  wiawi=ke  maa  nain  aaw-emi  mia  iinan=pa  wiar  sawik-a-k. \\
\textsc{add}  3s/p.father-\textsc{cf}  food  that1  take-\textsc{ss}.\textsc{sim}  body  on.top-\textsc{loc}  3.\textsc{dat}  pour-\textsc{pa}-3s \\
\glt ‘And taking the food their father poured it on them.’ \\
\z


\ea
\gll  Sawik-eya  ep-ap  onak  maak-e-mik,  “I  auwa=ke                         maa  mia  iinan=pa  yiar  sawik-a-k.” \\
pour-2/3s.\textsc{ds}  come-\textsc{ss.seq}  3s/p.mother  tell-\textsc{pa}-1/3p  1p.\textsc{unm}  1s/p.father-\textsc{cf}   food  body  on.top-\textsc{loc}  1p.\textsc{dat}  pour-\textsc{pa}-3s \\


\glt ‘When he poured (food on them) they came and told their mother, “Father poured food on top of us.” ’ \\
\z


\ea
\gll  Om-emi  maak-e-mik. \\
cry-\textsc{ss}.\textsc{sim}  tell-\textsc{pa}-1/3p \\
\glt ‘Crying they told her (that).’ \\
\z


\ea
\gll  Ne  naap  ik-ok  in-e-mik. \\
\textsc{add}  thus  be-\textsc{ss}  sleep-\textsc{pa}-1/3p \\
\glt ‘And then they slept.’ \\
\z


\ea
\gll  In-ep,  epa  wiim-eya  onak  maak-e-mik,  “Aite,                     i  kemuka=ko  yia  kemi-om-a.” \\
sleep-\textsc{ss.seq}  place  dawn-2/3s.\textsc{ds}  3s/p.mother  tell-\textsc{pa}-1/3p  1s/p.mother  1p.\textsc{unm}  string-\textsc{nf}  1p.\textsc{acc}  roll-\textsc{ben}-\textsc{bnfy}2.\textsc{imp}.2s \\


\glt ‘They slept, and in the morning when it dawned they told their mother, “Mother, roll string for us.” ’ \\
\z


\ea
\gll  Na-iwkin  onak  kemuka  wia  kemi-om-a-k. \\
say-2/3p.\textsc{ds}  3s/p.mother  string  3p.\textsc{acc}  roll-\textsc{ben}-\textsc{bnfy}2.\textsc{pa}-3s \\
\glt ‘They said (like that) and their mother rolled string for them.’ \\
\z


\ea
\gll  Wia  kemi-e-k  kemi-e-k  kemi-e-k,  kemuka  nain  maay-ar-e-k. \\
3p.\textsc{acc}  roll-\textsc{pa}.3s  roll-\textsc{pa}.3s  roll-\textsc{pa}.3s  string  that1  long-\textsc{appl}-\textsc{pa}-3s \\
\glt ‘She rolled and rolled and rolled it (for) them and the string became long.’ \\
\z


\ea
\gll  Maay-ar-eya  wia  maak-e-k. \\
long-\textsc{appl}-2/3s.\textsc{ds}  3p.\textsc{acc}  tell-\textsc{pa}-3s \\
\glt ‘It became long and she told it to them.’ \\
\z


\ea
\gll  “Ae,  aite,  i  uurika  ora-i-yan,  ifera                  un-owa  ora-i-yan.” \\
Yes  1s/p.mother  1p.\textsc{unm}  tomorrow  descend-Np-\textsc{fu}.1p  saltwater  fetch-\textsc{nmz}  descend-Np-\textsc{fu}.1p \\


\glt ‘Yes, mother, tomorrow we’ll go down, we’ll go down to fetch saltwater.’ \\
\z


\ea
\gll  Ne  onak=ke,  “A,  ifera  feeke  un-eka.” \\
\textsc{add}  3s/p.mother  \textsc{intj}  saltwater  here.\textsc{cf}  fetch-\textsc{imp}.2p \\
\glt ‘And their mother (said), “Oh, fetch the water (from) right here.” ’ \\
\z


\ea
\gll  Ne  wi  maak-e-mik,  “Wia,  i  oro-or-op            un-i-yan.” \\
\textsc{add}  3p.\textsc{unm}  tell-\textsc{pa}-1/3p  no  1p.\textsc{unm}  \textsc{rdp}-descend-\textsc{ss.seq} fetch-Np-\textsc{fu}.1p \\


\glt ‘But they told her, “No, we’ll go right down (=out to the sea) and fetch it.” ’ \\
\z


\ea
\gll  “A  neeke-r=iw  un-eka.” \\
\textsc{intj}  there.\textsc{cf}-Ø-\textsc{inst}  fetch-\textsc{imp}.2p \\
\glt ‘Oh, fetch it (from) right there.’ \\
\z


\ea
\gll  “Weetak,  i  oro-ora-i-yan.” \\
no  1p.\textsc{unm}  \textsc{rdp}-descend-Np-\textsc{fu}.1p \\
\glt ‘No, we’ll go right down.’ \\
\z


\ea
\gll  Ne  oro-oro-oro-or-omi  oro-oro-or-o-k,  onoma. \\
\textsc{add}  \textsc{rdp}-\textsc{rdp}-\textsc{rdp}-descend-\textsc{ss}.\textsc{sim}  \textsc{rdp}-\textsc{rdp}-descend-\textsc{pa}-3s  horizon \\
\glt ‘And going down and down and down it went down and down to the horizon.’ \\
\z


\ea
\gll  Kemuka  feekiya  op-ap  or-om-ik-e-mik. \\
string  with  hold-\textsc{ss.seq}  descend-\textsc{ss}.\textsc{sim}-be-\textsc{pa}.1/3 \\
\glt ‘They held onto the string and went down.’ \\
\z


\ea
\gll  Ne  kemuka  nain  pepek-er-eya  onak  ona  wiar  puuk-a-k. \\
\textsc{add}  string  that1  enough-go-2/3s.\textsc{ds}  3s/p.mother  3s.\textsc{gen}  3.\textsc{dat}  cut-\textsc{pa}-3s \\
\glt ‘And when the string was long enough their mother herself cut it.’ \\
\z


\ea
\gll  Puuk-eya  wi  erup  nain  onoma  or-o-mik. \\
cut-2/3s.\textsc{ds}  3p.\textsc{unm}  two  that1  horizon  descend-\textsc{pa}-1/3p \\
\glt ‘She cut it and the two of them went down to the horizon.’ \\
\z


\ea
\gll  Or-op  neeke  ika-iwkin  kokom-ar-e-k. \\
descend-\textsc{ss.seq}  there.\textsc{cf}  be-2/3p.\textsc{ds}  dark-\textsc{appl}-\textsc{pa}-3s \\
\glt ‘They went down and when they were there it became dark.’ \\
\z


\ea
\gll  Kokom-ar-eya  in-e-mik. \\
dark-\textsc{appl}-2/3s.\textsc{ds}  sleep-\textsc{pa}-1/3p \\
\glt ‘It became dark and they slept.’ \\
\z


\ea
\gll  Aria  epa  wiim-eya  ama  urup-o-k. \\
alright  place  dawn-2/3.\textsc{ds}  sun  rise-\textsc{pa}-3s \\
\glt ‘Alright when it dawned the sun rose.’ \\
\z


\ea
\gll  Ama  urup-emi  wiawi  kuum-o-k. \\
sun  rise-\textsc{ss}.\textsc{sim}  3s/p.father  burn-\textsc{pa}-3s \\
\glt ‘The sun rose and burned their father.’ \\
\z


\ea
\gll  Wiawi  kuum-eya,  kuum-eya,  kuum-eya  aw-ep                  eka  iw-a-k  na  wia,  eka=ke  saan-ar-e-k. \\
3s/p.father  burn-2/3s.\textsc{ds}  burn-2/3s.\textsc{ds}  burn-2/3s.\textsc{ds}  burn-\textsc{ss.seq}  river  go-\textsc{pa}-3s  \textsc{tp}  no  river-\textsc{cf}  dry-\textsc{pa}-3s \\


\glt ‘It burned and burned and burned their father and when he burned he went into a river but no, the river dried.’ \\
\z


\ea
\gll  Eka  oko  iw-a-k  na  wia,  eka  oko=ke  saan-ar-e-k. \\
river  other  go-\textsc{pa}-3s  \textsc{tp}  no  river  other-\textsc{cf}  dry-\textsc{pa}-3s \\
\glt ‘He went into another river but no, the other river dried.’ \\
\z


\ea
\gll  Ne  wiawi  kuum-eya  kuum-eya  um-o-k,  ama=ke  kuum-eya. \\
\textsc{add}  3s/p.father  burn-2/3s.\textsc{ds}  burn-2/3s.\textsc{ds}  die-\textsc{pa}-3  sun-\textsc{cf}  burn-2/3s.\textsc{ds} \\
\glt ‘And when it burned and burned their father he died; when the sun burned him.’ \\
\z

Yiena emeria mia damolalimik nain mia aka nain aawep pikiwep manina-pa upuna-pa wuak, mia aka nain aawep. 
Upuna-pa wueya muuka wiipiya kereremik. 
Kererep onak maakemik, “Aite, i aaya-ko yia aawomaya enimiyan.”

Ne onak-ke aaya wia aawomaya enim-orom-ikemik. 
Onak owawiya owowa oromik. 
Owowa orop owowa-pa onak maa uupok. 
Maa uupep fofola urupeya maa opiya iiwok. 
Iiwep wiipa muuka nain wia maakek, “Ni auwa maa pikiwomaka.” 
Maa pikiwomiwkin wiawi-ke wia mokak-urupok wia mokak-orok,  “I muuka marewa, wiipa marewa”, naap wia maakek. 
Ne wiawi-ke maa nain aawemi mia iinan-pa wiar sawikak. 
Sawikeya epap onak maakemik, “I auwa-ke maa mia iinan-pa yiar sawikak.” 
Omemi maakemik. 

Ne naap ikok inemik. 
Inep, epa wiimeya onak maakemik, “Aite, i kemuka-ko yia kemioma.” 
Naiwkin onak kemuka wia kemiomak. 
Wia kemiek kemiek kemiek, kemuka nain maayarek. 
Maayareya wia maakek.
“Ae, aite, i uurika oraiyan, ifera unowa oraiyan.” 
Ne onak-ke, “A, ifera feeke uneka.” 
Ne wi maakemik, “Wia, i oro-orop uniyan.” 
“A neekeriw uneka.” 
“Weetak, i oro-oraiyan.”

Ne oro-oro-oro-oromi oro-oro-orok, onoma. 
Kemuka feekiya opap orom-ikemik. 
Ne kemuka nain pepekereya onak ona wiar puukak. 
Puukeya wi erup nain onoma oromik. 
Orop neeke ikaiwkin kokomarek. 
Kokomareya inemik.

Aria epa wiimeya ama urupok. 
Ama urupemi wiawi kuumok. 
Wiawi kuumeya, kuumeya, kuumeya awep eka iwak na wia, eka-ke saanarek. 
Eka oko iwak na wia, eka oko-ke saanarek. 
Ne wiawi kuumeya kuumeya umok, ama-ke kuumeya. 


\section{Copra work}\label{app:2:copra} 
by Saror Aduna 
\ea
\gll  Yo  aakisa  iwera  mauw-owa  nain  ma-i-yem. \\
1s.\textsc{unm}  now  coconut  work-\textsc{nmz}  that1  say-Np-\textsc{pr}.1s \\ 
\glt ‘Now I talk about coconut/copra work.’ \\
\z


\ea
\gll  Kiikir  akena  iwera  opa  aaw-ep  up-i-mik. \\
first  very  coconut  seedling  get-\textsc{ss.seq}  plant-Np-\textsc{pr}.1/3p \\
\glt ‘First of all we\footnote{The whole text could also be in the third person: ‘\textit{they} get… and plant…’  etc., but because this is familiar activity to the writer, translation in the first person was chosen.} get coconut seedlings and plant them.’ \\
\z


\ea
\gll  Up-ep  mokoma  ikur  naap  ikiw-eya  maken-ar-i-ya. \\
plant-\textsc{ss.seq}  year  five  thus  go-2/3s.\textsc{ds}  fruit-\textsc{appl}-Np-3s \\
\glt ‘We plant them and when about five years have gone they bear fruit.’ \\
\z


\ea
\gll  Iwera  maken-ar-ep  ififa  wua-i-ya,  ne  ififa  ora-eya            fiirim-i-mik. \\
coconut  fruit-\textsc{appl}-\textsc{ss.seq}  dry  put-Np-\textsc{pr}.3s  \textsc{add}  dry  descend-2/3s.\textsc{ds}  gather-Np-\textsc{pr}.1/3p \\


\glt ‘Coconut (trees) bear fruit and develop dry coconuts, and when dry coconuts drop we gather them.’ \\
\z


\ea
\gll  Emeria=ke  fiir-im-ikiw-ep  aria  ikoka  mua=ke  kais-i-mik. \\
woman-\textsc{cf}  father-\textsc{ss}.\textsc{sim}-go-\textsc{ss.seq}  alright  later  man-\textsc{cf}  peel-Np-\textsc{pr}.1/3p \\
\glt ‘The women have go around and gather them, alright later the men husk them.’ \\
\z


\ea
\gll  Mua=ke  kais-ap  neeke  wu-ap  miiw-aasa  nop-ap               miiw-aasa=ke  iwer  ififa  nain  aaw-ep  p-ekap-ep  epia  koora                    mamaiya=pa  wu-eya  fook-i-mik. \\
man-\textsc{cf}  peel-\textsc{ss.seq}  there.\textsc{cf}  put-\textsc{ss.seq}  land-canoe  search-\textsc{ss.seq} land-canoe-\textsc{cf}  coconut  dry  that1  take-\textsc{ss.seq}  Bpx-come-\textsc{ss.seq}  fire  house      close-\textsc{loc}  put-2/3s.\textsc{ds}  split-Np-1/3p \\




\glt ‘Men husk them and leave them there and look for a truck, and when the truck takes the dry coconuts and brings them and puts them close to the to the drying shed we split them.’ \\
\z


\ea
\gll  Fook-ap  p-urup-ep  koora=pa  wua-i-mik. \\
split-\textsc{ss.seq}  Bpx-ascend-\textsc{ss.seq}  house-\textsc{loc}  put-Np-\textsc{pr}.1/3s \\
\glt ‘We split them and take them up and put them in the shed.’ \\
\z


\ea
\gll  Koora=pa  wu-ap  weeser-eya  uurikona=pa  epia  wua-i-mik. \\
house-\textsc{loc}  put-\textsc{ss.seq}  finish-2/3s.\textsc{ds}  next.day-\textsc{loc}  fire(wood)  put-Np-\textsc{pr}.1/3p \\
\glt ‘When we put them in the shed and it is finished, the following day we light the fire.’ \\
\z


\ea
\gll  Epia  wu-ap  ikiw-ep  iwera  kuuf-am-ik-eya  iwera  reen-eya                  iwer  urupa  anum-i-mik. \\
fire  put-\textsc{ss.seq}  go-\textsc{ss.seq}  coconut  look-\textsc{ss}.\textsc{sim}-be-2/3s.\textsc{ds}  coconut  dry-2/3s.\textsc{ds}  coconut  shell  knock-Np-\textsc{pr}.1/3p \\


\glt ‘We light the fire and go and (the watchman) watches the coconuts, and when the coconuts have dried we knock (the copra out of) the coconut shells.’ \\
\z


\ea
\gll  Iwer  urupa  anum-ap  weeser-eya  p-or-op,                  owaruma  p-or-op  mik-i-mik. \\
coconut  shell  knock-\textsc{ss.seq}  finish-2/3s.\textsc{ds}  Bpx-descend-\textsc{ss.seq}  outside  Bpx-descend-\textsc{ss.seq}  hit-Np-\textsc{pr}.1/3p \\


\glt ‘We knock the coconut shells and when that is finished we take it (copra) down, we take it down outside and hit it (into the sacks).’ \\
\z


\ea
\gll  Mik-ap  weeser-eya  owow  maneka  sesek-ap  maamuma  aaw-i-mik. \\
hit-\textsc{ss.seq}  finish-2/3s.\textsc{ds}  village  big  send-\textsc{ss.seq}  money  get-Np-\textsc{pr}.1/3p \\
\glt ‘We hit it and when it is finished we send it to town and get money.’ \\
\z


\ea
\gll  Aakisa  opora  muut  naap,  weeser-e-k. \\
now  talk  only  thus  finish-\textsc{pa}-3s \\
\glt ‘Now the talk is only like that, it is finished.’ \\
\z

Yo aakisa iwera mauwowa nain maiyem.

Kiikir akena iwera opa aawep upimik. 
Upep mokoma ikur naap ikiweya makenariya. 
Iwera makenarep ififa wuaiya, ne ififa oraeya fiirimimik. 
Emeria-ke fiirim-ikiwep aria ikoka mua-ke kaisimik. 
Mua-ke kaisap neeke wuap miiwaasa nopap miiwaasa-ke iwer ififa nain aawep pekapep epia koora mamaiya-pa wueya fookimik. 
Fookap purupep koora-pa wuaimik. 
Koora-pa wuap weesereya uurikona-pa epia wuaimik. 
Epia wuap ikiwep iwera kuufam-ikeya iwera reeneya iwer urupa anumimik. 
Iwer urupa anumap weesereya porop, owaruma porop mikimik. 
Mikap weesereya owow maneka sesekap maamuma aawimik. 
\textrm{Aakisa opora muut naap, weeserek.}


\section{Garden work}\label{app:2:garden}
by Saror Aduna
\ea
\gll  Yo  aakisa  manina  uuw-owa  opora  ma-i-nen  na-ep. \\
1s.\textsc{unm}  now  garden  work-\textsc{nmz}  talk  say-Np-\textsc{fu}.1s  say-\textsc{ss.seq} \\
\glt ‘Now I want/intend to talk about garden work.’ \\
\z


\ea
\gll  Kiikir  akena  emeria=ke  manina  nop-i-mik. \\
first  very  woman-\textsc{cf}  garden  clear.bush-Np-\textsc{pr}.1/3p \\
\glt ‘First of all the women clear the bush for the garden.’ \\
\z


\ea
\gll  Nop-ap  weeser-eya  mua=ke  ikiw-ep  nomokowa         war-i-mik. \\
clear.bush-\textsc{ss.seq}  finish-2/3s.\textsc{ds}  man-\textsc{cf}  go-\textsc{ss.seq}  tree  cut-Np-\textsc{pr}.1/3p \\


\glt ‘When they have finished clearing the bush the men go and cut the trees.’ \\
\z


\ea
\gll  Nomokowa  war-ep  or-omak-eya  naap  ik-ok  nomokowa     reen-eya  saama  kuum-i-mik. \\
tree  cut-\textsc{ss.seq}  descend-\textsc{distr}/\textsc{pl}-2/3s.\textsc{ds}  thus  be-\textsc{ss}  tree  dry-2/3s.\textsc{ds}  cleared.bush  burn-Np-\textsc{pr}.1/3p \\


\glt ‘When they have cut the many trees down and they have stayed like that and the trees have dried they burn the cleared bush.’ \\
\z


\ea
\gll  Saama  kuum-ep  weeser-eya  kafa  ik-i-mik. \\
cleared.bush  burn-\textsc{ss.seq}  finish-2/3s.\textsc{ds}  unburnt.wood  roast-Np-\textsc{pr}.1/3p \\
\glt ‘When they have finished burning the cleared bush they burn the unburnt wood.’ \\
\z


\ea
\gll  Kafa  ik-ep  wakoria  fo-fook-i-mik. \\
unburnt.wood  roast-\textsc{ss.seq}  section  \textsc{rdp}-split-Np-\textsc{pr}.1/3p \\
\glt ‘They burn the unburnt wood and split the garden into sections.’ \\
\z


\ea
\gll  Wakoria  fo-fook-ap  weeser-eya  weria  faker-i-mik. \\
section  \textsc{rdp}-split-\textsc{ss.seq}  finish-2/3s.\textsc{ds}  planting.stick  raise-Np-\textsc{pr}.1/3p \\
\glt ‘They split the sections and when it is finished they raise the planting sticks (to make planting holes).’ \\
\z


\ea
\gll  Weria  faker-ap  mua  unowa  wiinar-iwkin  emeria                     uupura  up-i-mik. \\
planting.stick  raise-\textsc{ss.seq}  man  many  make.planting.holes-2/3p.\textsc{ds}  woman taro.seedling  plant-Np-\textsc{pr}.1/3p \\


\glt ‘They raise the planting sticks and many/all men make planting holes and women plant taro seedlings.’ \\
\z


\ea
\gll  Moma  nan  miiwa=pa  ik-ok  siiwa  erepam=i  ikur  naap  moma parew-eya  perek-i-mik. \\
taro  there  ground-\textsc{loc}  be-\textsc{ss}  moon  four-\textsc{qm}  five  thus  taro     mature-2/3s.\textsc{ds}  harvest-Np-\textsc{pr}.1/3p \\


\glt ‘The taro is there in the ground and in about four or five months it matures and is harvested.’ \\
\z


\ea
\gll  Ne  manina  erup  on-i-mik. \\
\textsc{add}  garden  two  make-Np-\textsc{pr}.1/3p \\
\glt ‘And we make two (kinds of) gardens.’ \\
\z


\ea
\gll  Manin  maneka,  manin  gelemuta. \\
garden  big  garden  small \\
\glt ‘Big garden(s) and small garden(s).’ \\
\z


\ea
\gll  Manin  maneka  unuma  ekina,  aria  manin  gelemuta  unuma  esewa. \\
garden  big  name  ekina  alright  garden  small  name  esewa. \\
\glt ‘The name of the big garden is ‘ekina’, the name of the small garden is ‘esewa’.’ \\
\z


\ea
\gll  Esewa  naap:  mua  kuisow  erup,  arow,  naap  on-i-mik. \\
esewa  thus  man  one  two  three  thus  make-Np-\textsc{pr}.1/3p \\
\glt ‘ ‘Esewa’ is like that: one man (makes) two (or) three, we do/make like that.’ \\
\z


\ea
\gll  Aria  manin  maneka,  ekina,  pun  naap:  mua  kuisow  manina  erup=i arow=i  naap. \\
alright  garden  big  ekina  also  thus  man  one  garden  two-\textsc{qm}           three-\textsc{qm}  thus \\
\glt ‘Alright the big garden, ‘ekina’, is also like that: one man (makes) two or three, like that.’ \\
\z


\ea
\gll  Ona  mua  oona  ook-i-mik. \\
3s.\textsc{gen}  man  bone  follow-Np-\textsc{pr}.1/3p \\
\glt ‘We work according to (each) man’s strength (lit: We follow man’s bone(s))’ \\
\z


\ea
\gll  Ne  manin  gelemuti-tik,  esewa,  nena  kookal-owa=pa  perek-i-nan. \\
\textsc{add}  garden  small-\textsc{rdp}  esewa  2s.\textsc{gen}  like-\textsc{nmz}-\textsc{loc}  harvest-Np-\textsc{fu}.2s \\
\glt ‘And the small gardens, ‘esewas’, you will harvest at your own liking.’ \\
\z


\ea
\gll  Nena  kuuf-i-nan,  parew-i-non,  eliw  perek-i-nan. \\
2s.\textsc{gen}  look-Np-\textsc{fu}.2s  mature-Np-\textsc{fu}.3s  well  harvest-Np-\textsc{fu}.2s \\
\glt ‘You watch it yourself, it will mature, (and) you may harvest it.’ \\
\z

\ea
\gll  Manin  maneka,  ekina,  naisow  nena  kookal-owa=pa  perek-owa{\upshape\footnotemark}   weetak. \\
garden  big  ekina  2s.\textsc{isol}  2s.\textsc{gen}  like-\textsc{nmz}-\textsc{loc}  harvest-\textsc{nmz}  no \\
\glt ‘you are not allowed to harvest the big garden, ‘ekina’, by yourself at  your own liking.’ \\
\footnotetext{This refers to harvesting the first taros of the year’s crop.}\z

\ea
\gll  Nena  owowa  onaiya  aakun-ep  perek-owa=ke  ik-ua. \\
2s.\textsc{gen}  village  with  talk-\textsc{ss.seq}  harvest-\textsc{nmz}-\textsc{cf}  be-\textsc{pa}.3s \\
\glt ‘The harvesting must take place (only) after you have talked with your village.’ \\
\z


\ea
\gll  Perek-ami  en-ow(a)  gelemuta  on-i-nan. \\
harvest-\textsc{ss}.\textsc{sim}  eat-\textsc{nmz}  small  make-Np-\textsc{fu}.2s \\
\glt ‘When you harvest you will make a feast.’ \\
\z


\ea
\gll  Waaya  ika-i-non  waaya  uup-i-nan  naap,  e  owowa  oko   wienak-owa  pun  naap,  nena  waaya  ik-ok  eliw  wienak-i-nan. \\
pig  be-Np-\textsc{fu}.3s  pig  cook-Np-\textsc{fu}.2s  thus  or  village  other   feed.them-\textsc{nmz}  also  thus  2s.\textsc{gen}  pig  be-\textsc{ss}  well  feed.them-Np-\textsc{fu}.2s \\


\glt ‘If you have a pig, you will cook pork, (it is) like that; or giving food to other villages is also like that: when you have a pig you may feed (it to) them. ’ \\
\z


\ea
\gll  Ne  manina  pun  naap,  manina  mauw-ap  manina  uruf-ow(a)  mua onaiya  ika-i-ya. \\
\textsc{add}  garden  also  thus  garden  work-\textsc{ss.seq}  garden  look-\textsc{nmz}  man    with  be-Np-\textsc{pr}.3s \\


\glt ‘And the garden is like this too: when the garden is done, it has (lit: is with) a guardian.’ \\
\z


\ea
\gll  Manina  waisow  mauw-ap  neeke  wafur-ap-pu-owa          nain  weetak. \\
garden  3s.\textsc{isol}  work-\textsc{ss.seq}  there.\textsc{cf}  throw-\textsc{ss.seq}-\textsc{cmpl}-\textsc{nmz}  that1  no \\


\glt ‘You may not make the garden by yourself and (just) leave it there.’ \\
\z


\ea
\gll  Manina  kuuf-owa  mua  onaiya  ika-i-ya. \\
manina  look-\textsc{nmz}  man  with  be-Np-\textsc{pr}.3s \\
\glt ‘The garden has a guardian.’ \\
\z


\ea
\gll  Mua  nain=ke,  “Aakisa  moma  parew-o-k. \\
man  that1-\textsc{cf}  now  taro  mature-\textsc{pa}-3s \\
\glt ‘That man (will say), “Now the taro has matured.” ’ \\
\z


\ea
\gll  Aria  opora  fiirim-ep,  fofa  wu-ap,  iir  nain  perek-i-yan.” \\
alright  talk  gather-\textsc{ss.seq}  day  put-\textsc{ss.seq}  time  that1  harvest-Np-\textsc{fu}.1p \\
\glt ‘Alright, we will discuss it, set a date and at that time harvest it.” ’ \\
\z


\ea
\gll  Aria  ona  fofa  kerer-eya  perek-i-mik. \\
alright  3s.\textsc{gen}  day  appear-2/3s  harvest-Np-\textsc{pr}.1/3p \\
\glt ‘Alright when the day comes we harvest it.’ \\
\z


\ea
\gll  Nain  manin  maneka  ma-i-yem,  ekina. \\
that1  garden  big  say-Np-\textsc{pr}.1s  ekina \\
\glt ‘I am saying that about the big garden, ‘ekina’. ’ \\
\z


\ea
\gll  O  manin  gelemuta,  esewa,  nain  naap,  ona  mua  kookal-owa. \\
\textsc{intj}  garden  small  esewa  that1  thus  3s.\textsc{gen}  man  like-\textsc{nmz} \\
\glt ‘The small garden, ‘esewa’, is like this, (it can be worked according to) man’s own liking.’ \\
\z


\ea
\gll  Ona  weniwa=pa  en-owa  na-ep  uuw-i-mik. \\
3s.\textsc{gen}  hunger.time-\textsc{loc}  eat-\textsc{nmz}  say-\textsc{ss.seq}  work-Np-\textsc{pr}.1/3p \\
\glt ‘It is worked for eating during the “hunger time” (when there is no taro).’ \\
\z


\ea
\gll  Ne  esewa  nain  no  aakisa  feenap  nain  mauw-am-ika-i-nan. \\
\textsc{add}  esewa  that1  2s.\textsc{unm}  now  like.this  that1  work-\textsc{ss}.\textsc{sim}-be-Np-\textsc{fu}.2s \\
\glt ‘And the ‘esewa’ you can be working around this time.’ \\
\z


\ea
\gll  O  manin  maneka,  ekina=ke,  ikoka  mauw-owa  pun  eliw. \\
\textsc{intj}  garden  big  ekina-\textsc{cf}  later  work-\textsc{nmz}  also  well \\
\glt ‘The big garden, ‘ekina’, may also be made/worked later.’ \\
\z


\ea
\gll  Ne  moma  perek-owa  pun  naap,  ona  owow  saria=ke  kiikir       perek-i-mik,  mua  oro-oram  fain  weetak. \\
\textsc{add}  taro  harvest-\textsc{nmz}  also  thus  3s.\textsc{gen}  village  headman-\textsc{cf}  first   harvest-Np-\textsc{pr}.1/3p  man  \textsc{rdp}-insignificant  this  no \\


\glt ‘And  taro harvesting is also like that, the village headmen harvest it first, not these ordinary men.’ \\
\z


\ea
\gll  Owow  saria=ke  perek-iwkin  aria  mua  oko  pun  perek-i-mik. \\
village  headman-\textsc{cf}  harvest-2/3p.\textsc{ds}  alright  man  other  also  harvest-Np-\textsc{pr}.1/3p \\
\glt ‘When the village headmen have harvested first, other men harvest too.’ \\
\z


\ea
\gll  Opaimika  muut  nan-e-k,  weeser-e-k. \\
talk  only  there-\textsc{pa}-3s  finished-\textsc{pa}-3s \\
\glt ‘The talk is there, it is finished.’ \\
\z

Yo aakisa manina uuwowa opora mainen naep.

Kiikir akena emeria-ke manina nopimik. 
\textrm{Nopap weesereya  mua-ke ikiwep nomokowa warimik. }
\textrm{Nomokowa warep oromakeya naap ikok nomokowa reeneya  saama kuumimik. }
\textrm{Saama kuumep weesereya kafa ikimik. }
\textrm{Kafa ikep wakoria fo-fookimik. }
\textrm{Wakoria fo-fookap weesereya weria fakerimik. }
\textrm{Weria fakerap mua unowa wiinariwkin emeria uupura upimik. }
\textrm{Moma nan miiwa-pa ikok siiwa erepam-i ikur naap moma pareweya perekimik. }

Ne manina erup onimik. 
Manin maneka, manin gelemuta. 
\textrm{Manin maneka unuma }\textrm{\textit{ekina}}\textrm{,  aria manin gelemuta unuma }\textrm{\textit{esewa}}\textrm{. }
\textrm{Esewa naap:  mua kuisow, erup, arow, naap onimik. }
\textrm{Aria manin maneka, }\textrm{\textit{ekina}}\textrm{, pun naap:  mua kuisow manina erup-i arow-i naap. }
Ona mua oona ookimik. 

\textrm{Ne manin gelemutitik, }\textrm{\textit{esewa}}\textrm{, nena kookalowa-pa perekinan. }
Nena kuufinan,  parewinon,  eliw perekinan. 
\textrm{Manin maneka, }\textrm{\textit{ekina}}\textrm{, naisow nena kookalowa-pa perekowa weetak. }
\textrm{Nena owowa onaiya aakunep perekowa-ke ikua. }
Perekami  enow gelemuta oninan. 
\textrm{Waaya ikainon waaya uupinan naap,  e owowa oko wienakowa pun naap,  nena waaya ikok  eliw wienakinan. }

\textrm{Ne manina pun naap,  manina mauwap manina urufow mua onaiya ikaiya. }
Manina waisow mauwap neeke wafurap-puowa nain weetak. 
Manina kuufowa mua onaiya ikaiya. 
Mua nain-ke,  “Aakisa moma parewok. 
\textrm{Aria opora fiirimep,  fofa wuap,  iir nain perekiyan.” }
Aria ona fofa kerereya  perekimik. 
\textrm{Nain manin maneka maiyem, }\textrm{\textit{ekina}}\textrm{. }
\textrm{O manin gelemuta, }\textrm{\textit{esewa}}\textrm{, nain naap,  ona mua kookalowa. }
Ona weniwa-pa enowa naep uuwimik. 

\textrm{Ne }\textrm{\textit{esewa}}\textrm{ nain no aakisa feenap nain mauwam-ikainan. }
\textrm{O manin maneka, }\textrm{\textit{ekina}}\textrm{{}-ke, ikoka mauwowa pun eliw. }
Ne moma perekowa pun naap,  ona owow saria-ke kiikir perekimik,  mua or-oram fain weetak. 
Owow saria-ke perekiwkin  aria mua oko pun perekimik. 
Opaimika mut nanek, weeserek. 


\section{Girls’ initiation customs}\label{app:2:girls}
by Kalina Sarak
\ea
\gll  I  wiipa  siiwa  me  wia  kuuf-owa  ik-ok  wia  kuuf-i-ya          nain  aite=ke  koora=pa  yia  kaik-i-mik. \\
1p.\textsc{unm}  girl  moon  not  3p.\textsc{acc}  look-\textsc{nmz}  be-\textsc{ss}  3p.\textsc{acc}  look-Np-\textsc{pr}.3s  that1  mother-\textsc{cf}  house-\textsc{loc}  1p.\textsc{acc}  tie-Np-\textsc{pr}.1/3p \\


\glt ‘When the moon hasn’t yet looked at our girls and then looks at them (I.e. when they have their first menstruation) our mothers seclude us in the house.’ \\
\z


\ea
\gll  Koora=pa  yia  kaik-iwkin  nan  ika-i-mik. \\
house-\textsc{loc}  1p.\textsc{acc}  tie-2/3p.\textsc{ds}  there  be-Np-\textsc{pr}.1/3p \\
\glt ‘They seclude us in the house and we stay there.’ \\
\z


\ea
\gll  Nan  ik-ok  aite=ke  moma  yia  ik-om-i-ya,          akia  yia  ik-om-i-ya. \\
there  be-\textsc{ss}  mother-\textsc{cf}  taro  1p.\textsc{acc}  roast-\textsc{ben}-Np-\textsc{pr}.3s  banana  1p.\textsc{acc}  roast-\textsc{ben}-Np-\textsc{pr}.3s \\


\glt ‘We stay there and mother roasts us taros, she roasts us bananas’ \\
\z


\ea
\gll  Koora=pa  nan  yiam  kaik-ap  ik-ok  omopora  yiam  up-ep           ik-ok  akia  ik-owa  enim-i-mik,  moma  ik-owa  enim-i-mik. \\
house-\textsc{loc}  there  1p.\textsc{refl}  tie-\textsc{ss.seq}  be-\textsc{ss}  door  1p.\textsc{refl}  close-\textsc{ss.seq} be-\textsc{ss}  banana  roast-\textsc{nmz}  eat-Np-\textsc{pr}.1/3p  taro  roast-\textsc{nmz}  eat-Np-\textsc{pr}.1/3p \\


\glt ‘We are secluded there in the house and have locked the door on ourselves and eat roasted bananas and taros.’ \\
\z


\ea
\gll  Eka  me  enim-i-mik,  iwera  eka  me  enim-i-mik. \\
water  not  eat-Np-\textsc{pr}.1/3p  coconut  water  not  eat-Np-\textsc{pr}.1/3p \\
\glt ‘We don’t drink water, we don’t drink coconut water.’ \\
\z


\ea
\gll  Aaya  muutiw  en-em-ika-i-mik. \\
sugarcane  only  eat-\textsc{ss}.\textsc{sim}-be-Np-\textsc{pr}.1/3p \\
\glt ‘We only keep eating sugarcane (i.e. chew the sugarcane and “drink” the juice from it).’ \\
\z


\ea
\gll  Nain  en-em-ik-ok  in-i-mik,  kokom-ar-i-ya,            in-i-mik,  epa  wiim-i-ya. \\
that1  eat-\textsc{ss}.\textsc{sim}-be-\textsc{ss}  sleep-Np-\textsc{pr}.1/3p  dark-\textsc{appl}-Np-\textsc{pr}.3s sleep-Np-\textsc{pr}.1/3p  place  dawn-Np-\textsc{pr}.3s \\


\glt ‘We eat that and sleep, it becomes dark, we sleep, (then) it dawns.’ \\
\z


\ea
\gll  Aite  nainiw  maa  yia  ik-om-i-mik. \\
mother  again  food  1p.\textsc{acc}  roast-\textsc{ben}-Np-\textsc{pr}.1/3p \\
\glt ‘Our mothers again roast food for us.’ \\
\z


\ea
\gll  Ik-om-iwkin  enim-i-mik. \\
roast-\textsc{ben}-2/3p.\textsc{ds}  eat-Np-\textsc{pr}.1/3p \\
\glt ‘They roast it for us and we eat.’ \\
\z


\ea
\gll  Aaya  muuna  kuisow,  muuna  kuisow  enim-i-mik,  aite=ke             manina=pa  yia  aaw-om-iwkin. \\
sugarcane  joint.length  one  joint.length  one  eat-Np-\textsc{pr}.1/3p  mother-\textsc{cf}  garden-\textsc{loc}  1p.\textsc{acc}  get-\textsc{ben}-2/3p.\textsc{ds} \\


\glt ‘We eat one joint length of a sugarcane, one joint length, when our mothers have gotten it for us from the garden.’ \\
\z


\ea
\gll  Nan  ika-i-mik. \\
there  be-Np-\textsc{pr}.1/3p \\
\glt ‘We stay there.’ \\
\z


\ea
\gll  Ik-ok  ik-ok  aite  ona  siiwa  ara  onaria  ma-i-ya,  “Wiipa       no  aakisa  ora-e,  no  nan  pok-a-n,  owowa  uruma                         ora-e”. \\
be-\textsc{ss}  be-\textsc{ss}  mother  3s.\textsc{gen}  moon  section  with  say-Np-\textsc{pr}.3s  daughter 2s.\textsc{unm}  now  descend-\textsc{imp}.2s  2s.\textsc{unm}  there  sit-\textsc{pa}-2s  village  open.place  descend-\textsc{imp}.2s \\




\glt ‘We stay and stay and at the right time of the moon mother says, “Daughter, go down now, you have sat there, go down to the open.’ \\
\z


\ea
\gll  Yia  na-eya,  fofa  yia  wu-om-eya          i  yak-i-mik. \\
1p.\textsc{acc}  say-2/3s.\textsc{ds}  day  1p.\textsc{acc}  put-\textsc{ben}-2/3s.\textsc{ds} 1p.\textsc{unm}  bathe-Np-\textsc{pr}.1/3p \\


\glt ‘When she says so about us and sets the date for us we bathe.’ \\
\z


\ea
\gll  Yaki-ep  urup-emi  koora=pa  nan  pok-ap  ika-i-mik. \\
bathe.\textsc{ss.seq}  ascend-\textsc{ss}.\textsc{sim}  house-\textsc{loc}  there  sit-\textsc{ss.seq}  be-Np-\textsc{pr}.1/3p \\
\glt ‘We bathe and go (back) up and sit in the house.’ \\
\z


\ea
\gll  Sosora  a-i-mik,  aite=ke  fia  yia  aw-om-i-mik,                      ofa  op-i-mik,  epa  maneka  ora-i-mik. \\
grass.skirt  tie-Np-\textsc{pr}.1/3p  mother-\textsc{cf}  hair  1p.\textsc{acc}  shave-\textsc{ben}-Np-\textsc{pr}.1/3p   red.dye  hold-Np-\textsc{pr}.1/3p  place  big  descend-Np-\textsc{pr}.1/3p \\


\glt ‘We tie a grass skirt on, our mothers shave our hair, we are painted with red dye and we go down to the open.’ \\
\z


\ea
\gll  Epa  maneka  or-op  maa  ikina  ewur  me  enim-i-mik,         mera  eka. \\
place  big  descend-\textsc{ss.seq}  food  smell  soon  not  eat-Np-\textsc{pr}.1/3p  fish  water \\


\glt ‘We go down to the open but we do not eat meaty (lit: smelly) food soon, or fish soup.’ \\
\z


\ea
\gll  Nan  ika-i-mik,  nan  ik-ok,  siiwa  oko  kerer-eya  i             maa  ikina  enim-i-mik. \\
there  be-Np-\textsc{pr}.1/3p  there  be-\textsc{ss}  moon  other  appear-2/3s.\textsc{ds}  1p.\textsc{unm}  food  smell  eat-Np-\textsc{pr}.1/3p \\


\glt ‘We stay there, we stay there and when another moon appears we eat meaty food.’ \\
\z


\ea
\gll  Aite=ke  ma-i-mik,  “No  aakisa  maa  ikina=ko  enim-e”,  a        i  maa  ikina  enim-i-mik. \\
mother-\textsc{cf}  say-Np-\textsc{pr}.1/3p  2s.\textsc{unm}  now  food  smell-\textsc{nf}  eat-\textsc{imp}.2s  \textsc{intj} 1p.\textsc{unm}  food  smell  eat-Np-\textsc{pr}.1/3p \\


\glt ‘Our mothers say, “Now eat meaty food”, so we eat meaty food.’ \\
\z


\ea
\gll  A  weeser-e-k,  i  maa  momor  enim-i-yen,  waaya,                maa  mauwa,  aara,  nepa  enim-i-mik. \\
\textsc{intj}  finish-\textsc{pa}-3s  1p.\textsc{unm}  food  indiscriminately  eat-Np-\textsc{fu}.1p  pig   food  what  chicken  bird  eat-Np-\textsc{pr}.1/3p \\


\glt ‘It is finished, and we eat any food, we eat pork, whatever, chicken, birds.’ \\
\z


\ea
\gll  I  ikoka  ikoka  wia,  maa  momor  enim-i-yen. \\
1p.\textsc{unm}  later  later  no  food  indiscriminately  eat-Np-\textsc{fu}.1p \\
\glt ‘Not sometime in the future, we (can now) eat food indiscriminately.’ \\
\z


\ea
\gll  I  naap  on-i-mik,  i  sira.  Weeser-e-k. \\
1p.\textsc{unm}  thus  do-Np-\textsc{pr}.1/3p  1p.\textsc{unm}  custom  finish-\textsc{pa}-3s \\
\glt ‘We do like that, it is our custom. The end.’ \\
\z

I wiipa siiwa me wia kuufowa ikok wia kuufiya nain nain aite-ke koora-pa yia kaikimik. 
Koora-pa yia kaikiwkin  nan ikaimik. 
Nan ikok aite-ke moma yia ikomiya, akia yia ikomiya. 
Koora-pa nan yiam kaikap ikok omopora yiam upep ikok  akia ikowa enimimik,  moma ikowa enimimik. 
Eka me enimimik, iwera eka me enimimik. 
Aaya muutiw enem-ikaimik. 
Nain enem-ikok inimik, kokomariya, inimik, epa wiimiya. 
Aite nainiw maa yia ikomimik. 
Ikomiwkin enimimik. 
Aaya muuna kuisow, muuna kuisow enimimik, aite-ke manina-pa yia aawomiwkin. 
Nan ikaimik.

Ikok ikok aite ona siiwa ara onaria maiya, “Wiipa no aakisa orae, no nan pokan, owowa uruma orae.”
Yia naeya, fofa yia wuomeya i yakimik. 
Yakiep urupemi koora-pa nan pokap ikaimik. 
Sosora aimik, aite-ke fia yia awomimik, ofa opimik, epa maneka oraimik. 
Epa maneka orop maa ikina ewur me enimimik, mera eka.

Nan ikaimik, nan ikok siiwa oko kerereya i maa ikina enimimik. 
Aite-ke maimik, “No aakisa maa ikina-ko enime”, a i maa ikina enimimik. 
A weeserek, i maa momor enimiyen, waaya, maa mauwa, aara, nepa enimimik. 
I ikoka ikoka wia, maa momor enimiyen. 
I naap onimik, i sira. Weeserek. 


\section{Funeral customs}\label{app:2:funeral}
by Kalina Sarak
\ea
\gll  I  mua  soop-owa  sira,  i  yiena  kae  sira,                    kome  sira,  a  naap,  i  wiipa  mauwa  sira  saarik. \\
1p.\textsc{unm}  man  bury-\textsc{nmz}  custom  1p.\textsc{unm}  1p.\textsc{gen}  1s/p.grandfather  custom 1s/p.grandmother  custom  \textsc{intj}  thus  1p.\textsc{unm}  daughter  what  custom  like \\


\glt ‘Our custom of burying our husbands – our grandfathers’ custom, our grandmothers’ custom – is like that, similar to the custom of our daughters’ what (initiation) custom.’ \\
\z


\ea
\gll  I  mua  um-iya,  mua  om-ep  om-ep  om-ep  mua                    napuma  yiar  soop-i-mik. \\
1p.\textsc{unm}  man  die-2/3s.\textsc{ds}  man  cry-\textsc{ss.seq}  cry-\textsc{ss.seq}  cry-\textsc{ss.seq}  man  body  1p.\textsc{dat}  bury-Np-\textsc{pr}.1/3p \\


\glt ‘When the husband dies we cry and cry and cry and the man’s body is buried.’ \\
\z


\ea
\gll  Ikiw-ep  eruwa=pa  wia  wua-i-mik. \\
go-\textsc{ss.seq}  grave-\textsc{loc}  3p.\textsc{acc}  put-Np-\textsc{pr}.1/3p \\
\glt ‘It is taken and put in the grave.’ \\
\z


\ea
\gll  Eruwa=pa  wia  wua-iwkin  i  emeria  apura  yiena      mua  weria  emeria  nain=ke  yia  amap-ikiw-ep  eka=pa                  yia  yakuw-ap  ekap-emi  koora=pa  yia  wua-i-mik. \\
grave-\textsc{loc}  3p.\textsc{acc}  put-2/3p.\textsc{ds}  1p.\textsc{unm}  woman  widow  1p.\textsc{gen} man  weria.relative  woman  that1-\textsc{cf}  1p.\textsc{acc}  Bpx-go-\textsc{ss.seq}  water-\textsc{loc}  1p.\textsc{acc}  wash-\textsc{ss.seq}  come-\textsc{ss}.\textsc{sim}  house-\textsc{loc}  1p.\textsc{acc}  put-Np-\textsc{pr}.1/3p \\




\glt ‘It is put in the grave and we widows are taken by our \textit{weria}{}-relatives’\footnote{ Certain relatives (maternal uncles and male cousins) responsible for burying a person} wives and washed in a spring and brought and put in the house.’ \\
\z


\ea
\gll  Yiena  koora=pa  yia  wua-iwkin  naap  yiena  koora=pa      nan  ika-i-mik. \\
1p.\textsc{gen}  house-\textsc{loc}  1p.\textsc{acc}  put-2/3p.\textsc{ds}  thus  1p.\textsc{gen}  house-\textsc{loc}   there  be-Np-1/3p \\


\glt ‘They put us in the house and we stay there in our house like that.’ \\
\z


\ea
\gll  Koora=pa  nan  kaik-ap  ika-i-mik. \\
house-\textsc{loc}  there  tie-\textsc{ss.seq}  be-Np-\textsc{pr}.1/3p \\
\glt ‘We stay secluded there in the house.’ \\
\z


\ea
\gll  Ik-ok  ik-ok  moma  ik-owa  en-em-ik-ok,  siiwa  kuisow. \\
be-\textsc{ss}  be-\textsc{ss}  taro  roast-\textsc{nmz}  eat-\textsc{ss}.\textsc{sim}-be-\textsc{ss}  moon  one \\
\glt ‘We stay and stay and (only) keep eating roasted taro (until) one month (is gone).’ \\
\z


\ea
\gll  Siiwa  kuisow  nain  okaiwi=pa  kerer-eya  yia  maak-i-mik,           “Aria  aakisa  apura  nain  yak-inok.” \\
moon  one  that1  other.side-\textsc{loc}  appear-2/3s.\textsc{ds}  1p.\textsc{acc}  tell-Np-\textsc{pr}.1/3p   alright  now  widow  that1  bathe-\textsc{imp}.3s \\


\glt ‘When that one month (is finished and) another starts they tell us, “Alright now let the widow bathe.” ’ \\
\z


\ea
\gll  A  yak-i-mik,  yiena  mua  weria  emeria=ke             yook-ap  er-iwkin  yak-i-mik. \\
\textsc{intj}  bathe-Np-\textsc{pr}.1/3p  1p.\textsc{gen}  man  weria.relative  woman-\textsc{cf} follow.us-\textsc{ss.seq}  go-2/3p.\textsc{ds}  bathe-Np-\textsc{pr}.1/3p \\


\glt ‘So we bathe, our \textit{weria}{}-relatives’ wives follow us (there) and we bathe.’ \\
\z


\ea
\gll  Er-ap  eka  damola=pa  yaki-ep  sosora  a-i-mik,                maa  bala  suuw-i-mik,  kamukamu,  kululuma,  sagat. \\
go-\textsc{ss.seq}  water  bad-\textsc{loc}  bathe-\textsc{ss.seq}  grass.skirt  tie-Np-\textsc{pr}.1/3p  thing  ornament  push-Np-\textsc{pr}.1/3p  Job’s.tears  coloured.bead  sagat.shell \\


\glt ‘We go and bathe in the bad spring and (then) tie the grass skirt on and put on decorations, Job’s tears, coloured beads and sagat shells.’ \\
\z


\ea
\gll  Fia  yia  aw-om-iwkin  ofa,  ir-ap  owowa=pa,              yiena  koora=pa. \\
hair  1p.\textsc{acc}  shave-\textsc{ben}-2/3p.\textsc{ds}  red.dye  go-\textsc{ss.seq}  village-\textsc{loc} 1p.\textsc{gen}  house-\textsc{loc} \\


\glt ‘When they have shaved our heads we (put on) red dye – having come back – in the village, in our house.’ \\
\z


\ea
\gll  Yiena  koora=pa  op-ap,  maa  pi-ep,                      saasaria  pi-ep  owowa  uruma  ora-i-mik. \\
1p.\textsc{gen}  house-\textsc{loc}  hold-\textsc{ss.seq}  thing  stick.in.armband-\textsc{ss.seq}  plant.sp.  stick.in.armband-\textsc{ss.seq}  village  open.space  descend-Np-\textsc{pr}.1/3p \\


\glt ‘In our house we put on (the red dye), stick things in the armbands, stick \textit{saasaria} plant in the armbands and go down in the open.’ \\
\z


\ea
\gll  Or-omkun  mua  weria=ke  meta  urupa  op-ap                      keemamuuna,  umakuna  meta  yia  mik-i-mik. \\
descend-1s/p.\textsc{ds}  man  weria.relative-\textsc{cf}  meta.paste  cup  hold-\textsc{ss.seq}  knee  neck  meta.paste  1p.\textsc{acc}  stick-Np-\textsc{pr}.1/3p \\


\glt ‘We go down and the \textit{weria}{}-relatives hold the \textit{meta}{}-paste cup and stick the \textit{meta} paste on (the back of) our knees and on the neck.’ \\
\z


\ea
\gll  Meta  yia  mik-iwkin  mera,  waaya,  oposia  tiira  nain    yia  aaw-om-iwkin  furun-i-mik. \\
meta.paste  1p.\textsc{acc}  stick-2/3p.\textsc{ds}  fish  pig  meat  slice  that1  1p.\textsc{acc}  get-\textsc{ben}-2/3p.\textsc{ds}  spit-Np-\textsc{pr}.1/3p \\


\glt ‘They\textsubscript{1} stick the \textit{meta} paste on us and when they\textsubscript{2} (=others) get fish and pork, pieces of those meats for us they\textsubscript{1} spit it.’ \\
\z


\ea
\gll  Ama  urup-owa,  ama  or-owa  furun-i-mik,  koora  ikiw-i-mik. \\
sun  ascend-\textsc{nmz}  sun  descend-\textsc{nmz}  spit-Np-\textsc{pr}.1/3p  house  go-Np-\textsc{pr}.1/3p \\
\glt ‘They spit it towards east, they spit it towards west, and we go into the house.’ \\
\z


\ea
\gll  Koora  ikiw-i-mik  nain  mera  eka  me  enim-i-mik. \\
house  go-Np-\textsc{pr}.1/3p  that1  fish  water  not  eat-Np-\textsc{pr}.1/3p \\
\glt ‘We who go into the house do not eat fish soup.’ \\
\z


\ea
\gll  Mera  eka  en-owa  marew  nan  ika-i-mik. \\
fish  water  eat-\textsc{nmz}  no(ne)  there  be-Np-\textsc{pr}.1/3p \\
\glt ‘We stay there without eating fish soup.’ \\
\z


\ea
\gll  Mera  eka  en-owa  marew  ik-ok  ik-ok  mua  weria=ke  ma-i-ya,            “No  aakisa  maa  ikina=ko  enim-e.” \\
fish  water  eat-\textsc{nmz}  no(ne)  be-\textsc{ss}  be-\textsc{ss}  man  weria.relative-\textsc{cf}  say-Np-\textsc{pr}.3s 2s.\textsc{unm}  now  food  smell-\textsc{nf}  eat-\textsc{imp}.2s \\


\glt ‘We stay and stay without eating fish soup and (then) a \textit{weria}{}-relative says, Now you (can) eat meaty food.’ \\
\z


\ea
\gll  Ne  maa  ikina  enim-i-mik,  mera  eka  enim-i-mik. \\
\textsc{add}  food  smell  eat-Np-\textsc{pr}.1/3p  fish  water  eat-Np-\textsc{pr}.1/3p \\
\glt ‘And we eat meaty food, we eat fish soup.’ \\
\z


\ea
\gll  Mera  eka  enim-i-mik  nain  i  mangala  me  enim-i-mik,               waaya  me  enim-i-mik,  mua  wia  soop-i-mik  nain. \\
fish  water  eat-Np-\textsc{pr}.1/3p  that1  1p.\textsc{unm}  drupa.shell  not  eat-Np-\textsc{pr}.1/3p pig  not  eat-Np-\textsc{pr}.1/3p  man  3p.\textsc{acc}  bury-Np-\textsc{pr}.1/3p  that \\


\glt ‘We eat fish soup but we we do not eat drupa shells, we do not eat pork, we who bury our husbands.’ \\
\z


\ea
\gll  A  i  sira,  aria  nepa  me  enim-i-mik,  aara              me  enim-i-mik,  maroka  me  enim-i-mik,  sibaur         me  enim-i-mik. \\
\textsc{intj}  1p.\textsc{unm}  custom  alright  bird  not  eat-Np-\textsc{pr}.1/3p  chicken not  eat-Np-\textsc{pr}.1/3p  prawn  not  eat-Np-\textsc{pr}.1/3p  lobster not  eat-Np-\textsc{pr}.1/3p \\




\glt ‘Oh that is our custom, alright we do not eat bird meat, we do not eat chicken, we do not eat prawns, we do not eat lobsters.’ \\
\z


\ea
\gll  Mangala  eka  me  enim-i-mik. \\
drupa.shell  water  not  eat-Np-\textsc{pr}.1/3p \\
\glt ‘We do not eat drupa shell soup.’ \\
\z


\ea
\gll  Emi  kekanowa  maneka,  naap  oram  ika-i-mik. \\
taboo  strong  big  thus  just  be-Np-\textsc{pr}.1/3p \\
\glt ‘It is a big taboo, we stay like that just (without many foods).’ \\
\z


\ea
\gll  I  ikoka  nain  enim-i-yen  mua  kukusa=ke  mia  yia       damol-iwkin  i  eliwa  me  ika-i-yen. \\
1p.\textsc{unm}  later  that1  eat-Np-\textsc{fu}.1p  man  spirit-\textsc{cf}  body  1p.\textsc{acc} bad-2/3p.\textsc{ds}  1p.\textsc{unm}  good  not  be-Np-\textsc{fu}.1p \\


\glt ‘If we later eat that our husbands’ spirits will damage our bodies and we will not be well.’ \\
\z


\ea
\gll  Mia  nigisir-i-yan,  panewowa  saarik  ika-i-yan. \\
body  shrink-Np-\textsc{fu}.1p  old  like  be-Np-\textsc{fu}.1p \\
\glt ‘Our bodies will shrink and we will be like old people.’ \\
\z


\ea
\gll  I  sira  naap  yiar  ik-ua,  i  mua  soop-owa  sira,          i  emeria  sira=ke. \\
1p.\textsc{unm}  custom  thus  1p.\textsc{dat}  1p.\textsc{unm}  1p.\textsc{unm}  man  bury-\textsc{nmz}  custom 1p.\textsc{unm}  woman  custom-\textsc{cf} \\


\glt ‘We have a custom like that, the custom of burying our husbands, it is the womens’ custom.’ \\
\z


\ea
\gll  Aria  i  emeria  sira  nan  weeser-e-k. \\
alright  1p.\textsc{unm}  woman  custom  there  finish-\textsc{pa}.3s \\
\glt ‘Alright (telling about) the women’s custom is finished.’ \\
\z


\ea
\gll  A  pun  naap,  mua  emeria  um-i-mik  wi  koora=pa  nan          ika-i-mik. \\
\textsc{intj}  also  thus  man  woman  die-Np-\textsc{pr}.1/3p  3p.\textsc{unm}  house-\textsc{loc}  there be-Np-\textsc{pr}.1/3p \\


\glt ‘Also in the same way, the men’s wives die and they stay there in the house.’ \\
\z


\ea
\gll  Onak=ke  moma  wia  ik-em-ik-om-i-mik,  maa                   muutumut,  akia,  iwoka. \\
3s/p.mother-\textsc{cf}  taro  3p.\textsc{acc}  roast-\textsc{ss}.\textsc{sim}-be-\textsc{ben}-Np-\textsc{pr}.1/3p  food  all.kinds  banana  yam \\


\glt ‘Their mothers roast taro for them, and all kinds of food, banana, yam.’ \\
\z


\ea
\gll  Wia  ik-om-iwkin  en-em-ika-iwkin  en-em-ika-iwkin                siiwa  kuisow,  koora=pa  nan  ika-i-mik  nain. \\
3p.\textsc{acc}  roast-\textsc{ben}-2/3p.\textsc{ds}  eat-\textsc{ss}.\textsc{sim}-be-2/3p.\textsc{ds}  eat-\textsc{ss}.\textsc{sim}-be-2/3p.\textsc{ds} moon  one  house-\textsc{loc}  there  be-Np-\textsc{pr}-1/3p  that1 \\


\glt ‘They\textsubscript{1} roast it for them\textsubscript{2 }and they\textsubscript{2} keep eating it (until) a month (is gone), those who stay there in the house.’ \\
\z


\ea
\gll  Siiwa  kuisow  ona  onak=ke  wiawi=ke  ma-i-ya,  “Aakisa        yo  muuka  yak-i-ya.” \\
moon  one  3s.\textsc{gen}  3s/p.mother-\textsc{cf}  3s/p.father-\textsc{cf}  say-Np-\textsc{pr}.3s  now  1s.\textsc{unm}  son  bathe-Np-\textsc{pr}.3s \\


\glt ‘(After) one month his mother (or) his father says, “Now my son bathes.” ’ \\
\z


\ea
\gll  Mua  weria  ekap-ep  koora=pa  nan  ik-ok  muuka         aap-ora-i-mik. \\
man  weria.relative  come-\textsc{ss.seq}  house-\textsc{loc}  there  be-\textsc{ss}  son Bpx-descend-Np-\textsc{pr}.1/3p \\


\glt ‘The \textit{weria}{}-relatives come and stay (a while) in the house and bring the son down.’ \\
\z


\ea
\gll  Muuka  p-or-op  p-er-iwkin  yak-i-ya. \\
son  Bpx-descend-\textsc{ss.seq}  Bpx-go-2/3p.\textsc{ds}  bathe-Np-\textsc{pr}.3s \\
\glt ‘They bring the son down and take him (to the spring) and he bathes.’ \\
\z


\ea
\gll  Eka=pa  yakuw-ap  p-ir-i-mik. \\
water-\textsc{loc}  wash-\textsc{ss.seq}  Bpx-come-Np-\textsc{pr}.1/3p \\
\glt ‘They wash him in the water and bring him (to the village).’ \\
\z


\ea
\gll  P-ir-ami  sira  naap  emeria  saarik. \\
Bpx-come-\textsc{ss}.\textsc{sim}  custom  thus  woman  like \\
\glt ‘They bring him and (do) the custom like that, similar to the women(’s custom).’ \\
\z


\ea
\gll  Mera  eka  me  ewur  enim-i-ya,  nan  ika-i-ya,  eka=iw           en-em-ika-i-non. \\
fish  water  not  quickly  eat-Np-\textsc{pr}.3s  there  be-Np-\textsc{pr}.3s  water-\textsc{inst} eat-\textsc{ss}.\textsc{sim}-be-Np-\textsc{fu}.3s \\


\glt ‘He does not eat fish soup soon, he stays there, he will (only) eat (food cooked) with water.’ \\
\z


\ea
\gll  Maa  eka=iw  en-ep  en-ep  siiwa  ara  onaria,  maa  ikina,          mangala  wia,  waaya  wia,  aara  wia,  urema  wia,  maroka  wia, maa  muutumut  wia. \\
food  water-\textsc{inst}  est-\textsc{ss.seq}  eat-\textsc{ss.seq}  moon  section  with  food  smell drupa.shell  no  pig  no  chicken  no  bandicoot  no  prawn  no   food  all.kinds  no \\




\glt ‘He will keep eating food (cooked) with water until the moon is right, no meaty food, drupa shell, no pork, no chicken, no bandicoot, no prawns; all kinds of food are forbidden.’ \\
\z


\ea
\gll  Ikoka  ona  mua  weria=ke  wia  op-om-iwkin  enim-i-non. \\
later  3s.\textsc{gen}  man  weria.relative-\textsc{cf}  3p.\textsc{acc}  hold-\textsc{ben}-2/3p.\textsc{ds}  eat-Np-\textsc{fu}.3s \\
\glt ‘Later when his \textit{weria}{}-relatives hold them for him (=make a ceremony) he will eat them.’ \\
\z


\ea
\gll  A  wi  mua  sira=ke,  emeria  soop-owa  sira,  wi  mua        era=ke. \\
\textsc{intj}  3p.\textsc{unm}  man  custom-\textsc{cf}  woman  bury-\textsc{nmz}  custom,  3p.\textsc{unm}  man way-\textsc{cf} \\


\glt ‘That is the men’s custom, the custom of burying one’s wife, the way of the men.’ \\
\z


\ea
\gll  Opaimika  muut  nan-e-k. \\
talk  only  there-\textsc{pa}-3s \\
\glt ‘There’s the talk.’ \\
\z


\ea
\gll  Weeser-e-k. \\
finish-\textsc{pa}-3s \\
\glt ‘It is finished.’ \\
\z

I mua soopowa sira, i yiena kae sira, kome sira, a naap, i wiipa mauwa sira saarik. 
I mua umiya, mua omep omep omep mua napuma yiar soopimik.
Ikiwep eruwa-pa wia wuaimik. 
Eruwa-pa wia wuaiwkin i emeria apura yiena mua weria emeria nain-ke yia amapikiwep eka-pa yia yakuwap ekapemi koora-pa yia wuaimik. 
Yiena koora-pa yia wuaiwkin naap yiena koora-pa nan ikaimik. 
Koora-pa nan kaikap ikaimik.

Ikok ikok moma ikowa enem-ikok, siiwa kuisow. 
Siiwa kuisow nain okaiwi-pa kerereya yia maakimik, “Aria aakisa apura nain yakinok.” 
A yakimik. Yiena mua weria emeria-ke yookap eriwkin yakimik. 
Erap eka damola-pa yakiep sosora aimik, maabala suuwimik, kamukamu, kululuma, sagat. 
Fia yia awomiwkin ofa, irap owowa-pa, yiena koora-pa. 
Yiena koora-pa opap, maa piep, saasaria piep owowa uruma oraimik. 
Oromkun mua weria-ke meta urupa opap keema-muuna, umakuna meta yia mikimik.
Meta yia mikiwkin mera, waaya, oposia tiira nain yia aawomiwkin furunimik. 
Ama urupowa, ama orowa furunimik, koora ikiwimik. 
Koora ikiwimik nain mera eka me enimimik. 
Mera eka enowa marew nan ikaimik.
Mera eka enowa marew ikok ikok mua weria-ke maiya, “No aakisa maa ikina-ko enime.” 
Ne maa ikina enimimik, mera eka enimimik.

Mera eka enimimik nain i mangala me enimimik, waaya me enimimik, mua wia soopimik nain. 
A i sira, aria nepa me enimimik, aara me enimimik, maroka me enimimik, sibaur me enimimik. 
Mangala eka me enimimik. 
Emi kekanowa maneka, naap oram ikaimik. 
I ikoka nain enimiyen, mua kukusa-ke mia yia damoliwkin i eliwa me ikaiyen. 
Mia nigisiriyan, panewowa saarik ikaiyan.

I sira naap yiar ikua, i mua soopowa sira, i emeria sira-ke. 
Aria i emeria sira nan weeserek.

A pun naap, mua emeria umimik wi koora-pa nan ikaimik. 
Onak-ke moma wia ikem-ikomimik, maa mutmut, akia, iwoka. 
Wia ikomiwkin enem-ikaiwkin enem-ikaiwkin siiwa kuisow, koora-pa nan ikaimik nain. 
Siiwa kuisow ona onak-ke wiawi-ke maiya, “Aakisa yo muuka yakiya.” 

Mua weria ekapep koora-pa nan ikok muuka aaporaimik. 
Muuka porop periwkin yakiya. 
Eka-pa yakuwap pirimik. 
Pirami sira naap emeria saarik. 
Mera eka me ewur enimiya, nan ikaiya, eka-iw enem-ikainon. 
Maa eka-iw enep enep siiwa ara onaria, maa ikina, mangala wia, waaya wia, aara wia, urema wia, maroka wia, maa muutumut wia.
Ikoka ona mua weria-ke wia opomiwkin eniminon. 

A wi mua sira-ke, emeria soopowa sira, wi mua era-ke. 
Opaimika muut nanek. 
Weeserek.

\section{Tidal wave}\label{app:2:wave}
by Saror Aduna
\ea
\gll  Yo  aakisa  ifer  maneka  urup-i-ya  nain  ma-i-yem. \\
1s.\textsc{unm}  now  sea  big  ascend-Np-\textsc{pr}.3s  that1  say-Np-\textsc{pr}.1s \\
\glt ‘Now I tell about a tidal wave (lit: that when the big sea rises).’ \\
\z


\ea
\gll  Yo  me  baliwep  paayar-e-m,  oram  iperowa=ke      nanar-iwkin  miim-a-m. \\
1s.\textsc{unm}  not  well  understand-\textsc{pa}-1s  just  middle.aged-\textsc{cf} tell.story-2/3p.\textsc{ds}  hear-\textsc{pa}-1s \\


\glt ‘I do not understand it well, I have just heard the elders tell stories about it.’ \\
\z


\ea
\gll  Kiikir  akena  menat  maneka  goron-ep  ora-i-ya. \\
first  very  tide  big  go.down-\textsc{ss.seq}  descend-Np-\textsc{pr}.3s \\
\glt ‘First of all the big tide goes very low down.’ \\
\z


\ea
\gll  Goron-ep  ora-i-ya  nain,  mua  pepena=ke  menat=ke           ek-i-ya  na-ep  menat  ora-i-nan. \\
go.down-\textsc{ss.seq}  descend-Np-\textsc{pr}.3s  that1  man  ignorant-\textsc{cf}  tide-\textsc{cf}  go-Np-\textsc{pr}.3s  say/think-\textsc{ss.seq}  tide  descend-Np-\textsc{fu}.2s \\


\glt ‘When it goes low down, an ignorant man thinks that the ebb tide is receding and will go down (to catch shellfish).’ \\
\z


\ea
\gll  O  mua  amis-ar-owa  nain=ke  baurar-i-kuan. \\
\textsc{intj}  man  knowledge-\textsc{inch}  that-\textsc{cf}  flee-Np-\textsc{fu}.3p \\
\glt ‘(But) the knowledgeable people run away.’ \\
\z


\ea
\gll  Ae,  menat  maneka  goron-ep  ora-i-ya  nain  or-op,                malol. \\
yes  tide  big  go.down-\textsc{ss.seq}  descend-Np-\textsc{pr}.3s  that1  descend-\textsc{ss.seq}   deep.sea \\


\glt ‘Yes, the big tide that goes down, goes down (and reaches) the deep sea.’ \\
\z


\ea
\gll  Malol=pa  neeke  nainiw  suuw-urup-i-ya. \\
deep.sea-\textsc{loc}  there.\textsc{cf}  again  push-ascend-Np-\textsc{pr}.3s \\
\glt ‘From the deep sea there it pushes back up again.’ \\
\z


\ea
\gll  Suuw-urup-ep  urup-ep  owowa  erepura=pa  nan  nainiw         goron-ep  ora-i-ya. \\
push-ascend-\textsc{ss.seq}  ascend-\textsc{ss.seq}  village  side-\textsc{loc}  there  again  go.down-\textsc{ss.seq}  descend-Np-\textsc{pr}.3s \\


\glt ‘It pushes up and rises and from the (upper) side of the village it goes down again.’ \\
\z


\ea
\gll  Goron-ep  ora-i-ya  nain  maa  muutitik  iiwawun                 lalat-i-ya. \\
go.down-\textsc{ss.seq}  descend-Np-\textsc{pr}.3s  that1  thing  all.kinds  altogether  wash.away-Np-\textsc{pr}.3s \\


\glt ‘When it goes down it washes away everything.’ \\
\z


\ea
\gll  Koora=ki  e  maa  mauwa  nain  iiwawun  samor-i-ya. \\
house-\textsc{cf}.\textsc{qm}  or  thing  what  that1  altogether  bad-Np-\textsc{pr}.3s \\
\glt ‘It completely destroys houses or whatever.’ \\
\z


\ea
\gll  Ifera  suuw-urup-i-ya  nain,  nain  pun  nomona,  mera,  iiwawun        onaiya  urup-i-ya. \\
sea  push-ascend-Np-\textsc{pr}.3s  that1  that1  also  stone/coral  fish  altogether  with  ascend-Np-\textsc{pr}.3s \\


\glt ‘When the sea rises, it rises altogether with corals and fish too (i.e. a tidal wave brings up corals and fish with it too).’ \\
\z


\ea
\gll  Ne  mua  baurar-ep  ikiw-ep  koka=pa  ika-i-mik. \\
\textsc{add}  man  flee-\textsc{ss.seq}  go-\textsc{ss.seq}  jungle-\textsc{loc}  be-Np-\textsc{pr}.1/3p \\
\glt ‘And people run away and stay in the jungle.’ \\
\z


\ea
\gll  Aakisa  fan  ifera  goron-ep  or-eya  wi  owowa            ora-i-mik. \\
now  here  sea  go.down.\textsc{ss.seq}  descend-2/3s.\textsc{ds}  3p.\textsc{unm}  village descend-Np-\textsc{pr}.1/3p \\


\glt ‘Only (lit: just now) when the sea has receded they come down to the village.’ \\
\z


\ea
\gll  Owowa  or-op  owowa  kuuf-i-mik  na  owowa  iiwawun                  samor-ar-e-k,  ifera=ke  samor-a-k. \\
village  descend-\textsc{ss.seq}  village  look-Np-\textsc{pr}.1/3p  \textsc{tp}  village  altogether bad-\textsc{inch}-\textsc{pa}-3s  sea-\textsc{cf}  bad-\textsc{pa}-3s \\


\glt ‘They come down to the village and look at the village – it is completely destroyed; it is the sea that has destroyed it.’ \\
\z


\ea
\gll  Ne  ifera  suuw-i-ya  nain  koora  pun  iiwawun  mu-i-ya. \\
\textsc{add}  sea  push-Np-\textsc{pr}.3s  that1  house  also  altogether  swallow-Np-\textsc{pr}.3s \\
\glt ‘And when the sea pushes up it completely swallows the houses.’ \\
\z


\ea
\gll  O  no  mua  me  baurar-i-nan  nain  ikoka  ifera=ke  iiwawun      nefa  ifakim-i-non. \\
\textsc{intj}  2s.\textsc{unm}  man  not  flee-Np-\textsc{fu}.2s  that1  later  sea-\textsc{cf}  altogether 2s.\textsc{acc}  kill-Np-\textsc{fu}.3s \\


\glt ‘You who do not run away will later be completely killed by the sea.’ \\
\z


\ea
\gll  Ae,  opaimika  muut  nanek. \\
yes  talk  only  there-\textsc{pa}-3s \\
\glt ‘Yes, that is all the talk.’ \\
\z

Yo aakisa ifer maneka urupiya nain maiyem.

Yo me baliwep paayarem, oram iperowa-ke nanariwkin miimam. 
Kiikir akena menat maneka goronep oraiya. 
Goronep oraiya nain mua pepena-ke menat-ke ekiya naep menat orainan. 
O mua amisarowa nain-ke baurarikuan. 
Ae, menat maneka goronep oraiya nain orop, malol. 
Malol-pa neeke nainiw suuw-urupiya. 
Suuw-urupep urupep owowa erepura-pa nan nainiw goronep oraiya.
Goronep oraiya nain maa muutitik iiwawun lalatiya. 
Koora-ki e maa mauwa nain iiwawun samoriya.

Ifera suuw-urupiya nain nain pun nomona, mera, iiwawun onaiya urupiya.
Ne mua baurarep  ikiwep  koka-pa ikaimik. 
Aakisa fan ifera goronep oreya  wi owowa oraimik. 
Owowa orop owowa kuufimik na owowa iiwawun samorarek, ifera-ke samorak. 
Ne ifera suuwiya nain koora pun iiwawun muiya. 
O no mua me baurarinan nain ikoka ifera-ke iiwawun nefa ifakiminon. 
\textrm{Ae, opaimika muut nanek.}